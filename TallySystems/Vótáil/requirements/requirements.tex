%% Votail Cuntais - Functional Requirements Specification
%%

\documentclass{usenixsubmit}
\usepackage{times}
\usepackage[margin=1in]{geometry}
\usepackage[list]{todo}
\usepackage{ifpdf}

\ifpdf
 \usepackage[pdftex]{graphicx}
\else
 \usepackage{graphicx}
\fi

\usepackage{xspace}
\usepackage{tabularx}
\usepackage{epsfig}
\usepackage{amsmath}
\usepackage{amsfonts}
\usepackage{amssymb}
\usepackage{eucal}
\usepackage{float}
\usepackage{times}

\ifpdf
 \usepackage[pdftex,bookmarks=false,a4paper=false,
            plainpages=false,naturalnames=true,
            colorlinks=true,pdfstartview=FitV,
            linkcolor=blue,citecolor=blue,urlcolor=blue,
            pdfauthor="Dermot Cochran and Joseph R. Kiniry"]{hyperref}
\else
 \usepackage[dvips,linkcolor=blue,citecolor=blue,urlcolor=blue]{hyperref}
\fi

\usepackage{listings}
\include{bon} % BON stylings

% Consistent use of e.g., i.e., etc.
\newcommand{\eg}{e.g.,\xspace}
\newcommand{\ie}{i.e.,\xspace}
\newcommand{\etc}{etc.\xspace}
\def\votail{V\'{o}t\'{a}il\xspace}
\def\eireann{\'{E}ireann\xspace}
\def\dail{D\'{a}il\xspace}

\newcommand{\myhref}[2]{\href{#1}{#2}} % \myhref as href
% \newcommand{\myhref}[2]{#1\footnote{#2}} % \myhref as footnote
\newcommand{\hreffootnote}[3]{\href{#1}{#2}\footnote{#3 \href{#1}{#1}}}

\newcommand{\notef}[1]{\xspace$\textcolor{red}{\dagger^\textsf{fintan}}$\marginpar{\scriptsize\textsf{Fintan:} #1}}
\newcommand{\notej}[1]{\xspace$\frac{\varocircle}{\textsf{jk}}$\marginpar{\scriptsize\textsf{Joe:} #1}}

\begin{document}

\date{}
\title{Software Verification of Ballot Counting in Elections by Single Transferable Vote}
\author{
Dermot Cochran\\
LERO Graduate School of Software Engineering\\
School of Computer Science and Informatics\\
University College Dublin, Ireland\\
dermot.cochran@ucd.ie\\
\and
Joseph R. Kiniry\\
Complex Adaptive Systems Laboratory\\
School of Computer Science and Informatics\\
University College Dublin, Ireland\\
kiniry@ucd.ie}

\maketitle

%===========
\begin{abstract}

Electronic voting machines are no longer used in the Republic of Ireland.  The decision to stop using e-voting was based on several different factors, 
including an inability to verify the accuracy of the proprietary vote counting software.  \votail, is a formal specification and open source implementation of 
Ireland's Proportional Representation by Single Transferable Vote (PR-STV) ballot counting sub-system, which meets the requirements laid down by Ireland's former 
Commission on Electronic Voting.  This paper describes the dependable software engineering process used to implement the verified system including formalization.

\end{abstract}

%===============
\section{Introduction}
{V}{\'{o}t\'{a}il} is the Irish Gaelic word for Voting.  The Irish government is currently considering how it can save costs by disposing of its current 
generation of electronic voting machines.  The decision to stop using electronic voting was due to mixture of technical issues with regard to the actual voting 
machines in additional to more general concerns about the security of electronic voting.  The current political consensus is that electronic voting (in its 
current form) will never be used in the Republic of Ireland. 
 
Note that Irish legislation uses the term 'vote' to mean the contents of a ballot paper ~\cite{CEV00}.

\begin{quote}
SECTION 16 : MIXING, NUMBERING AND TRANSFER OF INDIVIDUAL VOTES
For the purpose of clarity, �vote� means the full set of candidate preferences recorded by a voter at an election.
\end{quote}

Oireachtas \eireann (the national parliament of Ireland) has two chambers of which the dominant \dail \eireann (Irish house of representatives or lower house) is 
directly elected by the people for a term of up to five years by a quota-based single transferable vote system in multi-seat constituencies.  

The Republic of Ireland uses Proportional Representation by Single Transferable Vote (PR-STV) for its national, local and european elections.  
PR-STV is similar to Instant Runoff Voting (IRV) in the United States ~\cite{FairVote}, and to BC-STV in Canada ~\cite{BCSTV}.  
It is rapidly gaining popularity but has been criticized by proponents of Median Score Voting ~\cite{Unger07}.  

The political significance of lost, corrupted or altered votes depends on the type of voting system (\eg STV) and the closeness of the election race. In PR-STV, 
it is not unusual to see the final seat in a multi-winner constituency determined in the last round of counting by a small number of votes.  

Manual recounts are often called for closely contested seats, as the results often vary slightly, indicating small errors in the manual process of counting votes.  
Paper based voting with counting by hand is popular in Ireland, and recent attempts at automation were frustrated by subtle logic errors in the vote counting 
software.  The logic errors exist, in part, due to the complexities and idiosyncrasies of PR-STV with regard to tie breaking and especially the rounding of vote 
transfers.  

Referenda to introduce plurality voting were rejected twice by the Irish electorate, in 1959 and again in 1968 ~\cite{VoteRef}.  Since then there have been no 
further legislative proposals to change the voting scheme used in Ireland. 

PR-STV has the disadvantage that ballots must be counted centrally in each constituency, rather than at the polling places. The creates an additional burden for 
the physical security and transportation of ballot papers.  It could be argued that some form of optical scan at the polling place would help to verify that the 
ballot papers leaving the polling place were the same as those arriving at the counting hall.
In the event that ballot papers were scanned as an additional security measure, it would be useful to count the ballot papers electronically as an additional 
check on the result of the manual count, rather than as a substiute for manual counting of ballots.

We need to put the concerns about the security of electronic voting in perspective. 
It would seem that voting systems require stronger security assurance than for example electronic commerce, yet cannot be considered to be as safety critical as medical devices or airplane control systems.  Yes, a failure of election technology could lead to the wrong selection of a candidate for high office, which is not necessarily the same as direct injury or loss of life, but is certainly more dangerous than small scale financial loss or risk in e-commerce or online banking.  In other words, a plane crash is obviously worse than electing the wrong candidate to high office, since the latter remains a subjective loss (and also begs the question of which voting system is best e.g. score voting), whereas the former is clearly unacceptable.

The tradeoffs between paper-based and electronic vote counting have discussed elsewhere, and that remains an open question.  
In Ireland, however, it was the failure of the vote counting subsystem that brought Electronic Voting into disrepute.  
The final report of Ireland's former Commission on Electronic Voting noted inaccuracies in the vote counting software, 
rather than the design of the voting machines as the principal cause of concern.

The following are selected quotes from the CEV report on the previous electronic voting system used in Ireland ~\cite{CEV2ndReport}:

\begin{itemize}
\item Design weaknesses,
including an error in the implementation of the count rules that could compromise the accuracy of
an election, have been identified and these have reduced the Commission's confidence in this
software.

\item The achievement of the full potential of the chosen system in terms of secrecy and accuracy
depends upon a number of software and hardware modifications, both major and minor, and
more significantly, is dependent on the reliability of its software being adequately proven.

\item Taking account of the ease and relative cost of making some of these modifications, the potential
advantages of the chosen system, once modified in accordance with the Commission's
recommendations, can make it a viable alternative to the existing paper system in terms of secrecy
and accuracy.
\end{itemize}

Cryptographers and security experts naturally tend to focus on the risks of electronic ballot casting, but accuracy in ballot counting is just as fundamental, if not even more important.  Manual paper-based counting is perceived as being more secure, but is potentially more error prone than electronic counting.  This paper addresses the software engineering challenges of non-trivial vote counting, using Irish PR-STV as the main example.  Note that the proposal to place anonymous ballots on a public bulletin board, for public counting, cannot be achieved with PR-STV due to the risk of vote signing~\cite{TeagueEtAl08}.

\section{Formal Specification of PR-STV}

Ireland's Commission on Electronic Voting laid down several recommendations for future use of electronic voting, including the following:
\begin{itemize}
\item clear definition of requirements and specifications,
\item robust and formal approach to design and development,
\item separation of critical concerns (election management, count rules, vote file, \etc),
\item the appropriate use of open source methods,
\item publication or public inspection of the source code,
\item open public testing of the vote recording software and the vote counting software via an on-line web interface designed to simulate the hardware interfaces of the 
system, and
\item full and formal process of requirements capture and functional specifications for any proposed new system.
\end{itemize}

This leads us to conclude that the next generation of electronic voting systems in Ireland (if any), will be developed in open source using formal methods, 
and that each functional module e.g. vote counting will be developed and tested independently.  
In particular it should be possible to develop two or more different implementations of vote counting, yet to get the same results. 
\votail is intended to be a reference implementation, formally guaranteed to give the correct results for any valid set of ballots.

Also, it will be required that each major subsystem will be implementated seperately and that any subsystem could be replaced by an equivalent subsystem.

The use of electronic counting would permit the use of fractional transfers and avoid the need for random sub-selection of ballots when distributing a surplus; 
this would help to ensure that the result is closer to the intent of the electorate and avoid the use of drawing lots to resolve ties.  
This specification allows for both the traditional method of counting i.e. simulation of  the manual counting process and for the use of fractional ballots - 
similar to the Seanad electoral system (Gregory Method) in which there are approximately 1500 electors. Note that with manual vote counting it is only possible to support
fractional vote transfers with a small electorate; electronic vote counting would have allowed this to be achieved for the whole electorate, but the previous 
(decomissioned)
evoting system did not allow for this possibility.

The \votail specification describes only the vote counting process and does not rely on assumptions about how the votes were cast, transmitted, encrypted or stored.

\subsection{Methodology}

There are three important and distinct stages in a verification-centric developement process ~\cite{EVT07}.

\begin{enumerate}
\item Table of Functional Requirements, "State Machine" Diagram and Informal (Extended) Business Object Notation
\item Java Modeling Language Specification (could have been VDM++ if working with C++ for example)
\item Java 1.4 Code, Unit Tests and Bytecode Verification
\end{enumeration}

The functional requirements are expressed using extended Business Object Notation (eBON), supplemented with digrams and tables.
BON ~\cite{EBON} is (programming) language indepedant. The formal level of BON is not used because it overlaps with with JML ~\cite{JML}.  
Vienna Development (Method) Language (VDM) ~\cite{VDM} could also have been used but has less tool support e.g. not richly integrated with Eclipse and Bytecode 
verification as with MOBIUS and JML.  Neither method/language allows full expression of the mathematical relationships and calculations used.  An interesting alternative would have been to use
Fortress porgramming language which supports formalisation of mathematical and scientific calculations ~\cite{Fortress}.
Model Checkers such as Bogor ~\cite{Bogor} were not used because we want to be able to verify (and eventually certify using MOBIUS) the actual software and not so 
much the underlying model \ie
we know that voting works on paper and that the PR-STV algorithm terminates in a finite time \etc.  Model checking might have been useful if we had concerns about a new 
method of voting
e.g. median score voting (a.k.a. range voting) and wanted to check if the model was sound (although apparently honeybees have been using it for a long time ~\cite{SmithRangeVoting}).

Note that the specification does not rely on any features of Java 1.5 or Java 1.6, so the implementation was made in Java 1.4 source code so as to fully exploit the 
existing mature JML toolset e.g. ESCJava2. ~\cite{ESCJava2} rather than the newer JML4 plugin with supports Java 1.5 or OpenJML which supports Java 1.6.

Also we currently have a JML specification for JDK 1.4 but not JDK 1.5 or upwards, so as to allow modular verification with regard to the Java 1.4 API - note that we do
not verify the JDK but treat it as trusted third party - this is dangerous so we seek to minimise our reliance on the JDK by adding a layer of indirection where required 
and to use primitive data structures e.g. arrays where possible, rather than making use of Java collections, for example.

%---------------------------------------------
\subsection{Requirements Analysis}

\paragraph{Constitutional Law}

Article 16, section 2, subsection 5 of the Irish Consitition states only that:

\begin{quote}
The members shall be elected on the system of proportional representation by means of the single transferable vote.
\end{quote}

which leaves considerable freedom for the electoral law to define the precise means of vote casting and counting.

\paragraph{Electoral Law}

The functional requirements are read from the electoral legislation and CEV guidelines and listed in a table ~\cite{Cochran06} e.g.

\begin{itemize}
\item 1	Counting does not begin until all votes are loaded.
\item 2	The total number of first preference votes must remain the same after each count.
\item 3	The (Droop) quota is equal to ((Number of valid votes cast)/(Number of seats being filled + 1)) + 1, ignoring any remainder.
\item 4	Any candidate with a quota or more than a quota of votes, is deemed to be elected.
\item 5	The surplus of an elected candidate is the difference between the quota and his/her total number of votes.
\item 6	The minimum number of votes required for a candidate to secure the return of his/her deposit is one plus one-quarter of a quota based on the total number of 
seats in the constituency.
\item 7	Any elected candidate automatically saves his/her deposit.
\end{itemize}

The full table of functional requirements have been listed elsewhere ~\cite{Cochran06}.

In addition Sections 32-37 relate to the format of the election report:

\begin{quote}
ELECTRONIC COUNTING
POST-COUNT MENU
SECTION 32 : RESULT SHEET IN WORD FORMAT
When clicked, the �Result sheet in Word format� should convert the Post-Count Result
sheet described in section 32 into a Word document in identical format ready for printing
on A4 and larger page sizes. This document should be stored in the system and via
�Explorer�, be capable of being printed and exported to a floppy disk or CD.
\end{quote}

We have chosen not to implement the above requirement, but instead to provide the results in a platform-neutral format e.g. 
XML which can be processed by a seperate document generation system to produce whichever format of document is required. It is important to the note that
the process of translating legal or "business requirements" (as specified by a business analyst or equivalent) into functional requirements needs to be guided by the
principles of good clean software design and architecture, so to remove unnecessary detail, to resolve ambiguities and to discover hidden assumptions which need
to be made explicit and then clarified.  Thus the process of functional analysis and design needs to be owned by the software engineer.

Each of the functional requirements is cross-referenced to the source document by section number, subsection and page number, so that all requirements are traceable.  
The requirements are listed in order of appearance in each source document so that it can be easily seen if any are missing.  
Not all parts of the legislation translate directly into requirements e.g. the requirement to delete invalid votes from the system:

Foor example Section Department of the Environment Commentary on Count Rules ~\cite{CEV00} states that:
\begin{quote}
The system should have a facility to remove from the �Votes as mixed and numbered� table a vote
(notably one cast on a voting machine from a postal or special ballot paper) which the court rules is
invalid and should not be included in the count. To do this, there should be a button in the Petitions
menu entitled �Deletion of invalid vote� which when pressed presents a table with the same column
headings as in the �Votes as mixed and numbered� table and one blank row beneath it. The user enters
in that row the preferences recorded on the vote to be removed from the table (he/she she is not
permitted to enter anything in the Mixed Votes No. column). The user then presses a �Find invalid
vote� button and the system searches the �Votes as mixed and numbered� table, finds the vote in
question and enters its Mixed Vote No. in the Mixed Votes No. column of the table where the user
entered the characteristics of the invalid vote. There should also be a �Remove invalid vote� button
which when pressed by the user would, where there is only one vote in the �Votes as mixed and
numbered� table with the characteristics of the invalid vote, remove that vote from the table and
decrease by one the mixed vote numbers of all votes which succeeded it in the original table.
\end{quote}

This requirement is not listed, but instead is treated as a \emph{precondition} to the vote counting process; 
it is required that the input to the vote counting sub-system contains only valid ballots.  
It is assumed that other parts of the system e.g. the vote-casting and election management system have provided us with the valid set of votes, 
and of course, we make no assumptions about how the ballots were cast, whether electronic or paper-based, at the polling place or remotely.  
This is part of the modular design by contract.  The software fulfills the following contract; 
if the input is a valid set of votes, we will generate a valid set of results according to the requirements 1992 Electoral Act, \etc.  
Also, if the local jurisdiction limits the number of preferences allowed e.g. North Carolina, then this can be enforced during the casting of ballots e.g. 
if all ballots contain only one preference, then STV reduces to plurality (first-past-the-post) voting.

%---------------------------------------------------------------------
\subsection{Architectural Description and Elaboration}

These semi-formal statements are then elaborated in an software architectural description language (ADL).  
This provides a precise formal specification which can be refined into a detailed design and yet is readable by non-programmers.  
We use an ADL called ``Extended BON'' (or EBON for short) to
describe our system at this early stage.  Business Object Notation (BON) is a method and graphical notation for high-level object-oriented analysis and design.
EBON is a system specification language akin to UML that differs in several key respects.  Contrary to UML, BON has a relatively small number of
charts and diagrams, a simple semantics, a graphical and human-readable textual syntax, and it is easy to maintain consistency
between a BON specification and a concrete realization of that specification.

The key constructs in EBON that we use in this work are class dictionaries and several kinds of informal charts, (system, cluster,
class, creation, and event charts), which describe concepts using structured natural language.

\paragraph{Class Dictionaries.}

In BON, domain constructs are known as \emph{classes}, not in a concrete object-oriented sense, but rather ``classifiers'' in a
generic sense.  In our process, a name and a single sentence definition is agreed upon for each class \emph{completely
  independently of design and implementation considerations}.  When writing a definition one uses only classes that are defined in the
model under analysis or classes from models on which the domain depends.

\paragraph{Informal Charts.}

A system chart enumerates a system's purpose and
major subsystems (clusters, in the BON vernacular), a cluster chart explains the purpose of a subsystem and, in turn, what subsystems and
classes it contains, and event charts identify important external and internal events (state transitions) in a subsystem.

Each class is described, again in English, using a class chart.  A
class chart contains four sections: a \emph{description},
\emph{queries}, \emph{commands}, and \emph{constraints}.  Queries and
commands are, collectively, known as features.

The description of a class is an expanded version of the
aforementioned class's definition.  Within our process, the first
sentence of the of the description is the definition, thus the formal
refinement between the description and the definition is simply a
prefix.

Queries are operations that do not change the class's state, whereas
commands are operations that do change a class's state.  Constraints
are restrictions on a class's state and behavior.

An example of a query is:

'Has this candidate reached a quota (or more) of votes?'
 
We also must describe how classes relate to one another with respect
to how classes come into being.  A creation chart summarizes these
relationships.  With respect to a verification-centric approach, our
goal in defining such relations is to minimize the exposure of a
class's creation and to identify ownership relations between classes.
The former goal reduces the amount of reasoning we must do about
object allocation and aliasing and the latter supports the use of
ownership type systems to support modular reasoning.

The high level behavior of the ballot counting system is expressed as follows:

\paragraph{Java Modeling Language}

The Java Modeling Language (JML) is a behavioral interface specification language that can be 
used to specify the behavior of Java modules. It combines the design by contract approach of 
Eiffel and the model-based specification approach of the Larch family of interface specification 
languages, with some elements of the refinement calculus.

\paragraph{Bytecode Modeling Language}

The Bytecode Modeling Language (BML) is similar to JML except that BML annotations are embedded directly into the 
Java Bytecode. This technique enables programmers to include their specifications directly in the compiled programs.

This is important since it allows the bytecode to be verified and not just the source code.  In particular, this
helps to detect an attempt to manipulate the bytecode.

The Mobius Program Verification Environment (PVE), contains for verification of both JML and BML, and thus is our preferred verification tool.

\subsection{Mathematical Representation}

The following mathematical objects were identified in the system:
\begin{itemize}
\item The Ballot Box
\item The Ballot Paper
\item The Candidate Identifier
\item The Candidate List
\item The Distribution of Ballot Papers between Candidates
\item The Fractional Ballot
\item The Allocation of Seats
\end{itemize}

with the following mathematical operators
\begin{itemize}
\item Transfer
\item Eliminate
\item Exclude
\item Reweight
\end{itemize}

and functions
\begin{itemize}
\item Vote
\end{itemize}


\paragraph{Candidate Identifier}

A candidate identifier ID has the property that for any two different candidates A and B; ID (A)  equals ID (A) and ID (A) is not the same as ID (B).  The identifier for each candidate is immutable - ID (A) always remains constant.

\paragraph{Candidate List}

An ordered set of pairs; each pair contains a Candidate Identifier and a Candidate Status.  Each Candidate Identifier appears only once.  The Candidate Status is one of the four values Elected, Continuing, Excluded But Saved Deposit or Lost Deposit.

\paragraph{Ballot Paper}

The ballot paper is modeled as an ordered list of Candidate Identifiers.

\paragraph{Ballot Box}

The ballot box is modeled as an ordered list of Ballot Papers.

\paragraph{Ballot Distribution}

The Ballot Distribution is a mapping from the Ballot Box to the Candidate List - it shows which ballot paper is currently assigned to elect candidate.  Initially each ballot paper is assigned to its first preference.

\subsection{Design by Contract}

\paragraph{Overall Preconditions}

The vote count system accepts the following inputs:
\begin{itemize}
\item The number of seats to be filled in this constituency: must be either three, four or five
\item An unordered list of sets of preferences; each set of preferences is an ordered list of
  candidates
\end{itemize}

\paragraph{Overall Postconditions}
The vote count system will generate the following information
\begin{itemize}
\item The quota
\item The deposit saving threshold
\item The list of winning candidates
\item The list of excluded candidates whose deposit was saved
\item The number of rounds of counting
\item The number of votes for each candidate at each count
\item The count at which each candidate was either elected or eliminated
\end{itemize}

\subsection{The Electoral System}

\paragraph{1992 Electoral Act, Section 37}

\begin{itemize}
\item (1) A \dail election shall be conducted in accordance with this Act and,
in case a \dail election is contested, the poll shall be taken according to the
principle of proportional representation, each elector having one transferable
vote.
\item (2) In this section "transferable vote" means a vote which is�
( a ) capable of being given so as to indicate the voter's preference for
the candidates in order, and
( b ) capable of being transferred to the next choice when the vote is
not required to give a prior choice the necessary quota of votes, or
when, owing to the deficiency in the number of the votes given for a
prior choice, that choice is excluded from the list of candidates.
\end{itemize}

\paragraph{Formal Assertions}
Section 37 contains very high level description of the voting system, which is elaborated in more detail by later sections of the act e.g. sections 120-122.  Nevertheless we can immediately derive the following requirements:

%\paragraph{EBON description}

\paragraph{The JML Specification}

\lstset{style={bon},language={java}}
\begin{lstlisting}
{
/**
 * Gets the next preference continuing candidate.
 * 
 * @design This is the nearest next preference i.e.
 * filter the list of preferences to contain continuing candidates and then
 * get the next preference to a continuing candidate, if any.
 * 
 * @param ballot Ballot paper from which to get the next preference
 * 
 * @return Candidate ID of next continuing candidate or NONTRANSFERABLE
 */
/*@ requires state == COUNTING;
  @ ensures (\result == Ballot.NONTRANSFERABLE) <=!=>
  @   (\exists int k; 1 <= k && k <= ballot.remainingPreferences(); 
  @     (\result == ballot.getNextPreference(k)) &&
  @     (\forall int i; 1 <= i && i < k;
  @       isContinuingCandidateID(ballot.getNextPreference(i)) == false)
  @   );
  @*/
}
\end{lstlisting}

\paragraph{The annotated Java code}

\begin{figure}[htbp]
\begin{center}
\epsfile{file={OuterStateMachine.eps},scale=0.8}
\caption{{\bf default}}
\label{default}
\end{center}
\end{figure}


\subsection{The Droop Quota}

\paragraph{1992 Electoral Act, Section 120}

\begin{quote}
(1) The returning officer shall then divide the number of all valid
papers by a number exceeding by one the number of vacancies to be filled; the
result increased by one, any fractional remainder being disregarded, shall be
the number of votes sufficient to secure the election of a candidate and this
number is referred to in this Act as "the quota".
(2) Where at the end of any count the number of votes credited to a
candidate is equal to or greater than the quota, that candidate shall be deemed
to be elected.
\end{quote}

\paragraph{Formal Assertions}


\paragraph{The BON Description}

\paragraph{The JML Specification}

\begin{figure}
\footnotesize
%\begin{verbatim}
%/**
% * Minimum number of votes needed to 
% * guarantee an election.
% */
% //@ public model long quota;
% //@ public invariant 0 <= quota;
% //@ public invariant quota <= totalVotes;
% /**
%  * @see requirement 3, section 3, 
%  *      item 3, page 13
%  */
% /*@ public invariant (PRECOUNT < state) ==> 
%   @  quota == 1 + (totalVotes / (seats + 1));
%   @*/
%\end{verbatim}
\normalsize
\caption{JML invariant for calculating a quota.}
\label{fig:JMLSimpleExample}
\end{figure}


\paragraph{The annotated Java code}

\subsection{Transfer of Surpluses}


\paragraph{1992 Electoral Act, Section 121}

\begin{itemize}
\item (1) Where at the end of any count the number of votes credited to a
candidate is greater than the quota, the surplus shall be transferred in
accordance with and subject to the provisions of this section to the continuing
candidate or candidates indicated on the ballot papers in the parcel or
sub-parcel of the candidate deemed to be elected according to the next available
preferences recorded thereon.
\item (2) Where the votes credited to a candidate deemed to be elected whose
surplus is to be transferred consist of original votes only, the returning officer
shall examine all the papers in the parcel of that candidate and shall arrange
the transferable papers in sub-parcels according to the next available
preferences recorded thereon.
\item (3) Where the votes credited to a candidate deemed to be elected whose
surplus is to be transferred consist of original and transferred votes, or of
transferred votes only, the returning officer shall examine the papers contained
in the sub-parcel last received by that candidate and shall arrange the
transferable papers therein in further sub-parcels according to the next
available preferences recorded thereon.
\item (4) In either of the cases referred to in subsections (2) and (3) the returning
officer shall make a separate sub-parcel of the nontransferable papers and shall
ascertain the number of papers in each sub-parcel of transferable papers and in
the sub-parcel of nontransferable papers.
\item (5) Where
( a ) the surplus is equal to the total number of papers in the
sub-parcels of transferable papers, the returning officer shall transfer
each sub-parcel of transferable papers to the continuing candidate
indicated thereon as the voters' next available preference,
( b ) the surplus is greater than the total number of papers in the
sub-parcels of transferable papers, the returning officer shall proceed as
specified in paragraph (a) and shall in addition make a sub-parcel of a
number of non-transferable papers equal to the difference between such
total number and the surplus and set the papers therein aside as finally
dealt with, such papers being, for the purposes of section 127 (1),
described as non-transferable papers not effective,
and the non-transferable papers or the remaining non-transferable papers, as
the case may be, also arranged as a sub-parcel shall be placed with the papers
of the candidate deemed to be elected.
\item (6) Where the surplus is less than the total number of transferable papers
the following provisions shall apply:
( a ) the returning officer shall transfer from each sub-parcel of
transferable papers to the continuing candidate indicated thereon as the
voters' next available preference that number of papers which bears the
same proportion to the number of papers in the sub-parcel as the
surplus bears to the total number of transferable papers,
( b ) the number of papers to be transferred from each sub-parcel shall
be ascertained by multiplying the number of papers in the sub-parcel
by the surplus and dividing the result by the total number of
transferable papers,
( c ) a note shall be made of the fractions (which may be expressed in
decimal form), if any, in each quotient ascertained in respect of each
candidate in accordance with paragraph (b) and if, owing to the
existence of such fractions, the number of papers to be transferred is
less than the surplus, so many of these fractions taken in the order of
their magnitude (beginning with the largest) as are necessary to make
the total number of papers to be transferred equal to the surplus shall
be reckoned as of the value of unity and the remaining fractions shall
be disregarded,
( d ) where two or more fractions are of equal magnitude, and it is not
possible for the purposes of paragraph (c) to reckon all of the said
fractions as of the value of unity, that fraction shall be deemed to be
the largest which arises from the largest sub-parcel, and if such
sub-parcels are equal in size, that fraction shall be deemed to be the
largest which relates to the candidate credited with the largest number
of original votes. Where the numbers of such original votes are equal,
regard shall be had to the total number of votes credited to such
candidates at the first count at which they were credited with an unequal
number of votes, and the fraction relating to the candidate credited with
the greatest number of votes at that count shall be deemed to be the
largest. Where the numbers of votes credited to such candidates were
equal at all counts the returning officer shall determine by lot which
fraction shall be deemed to be the largest.
\item (7) The papers to be transferred from each sub-parcel shall be those last
filed in the sub-parcel, and each paper so transferred shall be marked to
indicate the number of the count at which the transfer took place.
\item (8) The returning officer need not necessarily transfer the surplus of a
candidate deemed to be elected whenever that surplus, together with any other
surplus not transferred, is less than both the difference between the quota and
the number of votes credited to the highest continuing candidate and the
difference between the numbers of the votes credited to the two lowest
continuing candidates and either
( a ) the number of votes credited to the lowest candidate is greater than
one quarter of the quota, or
( b ) the sum of the number of votes credited to the lowest candidate
together with that surplus and any other surplus not transferred is not
greater than one quarter of the quota.
\item (9) Where at any time there are two or more surpluses which are to be
transferred, the greater or greatest of such surpluses shall first be transferred
and the remaining surplus or surpluses shall then, subject to subsection (8),
be transferred in the order of their descending magnitude.
\item (10) Where two or more candidates have each an equal surplus arising
from the same count regard shall be had to the number of original votes
credited to each candidate and the surplus of the candidate credited with the
largest number of original votes shall be first dealt with. Where the numbers
of such original votes are equal regard shall be had to the total number of
votes credited to such candidates at the first count at which they had an
unequal number of votes and the surplus of the candidate with the greatest
number of votes at that count shall be first dealt with. Where the numbers of
votes credited to such candidates were equal at all counts the returning officer
shall determine by lot which surplus he will first deal with.
\item (11) Subject to subsections (8) and (9), where two or more candidates have
a surplus arising from different counts, a surplus which arises at the end of
any count shall be transferred before a surplus which arises at a subsequent
count.
\end{itemize}

\paragraph{Analysis}
This section of the Electoral Act is presented here to show the complexities of the legal requirements.  For example, paper-based ballots cannot be treated as fractions, so rounding of votes is needed in the case where the surplus is less than
the number of transferable votes (votes with continuing preferences), then the subset of votes to be transfered must be chosen in a representative way.
\begin{itemize}
\item
\item
\end{itemize}


\paragraph{The BON Description}

\paragraph{The JML Specification}

%\begin{figure*}[t]
% {\footnotesize
% \begin{verbatim}
%   /** 
%    * Distribute the surplus votes.
%    * 
%    * @param candidateWithSurplus the candidate whose surplus is to be distributed.
%    * @design The highest surplus must be distributed if the total surplus could 
%    *         save the deposit of a candidate or change the relative position of
%    *         the two lowest continuing candidates, or would be enough to elect the
%    *         highest continuing candidate.
%    * @see requirements 14-18, section 5, item 2, page 18
%    * @see requirement 8, section 4, item 2, page 15    */
%   /*@ requires getSurplus(candidateWithSurplus) > 0;
%     @ requires state == COUNTING;
%     @ requires numberOfContinuingCandidates > remainingSeats;
%     @ requires (numberOfContinuingCandidates > remainingSeats + 1) ||
%     @          (sumOfSurpluses + lowestContinuingVote > nextHighestVote) ||
%     @          (numberOfEqualLowestContinuing > 1);
%     @ requires remainingSeats > 0;
%     @ requires (remainingSeats > 1) ||
%     @          ((highestContinuingVote < 
%     @            sumOfOtherContinuingVotes + sumOfSurpluses) &&
%     @          (numberOfEqualHighestContinuing == 1));
%     @ requires getSurplus (candidateWithSurplus) == highestSurplus;
%     @ requires (sumOfSurpluses + highestContinuingVote >= quota) ||
%     @   (sumOfSurpluses + lowestContinuingVote > nextHighestVote) ||
%     @   (numberOfEqualLowestContinuing > 1) ||
%     @   ((sumOfSurpluses + lowestContinuingVote >= depositSavingThreshold) &&
%     @      (lowestContinuingVote < depositSavingThreshold));
%     @ ensures getSurplus (candidateWithSurplus) == 0;
%     @*/
%   /** @see requirement 9, section 4, item 3, page 16 */
%   /*@ ensures countNumber == \old (countNumber) + 1;
%     @ ensures (state == COUNTING) || (state == FINISHED);
%     @*/
%   /** @see requirement 2, section 3, item 3, page 12 */
%   /*@
%     @ ensures totalVotes == nonTransferableVotes + 
%     @   (\sum int i; 0 <= i && i < totalCandidates; 
%     @   candidateList[i].getTotalVote());
%     @*/
%   protected void distributeSurplus(/*@ non_null @*/ election.tally.Candidate candidateWithSurplus);
% \end{verbatim}
% }
% \caption{JML Specification of the \texttt{distributeSurplus} method.}
% \label{fig:JMLModerateExample}
% \end{figure*}                           


\paragraph{The annotated Java code}

\subsection{Order of Elimination}

\paragraph{1992 Electoral Act, Section 122}

\begin {itemize}
\item (1) If at any time no candidate has a surplus (or when under section
121 (8) an existing surplus is not transferred) and one or more vacancies
remain unfilled, the returning officer shall
( a ) exclude the candidate credited with the lowest number of votes and
examine all the papers of that candidate;
( b ) arrange the transferable papers in sub-parcels according to the next
available preferences recorded thereon for continuing candidates and
transfer each sub-parcel to the candidate for whom the preference is
recorded;
( c ) make a separate sub-parcel of the non-transferable papers and set
them aside as finally dealt with, such papers being, for the purposes of
section 127 (1), described as nontransferable papers not effective.
\item (2) Where the total of the votes of the two or more lowest candidates
together with any surplus not transferred is less than the number of votes
credited to the next highest candidate, the returning officer shall in one
operation exclude such two or more lowest candidates provided that
( a ) the number of votes credited to the second lowest candidate is
greater than one quarter of the quota, or
( b ) where the number of votes credited to any one of such two or
more lowest candidates does not exceed one quarter of the quota, it is
clear that the exclusion of the candidates separately in accordance with
subsection (1) and the transfer of any untransferred surplus could not
result in a number of votes exceeding one quarter of the quota being
credited to any such candidate.
\item (3) If, when a candidate has to be excluded under this section, two or more
candidates have each the same number of votes and are lowest, regard shall be
had to the number of original votes credited to each of those candidates, and
the candidate with the lowest number of original votes shall be excluded and
where the numbers of the original votes are equal, regard shall be had to the
total numbers of votes credited to those candidates at the first count at which
they had an unequal number of votes and the candidate with the lowest
number of votes at that count shall be excluded and, where the numbers of
votes credited to those candidates were equal at all counts, the returning officer
shall determine by lot which shall be excluded.
\end {itemize}

%
%\paragraph{Formal Assertions}

%\paragraph{The BON Description}

%\paragraph{The JML Specification}

%
%\begin{verbatim}
% /** 
%   * Elimination of one or more candidates and transfer of votes
%   * 
%   * @param candidate This candidate
%   * 
%   * @design More than one candidate could be eliminated in the same round if the
%   *   combined vote held by the group is not enough to elect a candidate or to
%   *   save a deposit, and if there are no surplus votes available for distribution.
%   * 
%   * @see requirement 10, section 4, item 4, page 16
%   * @see requirement 12, section 4, item 4, page 16
%   */
%  /*@ requires 1 <= numberToEliminate;
%    @ requires numberToEliminate <= numberOfContinuingCandidates; 
%    @ requires (\forall int i;
%    @          0 <= i && i < numberToEliminate;
%    @          candidatesToEliminate[i].getTotalVote() == 0 ||
%    @          depositSavingThreshold <= candidatesToEliminate[i].getTotalVote() ||
%    @          candidatesToEliminate[i].getTotalVote() + 
%    @          sumOfSurpluses + (\sum int j;
%    @            0 <= j && j != i && j < numberToEliminate;
%    @            candidatesToEliminate[i].getTotalVote()) < depositSavingThreshold);
%    @ requires (\forall int i;
%    @          0 <= i && i < numberToEliminate;
%    @          candidatesToEliminate[i].getStatus() == Candidate.CONTINUING);
%    @ requires sumOfSurpluses + (\sum int i;
%    @          0 <= i && i < numberToEliminate;
%    @          candidatesToEliminate[i].getTotalVote()) < quota;
%    @ requires remainingSeats < numberOfContinuingCandidates;
%    @ requires (state == COUNTING);
%    @ ensures (\forall int i;
%    @          0 <= i && i < numberToEliminate;
%    @          candidatesToEliminate[i].getStatus() == Candidate.ELIMINATED &&
%    @          candidatesToEliminate[i].getTotalVote() == 0);
%    @ ensures numberEliminated == \old (numberEliminated) + numberToEliminate;
%    @ ensures remainingSeats <= numberOfContinuingCandidates;
%    @ ensures numberElected <= seats;
%    @ ensures \old(lowestContinuingVote) <= lowestContinuingVote;
%    @*/
%  protected void eliminateCandidates(
%		  /*@ non_null @*/ Candidate[] candidatesToEliminate,
%		  int numberToEliminate);
%\end{verbatim}

%\paragraph{The annotated Java code}



%=====================
\section{Election Verification}

Accidentally or maliciously altering a small number of votes can alter the outcome of a closely contested election. 
Every vote needs to be counted.  This means that even a very small error can have wide ranging consequences, 
undermining confidence in the democratic process and weakening the moral authority of elected leaders.

Full mathematical/scientific verification of an election has at least three major dimensions, \emph{in addition to} paper-based auditing and verification:

\begin{itemize}
\item Vote Verification i.e. software independent verification of the actual election results,
\item Hardware Verification of the balloting machines (when applicable) and
\item Software Verification i.e. formal specification and verification of the software used
\end{itemize}

In this paper we discuss software verification of a vote counting subsystem for PR-STV.

%--------------------------------------------------
\subsection {Vote Verification}

Cryptographic verification of the election relies on verifying the result and outcome of the election, rather than the software or hardware used.  In particular it relies on the ability of voters to verify that
\begin{itemize}
\item their vote was cast as intended,
\item their vote was counted as cast, and
\item all valid votes are counted correctly, especially when using complex schemes such as STV,
\end{itemize}
without knowing anything about the hardware or software used.  A number of proposed voting schemes support cryptographic verification but not software verification (\eg Helios).

For example, when using PR-STV, how can be prove that all votes are tallied correctly, without examining the software and violating voter privacy and ballot secret.   ~\cite{TeagueEtAl08} discuss how to defeat the 'Italian Attack' (vote signing).  However, there is another danger to consider.  Supposing the official tally software has at least one non-trivial undetected bug (a.k.a. logic error) and that we also have a large number of external organizations counting the votes in a cryptographicaly secure way.  Suppose a majority of observers agree with official tally, but a sizeable minority do not. No one has verified their software, but we need unaminity to verify the election, unless we are willing to accept a certain percentage of dissidence with the result.  The question then becomes: what percentage of observers are willing to accept the official result?  In the United States it seems that questions of this kind are decided by the judiciary [Florida2000] in somewhat arbitrary ways, based more on sublties of election law, rather than the actual intent of the voter.

In essense, the black box approach to verification risks on accidental agreement of a majority of independent counting systems, which might happen to contain similar errors.  For example, handing of ties and rounding of transfers in STV are highly probable sources of error in an unverified system. [Coyle05]


%--------------------------------------------
\subsection {Hardware Verification}

Hardware verification is essential if third-party voting machines are used for ballot casting \eg in polling place (non-remote) elections. Remote voting avoids this by allowing voters to choose to use their own devices, such as personal computers or mobile phones.

Note that remote voting is automatically \emph{hardware-independent} whereas polling-place voting machines would require hardware specification and verification[Sastry-Kohno-Wagner].  , so we won't discuss hardware verification in any detail.  Note that verification of the application-level software is not enough, we also need to have confidence in the operating system, unless the voting client software is \emph {platform independent}

A number of remote voting systems have been proposed or developed.  KOA was used in the Netherlands.  
Its open-source successor KOA/2 is publicly available (but cannot be run out-of-the-box).  Civitas has also been proposed, but is not yet a live implementation. Polling place voting systems could almost be considered as a special subset of remote voting systems, 
in that a supervised voting process requires weaker trust assumptions (a.k.a security garuantees) than a fully remote voting system.

Nevertheless, an online voting system allows a voter to use hardware of their own choice e.g. a laptop at home and the voter does not have to depend on the trustworthiness of the voting booth, instead they need to consider the trustworthiness of their own computer, the election server and the client software downloaded for the purpose of voting.  The use of proof carrying code (PCC) certificates, developed using the Mobius PVE, for example, can help to mitigate that risk.

Note that online voting includes but does not always imply web-based voting [Adida08] and would not necessarily require the use of a browser.

%-------------------------------------------
\subsection {Software Verification}

Software verification is needed to ensure that the election is in fact software-independent i.e. that the exact same results would be achieved with a completely different software implementation, provided the same specifications are followed.   

As a consequence of Rice's Theorem: there exists no automatic method that decides with generality non-trivial questions on the \emph{black-box} behavior of computer programs, which is why we need to formally verify our software.

%~~~~~~~~~~~~~~~~~~~~~~~~~~~
\paragraph{Extended Static Analysis}

Extended Static Checking (ESC) is the transformation of a program into a set of verification conditions, 
which can then be checked using
an automated theorem prover.

ESCJava2 is an Extended Static Checker for Java, which is integrated with JML and forms a subsystem of the Mobius PVE.

The Extended Static Checker for Java version 2 (ESC/Java2) is a programming tool that attempts to find common run-time errors in JML-annotated 
Java programs by static analysis of the program code and its formal annotations. Users can control the amount and kinds of checking that ESC/Java2 
performs by annotating their programs with specially formatted comments called pragmas.

%============================
\section{Conclusion and Future Work}

Existing verification tools are either still under active development or do not yet provide soundness and completeness of verification.  
As yet there is no full functional verification tool which has been fully verified in itself.  Since we can't yet achieve full assurance of verification, 
then we aren't yet ready for electronic voting, except for very low-risk elections for small decentralised organizations in which paper balloting might be 
difficult or uneconomical i.e. in which constituencies cannot be defined by geographic location and postal voting might be unreliable.

%\ifCLASSOPTIONcaptionsoff
%  \newpage
%\fi

\newpage 

Points that \textbf{must} be made in this paper:
\begin{itemize}
\item Elections are important, counting ballots is only one facet of the entire process, but is a critical component.

Cryptographic verification works with simpler counts such as first-past-the-post.  With PR-STV counting is much more complex and error prone.  Much research has already been done on non-STV systems.

\item Critical systems must be designed and constructed with care and consideration.  Dependable software engineering techniques are appropriate.

If electronic voting is to be used at all, then the software design needs to be flawless.

\item A typical argument against the use of formal methods is cost and time.  Major elections run every year, at best, so we have the time, and are critical components of our society in which millions are spent on TV advertising alone, so surely we can spend a few tens of thousands of the the critical software component.

In Ireland, it is feasible to count votes by hand due to the smaller population, but the cost of manual counting and of managing paper ballots in a central facility does not scale for larger populations.  Secure storage of ballot boxes in a central facility can become expensive.  STV requires central tallying of ballots.  In Ireland, it is currently less expensive to count votes by hand, than to use commercial proprietary voting machines.  However if Ireland decides to reduce the size of its parliament and therefore increase the size of its constituencies, then manual counting starts to become more expensive.

\item If a couple of academics in their free time can do this, then anyone can.  There is no reason whatsoever why any voting system used in a major election should not be fully verified.

It's both a conceptual barrier and due to immaturity of the verification toolsets.  An industrial strength verification suite would be nice, but is there a market for it? For all the wrong reasons, a commercially shrink wrapped software tool is seen as being more reliable than free software.

\item What have we learned about translating legal requirements to software?  Is this is wise methodology?  Did we see any inconsistencies and ambiguities in the law?  How are these resolved?

In practice there is a lot of experience with the manual process, but some of the rules for drawing lots, shuffling ballots and rounding of transfers are rendered obsolete by the change of technology, so minor changes in the legislation will be required.  In Ireland the CEV acted as the customer and provided clarifications on request, in line with the standard process for public tenders.  The law was then updated to validate the use of electronic voting and to clarify any such ambiguities.  Similar to banking and tax compliance software, there needs to be provision for updating the software to reflect changes in the law.

\item If one writes the full specification prior to starting implementation, does this make things more difficult?  Where is the balance between verification and testing as one implements a system such as this one?  What do you wish you'd have done differently?

I could have added more variation points to the design e.g. recalculation of the quota and fractional transfers - both of which are now possible under electronic voting but labour intensive under paper based counting.  Would also like to implement range/score voting for comparison.

Also, a more mathematical representation using JML model classes e.g. model ballots as tuples.

Writing JML before BON did make it much more difficult.  Using BON first would have helped to elaborate the overall architecture before making a more detailed design.  Using JML model classes for a more abstract view of ballots etc might also help.
There would still be  a case for test-driven development i.e. requirements, then BON, then JML then unit test cases with data, then finally the Java code.

\item Does executable model-based specification have legs?  Did it work well in this example?  If so, what does it buy us; if not, why?

No, it didn't work here because the quantifiers are not executable?  It does allow us to verify the simpler code and to flag the complex code for manual inspection.  False positives are possible, but can be mitigated by using a suite of other tools, whereas false negatives may discourage adoption of the tools.  

\item We must accurately characterize to the reviewers and editors the exact state of the system as it stands when we submit, and where we intend it to be by the time the journal is published.

The system is in redesign - effectively version 2 of \votail.

Neither the BON design nor the Java code are far enough advanced - only the functional analysis and JML specifications are done. 

\item Are there other formal methods that would be more appropriate for this kind of work?  E.g., VDM?

VDM would allow a more mathematical representation.

\item Will we ever achieve full functional verification with the PVE?  If not, why not?

The PVE is not yet capable, likewise JML4.  Concurrency might also be of concern.  Also ESC/Java2 failed to detect a large amount of redundant defensive coding e.g. if (precondition == true) {...} but the PVE is needed for working with bytecode...

\end{itemize}

%=================
%\bibliographystyle{plain}
%\bibliography{paper}

%=====================================================================
\section{Acknowledgments}

This work is being supported by the European Project Mobius within the IST 6th
Framework and national grants from the Science
Foundation Ireland including LERO C-SET and LGSSE
This paper reflects only the authors' views and the Community is not liable for
any use that may be made of the information contained therein.

%======================================================================
{\footnotesize \bibliographystyle{acm}
  \bibliography{paper}
}

\end{document}


