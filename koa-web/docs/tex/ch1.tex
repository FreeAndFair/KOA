\chapter{Introduction}\label{cha:introduction}

\section{Purpose}\label{sec1:perpose}

This document describes the functionality of the Experiment Distant
Voting\footnote{Dutch: Experiment Kiezen op Afstand. MW} (KOA
system). Its purpose is twofold: to give the customer insight in the
(internal) working of the system and to form as a basis for technical
specifications and testing of the system.

This document will be given to the customer for approval.

\section{Customer}\label{sec1:customer}

The customer is the Dutch ministry of internal affairs\footnote{Dutch:
Ministerie van Binnenlandse Zaken en Koninkrijksrelaties. MW}(BZK).

\section{Relation to other documents}\label{sec1:relat-other-docum}

The functional demands of BZK are documented in the initial
proposal. Because BZK chose for a service, the functional demands are
restricted to the essential conditions involving this service only. In
the initial proposal LogicaCGM has proposed a sketch of the to be
realized system, including a view of the system from the user's
perspective(Voter,Chairman,data maintainer). This description can be
found (in slightly altered form) in
Chapter~\ref{cha:role-steps-voting} and~\ref{cha:stat-funct-syst} of
this document.

The instructions for the voting stations (including screens) are not
included in this document. They can be found in the document
``Werkinstructie voorzitter''~\cite{Voorzitter}. The same holds for
the instructions for the data maintainer, these are included in the
document ``Werkinstructie Databeheerder''~\cite{Databeheerder}. Only
those reports that are generated by the KOA system as a means of
verifying its workings are included in Appendix~\ref{cha:reports}.

\section{Version control}\label{sec1:version-control}

Version control for this document is done according to the internal rules
of the LogicaCGM quality system Cortex\footnote{This does not hold for
this English translation. MW}.

\section{Version}\label{sec1:version}

This version conforms to the system after a first user test.

\section{Reading guide}\label{sec1:read-guide}

\begin{itemize}
	\item Chapter~\ref{cha:role-steps-voting}
	and~\ref{cha:stat-funct-syst} contain a description of the (to
	be realized) system in the form of a functional model and a
	description of the roles(Voter, Chairman, data maintainer) of
	the system.  

	\item Chapter~\ref{cha:interf-voting-mach} contains a point
	wise description of the functionality of the system in terms
	of the different states the system can have.

	\item Chapter~\ref{cha:interch-form-bkz} contains a functional
	description of the relation between the TSM and the VSL.

	\item Chapter~\ref{cha:data-model} contains a functional
	specification of the files that are used to exchange data
	(voter registration and candidate lists) between the KOA and
	the customer (the Dutch ministry of internal affairs).

	\item Chapter~\ref{cha:audit-logs-acco} contains the data
	model, as far as related to the functional specification.

	\item Chapter~\ref{cha:user-interface-tsm} contains a lists of
	logging data that is used for audits and accountability.

	\item Chapter~\ref{cha:web-interface} contains a detailed
	description of the call flow of the TSM

	\item The appendices contain a glossary, screen and report
	layouts and the deliverables.
\end{itemize}

\section{Application area}\label{sec1:application-area}

The system has been designed for the experiment of BKZ. The
specifications leave room at several points for a follow up (see RFP,
chapter 1) and can be extended to a national level. Depending on the
demands in different situations, additional features and/or changes to
the system will be required. 

The system, together with organizational and procedural measures, can
provide a voting service that meets the specifications and safety
rules of the RFP knock-out demand 1 and demand 8.
\begin{itemize}

\item Vote secret:
\begin{quote}
it is impossible to connect any given Voter to a valid vote, to
ensure confidentiality of the vote;
\end{quote}

\item Uniqueness:
\begin{quote}
every valid Voter can only vote once and this vote will exactly be
counted once in the end result;
\end{quote}

\item Valid voters:
\begin{quote}
Only voters who have the legal right to vote should be allowed to vote;
\end{quote}

\item Integrity:
\begin{quote}
the end result of the ballot can not be influenced in any other way
than by casting a legal vote.
\end{quote}

\item Accountability:
\begin{quote}
the system generates all the constitutionally required
information (to be able to verify the end result)
\end{quote}

\item Recounting:
\begin{quote}
conform the constitutional demands a recount is possible
\end{quote}

\item Availability:
\begin{quote}
Legally allowed voters should, as much as possible, be able to cast
their votes. The guidelines from the European commission with
respect to the availability of the web sites of the government and the
contents thereof have to be taken into consideration (see appendix 8,
``Toegankelijkheid van publieke websites en de inhoud
daarvan''\footnote{English: Availability of public web sites and the
content thereof. MW},COM (2001) 529 final, and the resolution of the
European Parliament concerning availability of web sites\\
(P5\_TAPROV(2002)0325));
\end{quote}

\item Transparency for the Voter
\begin{quote}
the Voter should be able to understand and trust the voting process
\end{quote}

\end{itemize}

\bibliographystyle{plain}
\bibliography{refs}

%%% Local Variables: ***
%%% mode:latex ***
%%% TeX-master: "funDesign.tex"  ***
%%% End: ***
