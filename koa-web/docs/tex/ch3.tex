\chapter{Roles and steps in the voting process}\label{cha:role-steps-voting}

Three different roles are distinguished: the data maintainer, the
Chairman and the Voter. All roles will be described in more detail in
this chapter.

All roles are realized by the standard given by the functionality that
is present in the underlying e-Platform. Data maintainer and Chairman
are explicitly loaded into the system as users during the preliminary
phase of the experiment. At this stage a role is assigned to them
which ensures that they can only perform functions associated with
their respective roles. Voters are anonymous, thus Voters form an
implicit role that is assigned to anybody who connects to the KOA
system via WSM and TSM.

\section{Data maintainer}\label{sec3:data-maintainer}

The sole purpose of the data maintainer's role is to maintain the
voter register and the candidate lists. The data maintainer is
equipped with a specially prepared laptop which allows access to the
KOA system. This laptop is equipped with tools for encrypting and
decrypting data files. Communication with the KOA system takes place
via a web-interface.

The voter register is edited with data files delivered by the
Customer. Such a file can be replacing, adding or deleting (see
Section~\ref{sec2:kr-choser-register}). If the elections have started
only adding and deleting files can be used.

Loading the voter register into the KOA system works as follows:

\begin{enumerate}
	\item The voter file is constructed by the Customer in the
	specified format (see Chapter~\ref{cha:interch-form-bkz}) and
	delivered to the data maintainer. A special
	letter\footnote{Dutch: oplegbrief MW.} belonging to this file
	is send via a different route which contains a reference
	number, date and time op delivery, type of file (replacing,
	adding or deleting) and the number of persons being replaced,
	added or deleted. The letter also contains the access code that
	is needed to decrypt the file.
	
	\item The date maintainer decrypts the file and loads it into
	the KOA system.

	\item A first check of the file is carried out. Both the
	(syntactic) format and the number of persons involved is
	checked at this moment. If this does not match the file not is
	processed to exclude any possibly of loading incorrect data
	into the KOA system.

	\item The data maintainer receives a report form this check
	and checks himself if the data in the report matches the data
	in the special letter described above. If everything checks
	out, he gives permission to process the file.

	\item The file is used to replace or alter the KR. The system
	generates three output files:

	\begin{itemize} 

		\item Return file: an export of the by the Customer
		delivered data (anonymous identity of the Voter, voter
		district and voter circle) and the voter codes for all
		the Voters in the KR. This file is also exported if
		data was added or deleted.

		\item Error file: a file containing the data of the
		persons that could not be processed (e.g.\ because the
		anonymous identity wasn't unique). This file is
		supposed to be empty.

		\item A processing file that contains any possible
		problems. This file can be used for auditing purposes.
	\end{itemize}

	\item The export files and the processing file are encrypted
	by the data maintainer and delivered to the Customer. The
	Customer matches these files with the originals to ensure that
	files are processed correctly and completely.

	\item If errors are signaled during processing then these are
	corrected by the Customer. As long as the elections haven't
	started the corrected files are loaded into the system again
	(see Section~\ref{sec2:kr-choser-register}).

	\item The Customer processes the data concerning a voter and
	sends this to the Voter.

\end{enumerate}

Processing candidate lists can only occur during the preliminary
stages (before the elections) have started. This is for the most part
equivalent with processing the voter lists:

\begin{enumerate}
\setcounter{enumi}{8}
	\item The Customer sends a file (see
	Chapter~\ref{cha:interch-form-bkz}) containing the candidate
	lists. Via an alternate route a special letter is send which
	contains the access code and control data.

	\item The data maintainer decrypts the file and delivers it to
	the VSL.

	\item In the VSL the data is checked. The result is send to
	the data maintainer.

	\item If the data maintainer confirms that the data is valid
	it can be processed and read into the VSL. At this time each
	candidate gets 1000 candidate codes assigned to him/her.

	\item The result is stored in three files:

	\begin{itemize}
		
		\item A return file: an export of the read data and
		the assigned access codes.
		
		\item An error file containing unprocessed records.

		\item A processing file

	\end{itemize}

	\item The data maintainer preforms a visual inspection of the
	web pages concerning the candidate lists that are generated by
	the WSM. 

	\item The data maintainer reports the result to the
	Chairman. Any necessary corrections are again processed by
	reloading the entire file from the Customer. The export files
	are returned to the Customer.

\end{enumerate}

\section{Chairman}\label{sec3:chairman}

\subsection{Tasks of the chairman before polling}\label{sec3:tasks-chairm-before}

\subsubsection{Opening the elections}\label{sec3:begin-elections}

\begin{enumerate}
\setcounter{enumi}{15}
	\item The Chairman connects to the KOA system using the
	specially prepared laptop facilitated by LogicaCMG. This
	laptop is equipped with all necessary programs and associated
	files. The connection with the KOA system is secured.

	The interface that the Chairman uses is menu-driven. The menu
	structure is simple: a main menu that displays the status
	overview of the system and sub menus for changing the status
	of the system, retrieving reports and retrieving files for
	further processing outside the KOA system.

	A pop-up message will always be displayed which asks for an
	explicit conformation of a status change of the system,
	because many status changes are irreversible (e.g.\ if the
	elections are closed they can't be reopened). This operation
	is also protected by two PIN codes; in order to change items
	of the system two different members of the polling station have to give
	their PIN codes.

	\item The Chairman orders that the KOA system can be
	initialized. The Chairman has to give the corresponding public
	key which is used to encrypt the votes. Initialization can
	take some time, thus this has to be started at least 10
	minutes before the start of the elections.

	\item The KOA system delivers a status report, which can be
	used by the Chairman to check if the system is ready for use
	(all components function as required, the voting register has
	been correctly initialized, the voter store is empty).

	\item The Chairman orders that the elections can begin. The
	telephone and internet connnections are now opened. The system
	confirms that it is fully operational.

\end{enumerate}

\subsection{Tasks of the chairman during polling}\label{sec3:tasks-chairm-during}

\subsubsection{Interrupting the elections}\label{sec3:interr-elect}

\begin{enumerate}
\setcounter{enumi}{19}
	\item The Chairman orders that the elections are interrupted.

	\item The VSL ensures that all voting is impossible from this
	moment onwards\footnote{It's possible that voting is already
	blocked by the system because of a critical error that makes
	continuing the elections not responsible. The interruption by
	the Chairman then forms the formal confirmation of this
	state. See Section~\ref{sec4:state-blocked}.}. The WSM and TSM
	are informed in order to ensure that voters that are still
	connected to the system receive the proper error message.

	\item The critical files of the system (KR, ESB and audit
	logs) are stored on removable media. An electronic fingerprint
	is made that can be used by the Chairman when the elections
	are continued.

\end{enumerate}

\subsubsection{Continuing the elections}\label{sec3:continuing-elections}

\begin{enumerate}
\setcounter{enumi}{22}
	\item The Chairman orders re-initialization of the system. The
	KR and ESB can be re-initialized from the secured copies; this
	requires an action by the operational maintainer. The system
	again calculates the electronic fingerprint and displays this
	as part of the status report.

	\item The Chairman checks the status of the system; the
	fingerprint has to match the fingerprint of the system when it
	was interrupted.

	\item If the Chairman is convinced that the elections can
	continue the order for continuing is given. The VSL module
	redirects this to the WSM and TSM, so that the elections can
	go on.

\end{enumerate}

\subsection{Tasks of the chairman after polling}\label{sec3:tasks-chairman-after}

\subsubsection{Closing the elections}\label{sec:closing-elections}

\begin{enumerate}
\setcounter{enumi}{25}
	\item The Chairman orders the closing of the elections.
	
	\item The VSL module ensures that the elections are closed
	immediately. This command is redirected to the WSM and TSM,
	which ensure that all connected users receive a proper
	error message stating that the elections are over.

	\item The critical files (KR, ESB, audit logs en statistics)
	are stored on a removable disk.

	\item Fingerprints are generated which are compared with
	previously taken fingerprints and are stored in the audit
	log. This data can be used (if necessary) to check if the KR
	and ESB are correct in case of an audit after the elections.

	From this moment onwards an export with the encrypted votes
	can be generated. This facilitates recounting outside the
	scope of the KOA system. 

\end{enumerate}

\subsubsection{Opening the votes and counting}\label{sec3:open-votes-count}

\begin{enumerate}
\setcounter{enumi}{29}
	\item The Chairman orders that the stored votes can be
	opened. The private key that is necessary to decrypt the votes
	is also transmitted at this point.

	\item An export file of the encrypted votes is generated.

	\item The votes are translated into a processable table in a
	database. This table can also be exported to facilitate
	recounting by another system.

	\item The votes are counted and the result is reported back to
	the Chairman.

	\item The audit log is updated.

	\item The KR and the counting files are frozen, but will
	remain available for a unspecified time period for audit and
	verification purposes.

	\item The Chairman request the audit reports which are added
	to the official result\footnote{The Dutch text literally
	speaks of a `proces-verbaal' (a specific kind of Dutch legal
	document) in stead of the `official result'. MW \label{footnote:official_result}}.

	\item The Chairman can at this (or some later) time request
	the cleaning of the entire KOA system.

\end{enumerate}

\subsection{Additional tasks of the Chairman}\label{sec:addit-tasks-chairm}

These functions can be used by the Chairman to request information
concerning the progress of the elections. Reports can be requested at
any given moment; thus it is not necessary to wait until the elections
are finished to request reports which are added to the official
result\footnote{`official result' as meant in the previous
footnote~\ref{footnote:official_result}. MW}

\begin{enumerate}
\setcounter{enumi}{37}
	\item Requesting reports concerning the audit logs. This
	function displays all the audit information available at this
	moment. It can be added to the official result.

	\item Requesting status. This function displays data
	concerning the progress of the elections.

	\item Requesting fingerprints concerning the candidates and
	associated data. The KOA system uses fingerprints to check
	that data in the system has not been altered in any illegal
	way. Most of these fingerprints are generated and checked
	automatically (see Chapter~\ref{cha:interf-voting-mach} and
	Chapter~\ref{cha:audit-logs-acco}); however since the fingerprint
	of the candidate lists never (should) change it is only
	checked on request of the Chairman. The function checks the
	complete candidate list (including all assigned candidate
	codes), and should therefore not be used during the election
	process.

	\item Exporting the assigned transaction codes (in XML
	format). This function is available after the elections have
	closed. The codes can be used by voters to verify that the
	code they received after the voting process is indeed used
	during the elections.

	\item Exporting encrypted votes (in XML format). This function
	is available after the elections have closed. The encrypted
	votes can be used to recount the votes outside the KOA system.

\end{enumerate}

\section{The voter}\label{sec3:voter}

\subsection{Voting via PC}\label{sec3:voting-via-pc}

\begin{enumerate}
\setcounter{enumi}{42}
	\item The Voter connects to the KOA system using his/her own
	ISP and a standard web browser.

	\item The Voter authenticates to the system with the voter
	code and the self-chosen access code.

	\item The voter code and access code are verified in the KR.

	\item The Voter inputs the candidate code of the candidate
	he/she wants to vote for. The system checks if the candidate
	code is valid for this specific voter and, if this is indeed
	the case, displays the data associated with the candidate on
	the screen. This to ensure that the correct candidate was
	chosen. The Voter can confirm his/her choice.

	\item The data associated with the voter and the chosen
	candidate are transmitted to the KOA system. Another check is
	preformed if the Voter can still cast a vote; in principle a
	vote could have been casted in the mean time via another PC or
	telephone.

	\item The vote is encrypted and stored in the ESB; in the same
	transaction the KR stores that the Voter has voted.

	\item The system generates the transaction code and transmits
	it to the WSM.

	\item The WSM displays a closing screen that (if so specified
	in the corresponding system parameter) displays the
	transaction code.

\end{enumerate}

\subsection{Voting via telephone}\label{sec3:voting-via-telephone}

\begin{enumerate}
\setcounter{enumi}{50}
	\item The Voter connects via a telephone using tone codes to
	the TSM.

	\item The Voter authenticates to the system with the voter
	code and the self-chosen access code.

	\item The voter code and access code are transmitted by the
	TSM to the VSL. The VSL verifies the Voter's right to vote
	with the KR.

	\item The Voter transmits the candidate code to the TSM. The
	TSM verifies with the VSL if this is a valid choice and asks
	the Voter to confirm it.

	\item The TSM transmits the vote to the VSL module. Another
	check is preformed if the Voter can still cast a vote; in
	principle a vote could have been casted in the mean time via
	another telephone or PC.

	\item The vote is encrypted and stored in the ESB; in the same
	transaction the KR stores the that Voter has casted a vote.

	\item The system generates the transaction code and sends it
	to the TSM.

	\item The TSM gives a closing message to the Voter with (if so
	specified in the associated systems parameter) the transaction
	code and disconnects.


\end{enumerate}


%%% Local Variables: ***
%%% mode:latex ***
%%% TeX-master: "funDesign.tex"  ***
%%% End: ***
