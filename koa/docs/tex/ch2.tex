\chapter{Functional Overview}\label{cha:functional-design}

The most important components of the KOA system are displayed in
Figure~\ref{fig:koa-components}.

\begin{figure}[H]
 \center
    \input{picts/Fundesign1.pdftex_t}
   \center
\caption{Functional components of the KOA system}
\label{fig:koa-components}
\end{figure}

\begin{itemize}
\item {\bf The virtual polling station (VSL)}. This is the central controlling
      function of the KOA system. On behalf of the Experiment the KOA
      contains one VSL; one VSL can serve multiple voting
      circles\footnote{Dutch: Kieskringen. MW}. Multiple voting machines
      can be connected to a VSL, this ensures that the different
      modules of the system (voting via the internet, via phone and
      possibly, in the future, other systems) can be realized. All
      voting machines are connected to the VSL in the same manner.

\item {\bf The electronic voting box (ESB)}. Each voting circle has one 
      ESB; all casted votes are encrypted and stored in the ESB.

\item {\bf The voting register (KR)}. Before the start of the elections 
      all the (anonymous) information of legal voters is stored in the
      KR. In addition the KR registers if a voter already casted
      his/her vote. There is only one KR, even if the Experiment will
      be extended with multiple virtual polling stations.

\item {\bf The web voting machine (WSM)}. This component realizes the 
      link between the PC of the Voter who wants to vote via the
      internet and the virtual polling station. The Experiment
      contains one WSM per VSL.

\item {\bf The telephonic voting machine (TSM)}. This component realizes 
      the link between the telephone of the Voter who wishes to vote
      per telephone and the virtual polling station.
\end{itemize}

The Chairman of the polling station (VZ) communicates on behalf of his
role via a secure connection with the VSL. The chairman does not play
a role in the actual casting of a vote by the Voter; he only controls
the VSL and receives from the VSL the reports that are required to
perform his function. Each VSL can have multiple Chairmen, but only
one Chairman is active at any given moment. The KOA system does not
make a distinction between the Chairman and the (other) members of the
polling station; only one role has been defined for the members of the
polling station. In this document ``Chairman'' refers to ``the persons
who, acting on behalf of the polling station, performs an action on
the system''. Only in one instance (changing the status of the System)
two members of the polling station have to give their approval (see
Section~\ref{sec3:tasks-chairm-before}).

The Voter of the voting service will at any given time only
communicate with one of the voting machines of the KOA system. Since
the voter registration is centralized (in the KR), the Voter does not
have to say beforehand which module he/she will use for castings
his/her vote. Because of the dimensionality of the system it is
preferred to have some indication as to how many users will roughly
use each separate module.

The systems data maintainer (DBH) has the following functions:
\begin{itemize}
	\item reading the candidate lists into the system and
	delivering the generated candidate codes to the customer. This
	function takes place before the actual voting has started;

	\item loading the KR with the anonymous records of legally
	allowed voters (before the start of the actual voting) and
	redelivering the file with voter codes to the customer;

	\item maintaining the voting register after the start of the
	election (adding voters and stopping the registration of
	voters). This can also be done during the elections.

\end{itemize}

Not included in the diagram are facilities for maintaining the system
and facilities dealing with audits of the system. Audit logs are
described in Chapter~\ref{cha:user-interface-tsm}.

\section{KR (Voter register)}\label{sec2:kr-choser-register}

The Voter (or voting) register contains all data related to legally
allowed voters. The Voter register consists of three parts: data to
authenticate the Voter, data concerning the casting of the vote and a
table that is used when filling the register.

In order to verify if a voter is legally allowed to vote the following
is recorded:

\begin{itemize}
	\item The voter code which the Voter uses to show that he/she
	is a legally allowed voter at the moment that he/she wants to
	cast his/her vote. This voter code is unique and is generated
	by the KOA system at the moment that all the data is
	loaded. The voter code is generated in such a way that it
	fulfills the 11-prove of the (Dutch) social security
	number\footnote{Dutch: de 11-proef van het Sofi nummer. MW}. A
	mistake by the Voter in one digit never results in a valid
	voter code; a mistake in two numbers only in less than 10
	\%. In combination with the 5 digit access code of the Voter
	this results in a chance of less then 1 in $10^6$ that a Voter
	votes with another voter code than his/her own.

	\item The access code chosen by the Voter. This code is
	delivered and stored in encrypted form. For the encryption a
	one-way encryption algorithm is used.

	\item Indication ``removed''. This indication is off by
	default. Only if after starting the election the voter is
	withdrawn this option can be set to ``true''; physically
	removing the Voter is unwanted, because there might be actions
	in the logs which refer to the Voter and removing the Voter
	can thus lead to inconsistencies.

	\item Voter circle and voter district. The voter circle is
	used to validate the candidate codes. When a vote is casted
	both the voter circle and voter district are stored with the
	vote. This ensures that a result per voter circle can be
	generated. Results per voter district are also possible,
	although this function will not be realized for the actual
	Experiment.

\end{itemize}

Data concerning the casting of the vote:

\begin{itemize}
	\item An indication if the Voter has already casted a vote or not.

	\item At which date and time a vote was casted using which module.

	\item The delivered transaction code.

	\item A registration of a failed attempt of a Voter to log in,
	consisting of:

	\begin{itemize}
		\item The actual counter of the number of attempts.
		\item The total number of attempts.
		\item The time of the last attempt. 
	\end{itemize}

\end{itemize}

For filling the KR a table is used (The Voter code allocation register
or SAR\footnote{Dutch: het Stemcode Allocatie Register. MW}):

\begin{itemize}
	\item The Voter identification. This is an anonymous, unique
	authentication of the legally allowed voter (delivered by the
	Customer).

	\item The Voter code that is generated for the voter (see above).

\end{itemize}

The SAR ensures that the system will always generate the same voter
code for an individual Voter, even if the same voter is delivered to the
system multiple times.

The KR is filled with an import file from the Customer. The Customer has
to maintain its own register of legally allowed voters. It is required
that the KR contains an exact copy of the data of the Customer,
therefore an import file is always used. In addition, each imported
file is re-exported and delivered back to the Customer for audit
purposes.

The import function can work in three different ways:

\begin{enumerate}
	\item In ``replace mode'' the import file contains a full copy
	of all the data of a voter circle. When loading this file, the
	complete internal copy of the voter circle is deleted and then
	the import file is processed fully. Thus in case corrections
	are desired, the full file has to be delivered again. The
	system guarantees that each legally allowed voter will again
	have the same voter code as before.

	\item In ``add'' mode no data is deleted, in stead only new
	voters are registered. This mode is meant for voters who
	registered after the beginning of the elections.

	\item In ``replace'' mode earlier registered voters are marked
	as no longer allowed to vote. To maintain system integrity,
	voters are not physically removed from the system. This
	functionality is used when a voter is no longer allowed to
	vote after the start of the elections. It only effects voters
	that haven't cast their vote yet; votes that are already cast
	(and thus (anonymously) stored in the system) can not be
	removed.

\end{enumerate}

As long as an election hasn't started the ``replace'' mode is used, in
order to keep the probability of errors as small as possible. The
``add'' and ``remove'' mode are defined because the registration and
election periods overlap. The system is designed in such a way that
replacing files (with a maximum of 600.000 voters) can be processed
within one working day and add/remove files (with a maximum of 30.000
voters) can be processed within one hour.

Other functions of the KR are:

\begin{itemize}
	\item Initialization: Here the systems checks if all
	indications that a particular voter has casted a vote and all
	other data is properly removed. This function is called from
	the VSL when the elections start.

	\item Interrupt/continue. When the system is interrupted all
	data is stored on a removable medium. In addition, an
	electronic fingerprint of the voter register is calculated
	that can be used to verify that the state of the system is
	identical when the system continues to the interrupted
	system. This fingerprint is delivered via the VSL to the
	Chairman.

	When the elections are continued either the removable medium
	or the internal database can be used. In both instances a
	comparison to the afore mentioned fingerprint, which was
	generated directly after an interrupt event, is performed. If
	the fingerprints don't match the Chairman has the final say if
	the elections can continue.

	\item Verification. Here the system checks if the voter code
	is part of the register, if the voter is legally allowed to
	vote and if the access code matches; if all these requirements
	are met the system checks if the Voter has already casted a
	vote or not. To check the access code, it is encrypted with the
	same one-way encryption function as mentioned before and
	matched with the stored encrypted code. For security reasons
	the system does not differentiate between the situations
	``access code is not present'' and ``access code not
	right''. The situation ``maximum number of attempts reached''
	\emph{is} displayed individually, together with the time that
	this situation remains.

	This functionality can be used when a transaction is started
	to authenticate the Voter and to verify that the Voter has
	casted a vote.

	The KR further registers how many times authentication
	fails. After a certain number of attempts all further attempts
	are blocked for a certain period. The parameters of this
	function can be changed. The default is a maximum of three
	attempts and a blocking period of an hour.

	\item Storing if a voter has casted a vote. This can only be
	done after an actual vote by a Voter.

	\item Cleaning: all data concerning the vote transactions are
	removed.

\end{itemize}

Checking if the maximum number of authentication attempts is reached
is done before the actual authentication itself and uses three data
items per Voter: the actual counter, the total number of failed
authentication attempts and the time of the last attempt. It works as follows:

\begin{itemize}

	\item The first check is if the maximum number of attempts is
	reached (actual counter $\geq$ maximum). If this is the case
	the system checks if the blocking period is still active (time
	last attempt + interval $>$ current time). If this is the case
	the attempt fails (code ``Voter blocked''). This is logged.

	\item If the blocking period ends the actual counter is
	restored to zero and the time of blocking is deleted.

	\item Then the actual authentication is performed. If
	authentication succeeds the status ``success'' is
	returned. This is not logged.

	\item In case authentication fails the actual counter and the
	total counter are incremented and the time of the last attempt
	is stored. If in this case the maximum is reached the status
	becomes once again ``Voter blocked''; otherwise the result is
	``authentication failed''. This is also logged.

\end{itemize}

The total counter per Voter (not to be confused with the counter of
the whole register) will thus never be returned to zero. It services
to signal systematic attempts to tamper with the data of a certain
voter over a longer period (see Section \ref{sec8:voting}).

\section{VSL (Virtual polling station)}\label{sec2:vslv-poll-stat}

The VSL is responsible for operating the election process. The VSL
distinguishes functions on behalf of the Chairman of the polling
station and functions for the operational part of the voting process.

The Chairman uses a specially prepared secure laptop to securely
communicate with the VSL. The functions of Chairman are:

\begin{itemize}
	\item Initialization. This facilitates the preparation of the
	voting process. The VSL module controls the initialization of
	all the components and guards the incoming confirmations of
	the separate components of the KOA system. If all components
	have confirmed to the Chairman that they are ready to perform
	their task the election can start.

	\item Opening the election. This can only happen when the
	system is properly initialized. At the start of the elections
	the VSL gets the public key of an RSA key pair from the
	Chairman. This key is used to encrypt all incoming votes.

	\item Interrupting the elections. From this moment onwards the
	VSL refuses any incoming commands from the voting
	machines. All components receive a command to halt their
	activities. Both the ESB and the KR confirm this by sending an
	electronic fingerprint (see
	Section~\ref{sec2:kr-choser-register} and
	Section~\ref{sec2:esbel-voting-boot}). The VSL bundles these
	and returns them to the Chairman.

	\item Re-initializing. This can only happen if the elections are
	interrupted. When re-initialization is performed the Chairman
	can chose to do this using a removable media (which can be
	necessary when a backup system has to be used) or it can be
	done using the existing database. In both cases a new
	fingerprint is generated which is compared to the one that was
	generated at the moment that the elections where
	interrupted. If the fingerprints don't match the Chairman has
	to decide if the elections can continue.

	\item Continuing the elections. This is only possible if the
	system has been interrupted and re-initialized using one of
	methods explained above. The public key of the Chairman is
	retransmitted and compared to the original public key; this
	prevents that the votes in the ESB are encrypted with multiple
	keys. The VSL then clears the system for continuing the
	elections.

	\item Closing the elections. From this moment onwards the VSL
	refuses to accept any new votes casted from one of the voting
	machines. All components of the system receive a command to
	stop their activities. The ESB and KR confirm this by sending
	a fingerprint to the VSL (see
	Section~\ref{sec2:kr-choser-register} and
	Section~\ref{sec2:esbel-voting-boot}). The VSL bundles these
	and delivers them to the Chairman. Voters receive the message
	that the elections are closed.

	\item Opening the votes\footnote{The System indicates this
	with ``opnemen stemmen'' (English: loading votes) in stead of
	``lichten stemmen'' (English: Opening votes)}. This can only
	happen if the elections are closed and this action can only be
	performed once during any given election (excluding the case
	where a state of the system is restored using a back up and
	the election process continues with the appropriate
	action). With this action the Chairman has to use his private
	key. This private key is send to the ESB which uses it to
	decrypt the votes and transform them to a specially formatted
	file. This forms the basis for the counting process, that is
	started at this moment. The result (in report form) is
	delivered to the Chairman.

	\item Requesting status. This results in a standard report of
	the progress of the elections (See
	Section~\ref{sec8:status-report}). This report also services
	as a main interface for using the other functions of the
	system.

	\item Requesting incidents. If an incident occurs within the
	system that's detected and forms a threat (or can become a
	threat) for the election process, then this is logged within
	the system which can be requested by the Chairman.

\end{itemize}

The second function of the VSL module concerns the operational part of
the elections. In order to achieve this the voting machines (TSM, WSM)
communicate with the VSL module. The functions in this interface
consist of:

\begin{itemize}
	\item Verification if a voter is legally allowed to vote: this
	ensures that a voter is legally allowed to vote. The question
	consists of the access code and the voter code. The VSL module
	checks if the election is not interrupted or closed and
	answers the question by querying the KR (See
	Section~\ref{sec5:verification-voter} for possible responds).

	\item Verification candidate: this ensures for a voting
	machine that a candidate code is correct for a specific Voter,
	i.e., if the candidate code is contained within the candidate
	list that corresponds to the voter circle of this particular
	Voter.

	\item Casting a vote. the question consists of a voter code
	and an access code, together with the choice (candidate code)
	of the Voter. The VSL checks the same items as during
	verification; if the Voter is indeed allowed to vote, the
	choice is encrypted and stored in the ESB. The status of the
	Voter is corrected to prevent that double votes are
	possible. A transaction code is generated that is stored in
	the KR and passed on to the Voter as confirmation that a
	successful vote has been cast. This transaction code is a
	random number that is generated in the same manner and
	confirms to the same criteria as the voter code. Using the
	transaction code is not mandatory; a system parameter
	determines if a transaction code is passed on to the
	user. This does not concern the internal working of the
	system. The transaction code will always be generated and
	e.g.\ passed on to the TSM. The TSM has a similar parameter
	that decides if the transaction code will be read out aloud or
	not.
	
\end{itemize}

Besides these functions the VSL passes status commands
(initialization, elections interrupted, elections continued, elections
closed) to the voter machines.

For security reasons each candidate gets 1000\footnote{The number 1000
is a system parameter, that can be changed} different codes assigned
to him/her. The candidate lists that are send to the Voter contain one
of these codes. This makes it possible to generate a large number of
different candidate lists. This security measure is used for voting
via telephone; If a connection is monitored it is not easy to match
the candidate code to any specific candidate without a complete list
of all candidate codes. The complete list can not be generated form
any given candidate list. The VSL generates these codes when the
candidate lists are imported (see Section~\ref{sec6:deliv-cand-lists})
and maintains them. When a vote is stored this code is translated back
to a position at the original candidate list.

It is crucial in this setting that all data concerning the casting of
a single vote are performed in one transaction by the voting
machines. This simplifies the processing of the votes; it is not
necessary to keep track of the fact that the vote is in progress. In
case the Voter tries to cast multiple votes at once (e.g.\ via
telephone and internet) there will always be one first vote. This vote
will be stored; the second attempt will fail because the Voter already
casted a vote.

\section{ESB (Electronic voting boot)}\label{sec2:esbel-voting-boot}

All votes are encrypted and stored in the electronic voting boot
(ESB). There is one ESb per voter circle. Each casted vote is stored
once, it contains the contents of the candidate code of the candidate
that received the vote (or a fictive candidate with semantics ``blanc
vote''\footnote{Under Dutch law it is possible to cast a vote without
voting for a candidate, this is called a blanc vote (Dutch: blanco
stem). MW}. The name and initials of the candidate, the position on
the candidate list, the list number and the political party of the
candidate are added as well. This added data is not really necessary,
but in cases of emergency it can be used to obtain the results of the
elections form a removable media, outside the context of the virtual
polling station. The voting district is also stored together with the
vote, in order to (in a later version) give election results per voter
district.

Within the system blanc votes are realized as an always present list
with only one candidate. For security reasons the rest of the blanc
votes are treated as much as possible as ordinary candidates.

The votes are encrypted when they are stored using the public key of
the Chairman of the polling station (see
Section~\ref{sec2:vslv-poll-stat}). This guarantees two things: that
the votes can only be opened by the Chairman of the polling station
and that the corresponding VSL is responsible for the casted
votes. Random data is added to the votes when they are encrypted. This
ensures that votes within the same voter district and for the same
candidate have a different encryption result for each vote, making it
impossible to interpret encrypted votes.

The ESB has the following functions:

\begin{itemize}
	\item Initialization. The system checks if the storage medium
	is empty. The VSL calls this function when the elections are
	started.

	\item Interrupt/continue. When the system is interrupted, the
	votes are stored on a removable device. A fingerprint of the
	stored votes is made which ensures that the data is the same
	when the elections are continued. This fingerprint is returned
	via the VSL to the Chairman.

	When the elections continue the database can be restored from
	the removable device (re-initialization) or the database can
	be used in its current form. In both cases the previously
	generated fingerprint is compared to a freshly generated
	one. When the fingerprints don't match the Chairman has the
	final say in the continuation of the elections.

	\item Deposit/cast vote.

	\item Opening the votes: The ESB receives the private key of
	the Chairman (via the VSL). This key is used to decrypt the
	votes in a random order (to making tracing voters by the order
	in which they voted impossible). The (decrypted) votes are
	then stored in a specially formatted `counting' file. The VSL
	module uses this file to count the votes and deliver the
	result of the elections to the Chairman. 

	\item Recounting. This is not a function of the ESB itself,
	but the ESB can be used to facilitate recounting. The next
	diagram clarifies this:\\
	\newpage

	\begin{figure}[H]
	\begin{center}
	\includegraphics[scale=0.4]{picts/counting.jpg}
	\end{center}
	\caption{The counting process\label{fig:counting-process}}
	\end{figure}
	
	This diagram clarifies that recounting is possible in several
	different ways:

	\begin{enumerate} 
		
	\item A backup of the votes is stored within the system. This
	backup can be used to count the votes after which the whole
	process can be repeated.

	\item The votes can be exported in encrypted form. A separate
	system can then decrypt the votes and count them. The
	fingerprint of this file can be compared to the fingerprint of
	the that is stored in the audit log in order to ensure that
	the votes are the same ones as the ones that are actually
	casted. This forms, within the context of the KOA system, the
	most logical interpretation of the lawfully required
	possibility of a recount: a facility that allows a separate
	system to count the votes with a check that it are indeed the
	same votes that were actually casted.

	\item The decrypted votes can also be exported. This exported
	file can directly be loaded into a separate system that can
	then count the votes.

	\end{enumerate}

\end{itemize}

\section{TSM (Telephonic voting machine)}\label{sec2:tsmt-voting-mach}

The TSM forms a so called Voice Response system. It asks questions to
the Voter in a predetermined order. The Voter answers by pressing the
keys of his/her phone.

A system with a menu is not used, because the Customer demanded that the
votes per telephone should be protected against eaves dropping. It is
assumed that the Voter has his/her overview of the candidate lists
ready for use. 

The scenario for a Voter that uses this facility goes as follows:

\begin{itemize}
	\item The Voter hears a short welcome message as confirmation
	that he/she has dialed the correct telephone number.

	\item Then the Voter is asked to first give his voter code and
	then the access code that is chosen by the Voter
	him/herself. It is required that this access code is numerical
	and has a predetermined length. This length is chosen to be 5
	digits\footnote{The 5 digit code forms a compromise. A longer
	code gives more security, but makes the chance of errors
	bigger. Banks use a 4 digit PIN, but this is in combination
	with a physical token (the bank card). A relevant example
	forms the Dutch taxing bureau, which also uses a 5 digit PIN
	for online transactions.

	The rule that the access code is restricted to 5 digits is
	also used for voting via the PC, because one of the functional
	requirements states that the Voter can chose at the last
	moment which voting facility he/she will use.}. The access
	code is used to ensure that somebody who intercepts the voter
	code can't vote for that Voter.

	The TSM checks the incoming data with the VSL. If the data is
	incorrect the Voter has two more tries before the connection
	is closed.
	
	\item Then the Voter is asked to give the number of the
	candidate he/she wants to vote for. Again, because of security
	concerns an acknowledgment is given by reading the candidate
	code back to the user. This only confirms that the Voter voted
	for the correct person. One could also speak the name of the
	candidate, but this ignores the security restrictions (i.e.\
	eaves dropping). 
	
	If the Voter does not confirm his/her vote then he/she can
	continue with a new vote (phase 3) or cancel the process.

	\item The TSM sends the voter data to the VSL and receives a
	transaction code. This code is read out to the Voter
	(depending on the system parameter that allows this, see
	Section~\ref{sec2:vslv-poll-stat}). Then the connection is
	closed. At the moment that the VSL receives the vote it is
	final (unless a technical error occurs when the vote is
	stored). So this is still the case if the Voter disconnects
	before the transaction code is received.
	

\end{itemize}

The Chairman's status commands are redirected to the TSM via the
VSL. The TSM becomes active at the moment that the initialization
command is received; until this moment each call receives a short
standard message stating that the elections haven't started yet. The
same holds when the system is interrupted (``Elections interrupted'')
and when the TSM has no live connection with the VSL (``The voting
service is not accessible at this moment''). After the elections have
ended the standard message is substituted for a message which states
that the elections have closed.

The TSM checks at three moments if the VSL is accessible (and the
elections are open): when the Voter is verified (stage 2), when the
candidate is checked and when the vote is casted. The Voter will be
warned at moments that he/she him/herself undertakes an action; the
dialog will never be interrupted with a warning that the voting
station is closed or the elections are interrupted.

\section{WSM (Web voting machine)}\label{sec2:wsm-web-voting}

The Web Voting machine (WSM) is a standard web-server. All web-pages
of the WSM conform, as far as appropriate, to the demands of the
Customer\footnote{See Section 8 of the initial proposal.}. The following
guidelines have been used for implementation of the WSM:

\begin{itemize}
	\item The interface has been kept as sober as possible. This
	ensures that a minimum of data has to be transfered, which
	ensures that voters with a slow connection can still use this
	facility.
	
	\item The requirements for the browser are kept to a
	minimum. The most important requirement is that the browser
	should support encrypted connections; almost all browsers
	conform to this. The browser also has to allow session cookies
	(cookies which are removed when the connection is
	aborted). This is almost always the case, even for PC's behind
	a firewall\footnote{SSL version 3.0 or TLS (the official name
	for SSL V3.1) is used for the secure connection. All
	communication uses the standard port 443 (for https)}.

	Users that try to connect to the standard http port are
	immediately redirected to the secure pages.

\end{itemize}

The menu structure of the web-interface is as follows:

\begin{enumerate}
	\item The Voter first sees a couple of pages which briefly
	explain the voting process, then a verification screen is
	displayed which requires the Voter to fill the voter code and
	in his/her (self chosen) access code.

	The WSM verifies these with the VSL. If the data is incorrect
	the voters can try a predetermined number of times to obtain;
	if the verifications keeps on failing an error message is
	displayed. A new attempt is no longer possible, the voter has
	to log in again. It is also possible that the polling station
	is not open yet or that the voter is blocked; in these cases
	an appropriate error message is displayed and the Voter cannot
	try to log in again.

	If it turns out, during verification, that the Voter has
	already voted then a message is displayed which states this
	together with the earlier acquired transaction code. This
	ensures that the Voter can verify if his/her vote has been
	registered, e.g.\ because the connection was interrupted
	before the transaction code was displayed during an earlier
	attempt. 

	\item In stage 2 the Voter sees a screen that allows him/her
	to enter the candidate code. It is assumed that the Voter has
	his/her list with candidate codes within reach. This candidate
	code list also lists the candidate code for a blanc vote. This
	implementation was chosen to keep the interfaces via the web
	and phone as closely related as possible.

	\item The vote is transfered to the VSL in stage 3, in order
	to verify if the candidate code is correct, i.e.\ is the
	candidate code equal to the number that is used for a
	candidate in the voter circle of that particular Voter. If
	this is correct the system displays a screen that contains the
	name of the list (political party) of the candidate and the
	name and place of residence of the candidate. From this screen
	the Voter can either confirm his/her vote or return to stage
	2.

	\item In stage 4 the WSM transfers the vote of the Voter to
	the VSL which confirms the vote by sending a transaction code
	back to the WSM. This transaction code is then displayed on
	the final screen (depending on the corresponding system
	parameter, see Section~\ref{sec2:vslv-poll-stat}. At the
	moment that the vote is transfered to the VSL the vote is
	final (unless a technical error occurs when the vote is
	stored). Thus this is also the case if the Voter aborts the
	connection before the transaction code can be displayed.

\end{enumerate}

The system checks at three different occasions, the same as with the
TSM, if the polling station is open: when verifying the data of the
Voter (stage 2), when the candidate is checked (stage 3) and when the
actual vote is casted (stage 4). The Voter is only warned that the
polling station has closed when he/she him/herself commits an action;
so no pop-up windows (or the like) are used to warn the user that the
elections have stopped or are interrupted.

All status commands of the Chairman are redirected to the WSM via the
VSL. The WSM becomes active at the moment that the command to start
the elections has been received; until this moment a standard message
is displayed that states that the elections haven't started yet. When
the elections are interrupted another message is displayed
(``elections interrupted''). If the WSM can (temporarily) not connect
to the VSL the message ``the voting service is temporarily not
available'' is displayed.

When the elections are closed the standard message is replaced by a
message which stated that the polling station is closed.	

The WSM is only used for voting. A separate web-server has to be used
for information purposes and other such activities.


%%% Local Variables: ***
%%% mode:latex ***
%%% TeX-master: "funDesign.tex"  ***
%%% End: ***
