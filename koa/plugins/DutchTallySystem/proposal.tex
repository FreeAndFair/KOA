% $Id$ 
\documentclass{article}

\newif\ifpdf
\ifx\pdfoutput\undefined
   \pdffalse     % no PDFLaTeX
\else
   \pdfoutput=1  % PDFLaTeX
   \pdftrue
\fi

\ifpdf
\usepackage[pdftex,bookmarks=false,
            plainpages=false,naturalnames=true,
            colorlinks=true,pdfstartview=FitV,
            linkcolor=blue,citecolor=blue,urlcolor=blue]{hyperref}
\else
\usepackage[dvips]{hyperref}
\fi

\usepackage{url}
\usepackage{xspace}
\usepackage{eurosym}

\title{Proposal for `Hertelapplicatie'}
\author{Security of Systems Group, University of Nijmegen \\
Contact person: Engelbert Hubbers \\
Email: {\tt hubbers@cs.kun.nl}}

\newcommand{\myhref}[2]{\ifpdf\href{#1}{#2}\else\texttt{#2}\fi}
\newcommand{\eg}{e.g.,\xspace}
\newcommand{\ie}{i.e.,\xspace}
\newcommand{\etc}{etc.\xspace}

\begin{document}
\maketitle

\section{Introduction}
We propose to design and implement the `Hertelapplicatie'
vote-counting system as specified in the call for proposals \cite{CFPBZK}
in a rigorous fashion.  Our
methodology guarantees that the system is correct, that it fulfills
the specification of the Ministry of the Interior and Kingdom Relations (BZK), and that it is better specified and
documented than any system designed and implemented by competing
companies.

\subsection*{Motivation}
We have a two-fold reason for making this proposal in this fashion.

The first reason is that our research group has an interest (both
scientific and societal) in systems for electronic voting, as can be
seen from the
publications~\cite{Breunesse_Jacobs_Oostdijk:02voting,Jacobs:03inaug},
and from our earlier involvement in a black-box security test of this
particular electronic voting system.

The second reason is that we would like to use this opportunity to use
our extensive experience in formal methods in software design (i.e.,
rigorous software construction) for an application which is to be
used in practice.  Since the correctness of this application is of the
utmost importance, we want to use the latest proven technology to show
that our solution is superior.

\vspace{0.5cm}
\noindent Note that, because of the international character of our group, we
have written this proposal in English.

\section{Our Proposal}

\subsection{General Information}

In the rest of this proposal we refer to the `Hertelapplicatie' by
calling it `the application'.

We will implement the application in Java.  More precisely, we will
use Sun's Java 2 Platform, Standard Edition (J2SE), version
1.4~\cite{SUNj2se}.  We will also use the `Design by Contract'
methodology to design and construct the system, by using the JML
specification language~\cite{JMLurl}.

JML stands for ``Java Modeling Language''.  JML provides an extension
to the Java language with which one can write formal specifications.
Each core component of a system is formalized by writing
specifications which identify the key properties the final system must
have.  For example, one might specify that the total number of votes
must be a specific value.  

Using specialized tools we can formally verify whether a given
implementation matches the original formal contract\footnote{Note that
  this is only interesting for the calculation engine of the
  application.  We do not have the expertise to fully formally verify
  the system modules relating to the graphical user interface.}.  For
example, one of the tools we will use is the ESC/Java2 tool, which is a
static checker~\cite{ESCurl}.  ESC/Java2 does not need any test input
for its verification process.

In the recently finished European VerifiCard project we gained
significant expertise in this type of work, particularly in the area
of Java-based smart cards~\cite{VERurl}. We have extensive experience
with Java and with its Java Cryptography Extensions (JCEs), which is
used, for instance, in the computer security courses that we teach to
our Masters students.  For this application we intend to use the open
source JCE implementation by Bouncy Castle~\cite{BCurl}.

Since we are employees of a university, being able to refer to this work
for publication is desirable, possibly in a scientific context, but
also perhaps in a public relations context (for instance, for new
students).  We will thus ask, in due course, for permission to do so,
in accordances with the general conditions of BZK listed in \cite{AVODIBZK}.
The fact that the application is intended to be made Open Source makes
it more attractive for us, since we can then refer to it publicly.

\subsection{Our Proposal's Strengths}

The unique selling point of our proposal obviously lies in the use of
formal techniques for ensuring correctness via the specification
language JML and our set of tools. Our JML specifications will be
included as special comments in the source code, which facilitates
code inspection by third parties.

The extra costs for writing down these specificiations will be
balanced by the fact that our testing methology is greatly reduced
compared to alternative approaches.  Also, the resulting system is
much better documented and easier to maintain.

\subsection{Planning}

The call for proposals states that it is the intention of BZK that on
5~March it will become clear which party will implement the
application.  Since the application should be ready by 2~April, this
leaves a period of approximately four weeks to do the job.  We propose
a plan whose goal is to have the software ready within three weeks.

\begin{tabular}{|l|r@{.\ }p{9.5cm}|}
\hline
Week &\multicolumn{2}{l|}{Task} \\ \hline
11   & 1& Study system specification and related materials provided by BZK \\
     & 2& Write a high-level technical design of the system \\
     & 3& Search for and choose Open Source libraries for decryption, XML and PDF routines \\
     & 4& Write formal contracts for the main module \\
     & 5& Implement the main module \\
     & 6& Verify the implementation with respect to the specifications \\
     & 7& Implement the helper methods for connecting to the libraries \\
     & 8& Design the graphical user interface \\ \hline
12   & 1& Implement the graphical user interface \\
     & 2& Integrate the GUI with the main module and the helper modules \\
     & 3& Perform functional testing on the application \\
     & 4& Check performance figures for heavy load \\ \hline
13   & 1& Implement installation kit \\
     & 2& Test kit both on Windows and Linux systems \\
     & 3& Write user manual \\
     & 4& Write supplimentary technical documentation \\ \hline
14   &\multicolumn{2}{l|}{Release application} \\ \hline
TBD  &\multicolumn{2}{l|}{Support for acceptance test} \\ \hline
24   &\multicolumn{2}{l|}{On-site support} \\
\hline
\end{tabular}

\subsection{Time Estimate}

We estimate that this project will take three person weeks at 80\% time.
This is a total of 288 hours.

\subsection{Deliverables}

Because the project is supposed to run only three weeks, we do not
plan on providing any intermediate deliverables.  The final
deliverables will be:
\begin{itemize}
\item An application conforming to the specification put forth
  in the call for proposals~\cite{PVEBZK}.
\item An installer. We develop the application on the Linux platform.
  Since it will be written in Java, it is easy to provide multiple
  installers.  We will provide installers for at least the Linux and
  Windows operating systems.
\item A user manual written in Dutch. This will be a PDF document.
\item Technical documentation in English. This documentation will
  comprise detailed Javadoc- and JML-based documentation for the core
  system describing all its modules.
\item Source code. Apart from the source code of the application
  itself, we will also include the source code of all Open Source
  libraries used.
\item On-site support limited to a few hours.
\end{itemize}

\subsection{Testing and Performance Evaluation}

As specified before, the testing process will be divided into two
parts: a functional test and a performance test.  

Obviously, the functional test will resemble the steps specified
in~\cite{PVEBZK}.  In order to perform such a test we will need to
have sample input files.  This includes files with candidate lists,
encrypted export files and, of course, the corresponding key pairs.

For this reason we intend to write an encryption module to generate
these export files.  However, since we will probably use a different
JCE implementation than the KOA application, it would be necessary for
us to be given additional sample files actually generated by the
latter.

The performance test will be performed on a desktop machine
approximating the specifications put forth in the call for proposals.

\subsection{Requirements}

Given the extremely aggressive schedule of this project, we put the
following requirements on BZK:
\begin{itemize}
\item A reasonable amount of test data generated by the KOA system
  must be made available \emph{prior to our beginning of the project}.
  We cannot be expected to rigorously design and construct a system
  that is this data-depended without a comparable rigorous
  specification of the data.
\item The design of the GUI specified at the start of the project is
  \emph{not} subject to revisions.  If BZK wants a more refined
  user interface beyond that which is stipulated by the current call
  for proposals, it is the responsibility of BZK to separately
  contract such revisions (either with us or with another party) only
  \emph{after} this contract has ended.
\item We offer only basic installation and initial usage help for this
  system.  If BZK needs more extensive assistance at a later point in
  time, it must be covered under a separate contract.
\end{itemize}

\subsection{Insurance}

Since we are not a commercial company, our work is {\em not} covered
by an appropriate insurance as stated in~\cite[article 25]{AVODIBZK}.
However, given our earlier track record, and given the limited size
(and cost) of the application, we do not foresee problems.  We do, of
course, accept the liability clauses.

\subsection{Team Members}

The work will be accomplished by the following staff of the Security
of Systems group at the University of Nijmegen:
\begin{itemize}
\item Dr.~Engelbert Hubbers, URL: \myhref{http://www.cs.kun.nl/~hubbers/}{http://www.cs.kun.nl/$\sim$hubbers}

\item Dr.~Joseph Kiniry, URL: \myhref{http://www.cs.kun.nl/~kiniry/}{http://www.cs.kun.nl/$\sim$kiniry}

\item Dr.~Martijn Oostdijk, URL: \myhref{http://www.cs.kun.nl/~martijno/}{http://www.cs.kun.nl/$\sim$martijno}
\end{itemize}

\noindent The general supervision is in the hands of the group's
leader, Prof.~dr.~Bart Jacobs.

All three team members are currently employed as postdoctoral scholars
in the Security of Systems group.  The tariff for a post-doc on a
commercial project such as this one is set to \euro{85} per hour per
person.

\subsection{Total Cost}
Based upon the number of hours presented above, this proposal implies a
total estimated cost of \euro{24.480} including tax.

\bibliographystyle{plain}
\bibliography{proposal}

\end{document}


