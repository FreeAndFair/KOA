%
% $Id: paper.tex 208 2006-04-04 02:51:21 +0000 (Tue, 04 Apr 2006) alanm $
%

\documentclass{llncs}
\usepackage{llncsdoc}
\usepackage{times}
\usepackage{ifpdf}

\ifpdf \usepackage[pdftex]{graphicx} \else
\usepackage{graphicx}
\fi

\usepackage{xspace}
\usepackage{tabularx}
\usepackage{epsfig}
\usepackage{amsmath}
\usepackage{amsfonts}
\usepackage{amssymb}
\usepackage{eucal}
\usepackage{float}

\ifpdf \usepackage[pdftex,bookmarks=false,a4paper=true, 
plainpages=false,naturalnames=true, colorlinks=true,pdfstartview=FitV, 
linkcolor=blue,citecolor=blue,urlcolor=blue, pdfauthor="Joseph R. 
Kiniry"]{hyperref} \else 
\usepackage[dvips,linkcolor=blue,citecolor=blue,urlcolor=blue]{hyperref} \fi

\newcommand{\todo}{\textbf{TODO:}}
\newcommand{\tablesize}{\footnotesize}
\newcommand{\eg}{e.g.,\xspace}
\newcommand{\ie}{i.e.,\xspace}
\newcommand{\etc}{etc.\xspace}
\newcommand{\myhref}[2]{\emph{#2}}
\newcommand{\NL}{the Netherlands\xspace}
\newcommand{\Votail}{Vot{\'a}il\xspace}

%=====================================================================

\begin{document}

% Omit page numbers and running heads.
% take the % away on next line to produce the final camera-ready version
\pagestyle{empty}

% --- Author Metadata here ---
%\conferenceinfo{TGC}{2006 2nd Symposium on Trusted Global Computing}
%\setpagenumber{50}
%\CopyrightYear{2006}
%\crdata{0-12345-67-8/90/01}  % Allows default copyright data (0-89791-88-6/97/05) to be over-ridden - IF NEED BE.
% --- End of Author Metadata ---

\title{The KOA Remote Voting System:\\
A Summary of Work To Date}

\author{Joseph R.~Kiniry, Alan E.~Morkan, Dermot Cochran and Fintan Fairmichael\inst{1}, \\
Patrice Chalin\inst{2}, Martijn Oostdijk and Engelbert Hubbers\inst{3}}

\institute{School of Computer Science and Informatics\\
University College Dublin\\
Belfield, Dublin 4, Ireland\\
\vspace{3 mm}
\and Department of Computer Science and Software Engineering\\
Concordia University\\
Montreal, Quebec, H3G 1M8, Canada
\vspace{3 mm}
\and Nijmegen Institute of Information and Computing Sciences\\
Radboud University Nijmegen\\
Postbus 9010, 6500GL Nijmegen, The Netherlands}

\maketitle
%======================================================================
\thispagestyle{empty}
\begin{abstract}
Remote internet voting incorporates many of the core challenges of trusted 
global computing. In this paper, we present the Kiezen op 
Afstand\footnote{``Kiezen op Afstand'' is literally translated from Dutch as 
``Remote Voting.''} (KOA) system. KOA is a Free Software, remote voting system 
developed for the Dutch government in 2003/2004. In addition to being Open 
Source, it is also partially formally specified and verified. This paper 
summarises the work carried out to date on the KOA system.  It charts the 
evolution of the system, from its initial conception by the Dutch Government, 
through to its current status. It also describes a roadmap of milestones 
towards completing its next release: a Free Software, general-purpose, formally 
specified and verified internet voting system, that incorporates Proof Carrying 
Code technology for software update and allows trustworthy voting from a mobile 
phone.  We propose that the KOA system should be used as an
\emph{experimental platform} for research in electronic and internet
voting; we are \emph{not} saying that we have solved any of the major
problems inherent in voting with computers.
\end{abstract}
%=====================================================================
\section{Introduction}

The Netherlands is known for its forward-thinking and progressive government, 
laws, and policies. Unfortunately, a government's progressiveness, particularly 
with respect to the adoption of new technology, is sometimes contrary to the 
good of its citizens.

Accordingly, in order to help avoid such a situation in the adoption of remote 
voting technology in \NL, the Security of Systems (SoS) Group at Radboud 
University Nijmegen became directly involved in the evaluation and development 
of the KOA system in 2004.

%~~~~~~~~~~~~~~~~~~~~~~~~~~~~~~~~~~~~~~~~~~~~~~~~~~~~~~~~~~~~~~~~~~~~~
\subsection{Voting Machines in the Netherlands}
The introduction of such a system was not as radical a development as it might 
be considered elsewhere. Electronic voting machines (EVMs) were introduced 
without controversy in \NL around 1998. They have been widely used in local and 
national elections ever since. The primary supplier of these machines is 
Nedap\footnote{Nedap --- \url{http://www.nedap.com/}.}, the same supplier as in 
Ireland.

Part of the reason that EVMs were so readily accepted is historical.  The 
Netherlands has used digital voting machines\footnote{The previous-generation 
systems with little-to-no software.} since the 1980s. Therefore, Dutch citizens 
are comfortable with the idea of using technology for voting. The security and 
reliability issues of the new generation of machines was not a serious problem 
at the time of their introduction, much like their adoption by other 
governments in the late 1990s.

Unfortunately, many aspects of these systems have not been made
public, contrary to the requests of concerned parties in \NL.  The
internals of such systems are secret and are only exposed to
evaluators.  Each system must be examined, according to an unknown set
of criteria, before being accepted by the Dutch parliament for use in
elections. Evaluation reports compiled by the national reviewer,
TNO\footnote{TNO --- Netherlands Organisation for Applied Scientific
  Research ---\\ \url{http://www.tpd.tno.nl/tno/index.xml}.}, are also
secret.

However, as attention has been focused the world over on EVMs, the Dutch 
parliament has begun to re-evaluate its approach.  Changes to the current 
systems are likely to be mandated soon, particularly with respect to voter 
verifiable paper trails.

In keeping with this reassessment, the Dutch parliament decided to conduct 
experiments with the next natural step in the use of technology for voting: 
remote voting using both the internet and telephone.  The main inspiration 
is that, nowadays, many personal transactions (\eg banking), can be carried 
out from arbitrary locations, so why not voting?

Indeed, it is believed by some that a remote voting system will increase 
electoral participation by making the process more convenient. Currently, Dutch 
citizens must find time during the extended business hours (08:00 to 20:00) of a 
single day of the working week.  Furthermore, each individual must vote in a 
particular location near their home, which may be far from their workplace.

However, given what we know about the unreliability and vulnerability of 
software and networks, do the risks inherent in the introduction of such a 
system outweigh such benefits?

These risks, together with the methods adopted in eliminating and minimising 
them in the KOA system form the basis for the rest of this paper. It is 
organised as follows. Section~\ref{sec:kiezen-op-afstand} presents some 
background information on the genesis of the KOA project. Past academic work on 
the system up to the end of 2005 is presented in 
Section~\ref{sec:academic-past-work}. A security assessment of the KOA system is
put forward in Section~\ref{sec:security}. 
%Alan: Was originally edited out due to space restrictions
Current work is discussed in Section~\ref{sec:acad-curr-work}. 
Related work is compared and contrasted in 
Section~\ref{sec:related-work}. Future work is considered in 
Section~\ref{sec:future-work} and Section~\ref{sec:conclusion} concludes.


%=====================================================================
\section{Kiezen op Afstand (KOA)}
\label{sec:kiezen-op-afstand}

The genesis of KOA stemmed from a promise made by the Dutch government to 
parliament that they would investigate possible developments to the Dutch 
voting system. This promise was fulfilled in the KOA experiment by allowing 
expatriates to vote in the elections to the European Parliament via the 
internet and by telephone.

However, Dutch national election law is quite explicit about what is permitted 
with respect to how votes may be cast. Therefore, in order to conduct an 
experiment in voting over the internet, some amendments to this general law 
were formulated. This formed the legal foundation for the KOA project.

% Alan: Was originally edited out due to space restrictions
Apart from the general rules governing internet voting, it also included some
additional rules detailing a citizen's right to vote from a different polling
booth other than the one originally appointed. However, in this paper we will
refer to the KOA project as if it consisted purely as an internet voting
experiment.

%~~~~~~~~~~~~~~~~~~~~~~~~~~~~~~~~~~~~~~~~~~~~~~~~~~~~~~~~~~~~~~~~~~~~~
\subsection{Internet Voting in \NL}

The elections to the European Parliament of June 2004 allowed remote voting via 
the internet and telephone.  It was limited to expatriates who were required to 
explicitly register beforehand.  It was thought that such a small-scale use 
(thousands of voters) would provide a useful real-world test for the technology.

The main reason why it was thought that an internet-based solution was suitable 
is decidedly non-technical.  Essentially, by significantly constraining the 
remote voting problem, particularly with respect to the registration and voting 
process itself, it was believed that a ``sufficiently secure'' and reliable 
system could be constructed.  In particular, the system needed to be at least 
as secure and reliable as the existing remote voting system which was based 
upon postal ballots.

%---------------------------------------------------------------------
\subsubsection{The Remote Voting Process}

When a citizen registers to use KOA, the voter chooses their own personal 
access code (a PIN). Some time later, a customised information packet is mailed 
to the voter. This packet contains general information about the vote itself 
(date, time, \etc), as well as voter-customised details that are known to only 
that voter. These details include information for voter authentication, 
including an identification code and the previously chosen access code.

Also included is a list of all candidates. Each candidate is assigned
a large set of unique random numbers\footnote{1,000 codes were
  generated for each candidate for the elections to the European
  Parliament in 2004.}, and exactly one of those numbers is given to
each voter. The set of codes per voter is determined randomly but is
not unique.

To vote, a registered voter logs in to a web site with their voter code and 
access code. They then step through a series of simple web pages, typing in 
their candidate codes as appropriate for their choices. The system shows the 
voter the actual names and parties of the candidates in question to confirm the 
accuracy of the vote. When a voter is finished, a transaction code is provided. 
This code can later be used to check in a published list that the voter's
choices were included correctly in the final tally.

Communication with the voting web site is secured with SSL. All votes are 
stored in a doubly-encrypted fashion; each vote is encrypted by a symmetric key 
per voter\footnote{This symmetric key is generated by hashing the assigned
identification code.} and the public key of the voting authority.
	
%~~~~~~~~~~~~~~~~~~~~~~~~~~~~~~~~~~~~~~~~~~~~~~~~~~~~~~~~~~~~~~~~~~~~~
\subsection{Use and GPL Release} 

The trial during the elections to the European Parliament in June, 2004 was 
restricted to roughly 16,000 eligible Dutch expatriates. Expatriates could vote 
either via the internet or by telephone. The telephone votes were fed into the 
KOA tally system. 5,351 people used one or other system.

Subsequently, in July 2004, the Dutch Government released the majority of the 
source code for the KOA system under the GNU General Public License (GPL) 
making it the first Open Source internet voting system in the world.
%=====================================================================
\section{Academic Past Work}
\label{sec:academic-past-work}

% Alan: reviewer considered this way of splitting up the work by university
% to be unnatural so it's being commented out
% Three research groups have now worked on the KOA system: Radboud University 
% Nijmegen in the Netherlands, Concordia University in Canada, and University 
% College Dublin in Ireland.  The work, which has both research and development 
% aspects, is summarised in this section.
% 
% %~~~~~~~~~~~~~~~~~~~~~~~~~~~~~~~~~~~~~~~~~~~~~~~~~~~~~~~~~~~~~~~~~~~~~
% \subsection{At Radboud University Nijmegen}
% 
% Several pieces of work were accomplished in Nijmegen in 2004 and 2005. This 
% work included participating in an expert panel, performing a network security 
% evaluation, and the development of a vote counting system with formal methods.

%---------------------------------------------------------------------
\subsection{External Security Evaluation}
\label{subsubsec:ddos}
In late 2003 Prof. Bart Jacobs of the Security of Systems (SoS) group at Radboud
 University Nijmegen participated in an external review of the requirements and 
design of this application. One of the recommendations made by the panel was 
that the system should not be designed, implemented and tested all by the same 
company.

The system itself was designed and implemented by 
LogicaCMG\footnote{\url{http://www.logicacmg.com/}.}.  Although eventually the 
government decided to make the system open source, during implementation it was 
not.  In order to improve its quality, the Dutch company Software Improvement 
Group\footnote{\url{http://www.sig.nl/}.} performed a code review of the system. 
However, they were only allowed to do so after signing Non-Disclosure 
Agreements (NDAs). In fact, it was unexpected that the government
ultimately opted for an Open Source solution.

% Alan: Joe - Could you give more details on how the penetration test 
% on the voting servers works?

% Joe: I have expanded on the evaluation below.  Please review it and
% trim if you like.

The SoS group did not take part in the design or implementation of the system. 
However, the group took an active part in the final stages of the project. The 
group performed two tasks: it wrote an independent tally application which will 
be explained in detail in Section~\ref{subsubsec:vote-counting-system} and it 
performed a penetration test on the vote servers.

The penetration test was set up as a black box test.  In particular the SoS 
group had virtually no knowledge regarding the actual hardware, software, 
networks or personnel involved with the server system. Indeed, the information 
it did possess could have been considered public information since it could 
easily be obtained by standard available analysis tools.

The main goal was to break into the system and try to compromise its
integrity.  The second goal was to test whether the system was
vulnerable to denial of service attacks.

Two evaluations were conducted.  The first was unsolicited and took
place during a private beta test of the system.  The second was
requested by the government, primarily because of the results of the
first evaluation.

During the first unsolicited evaluation the subnet running the service
was gently probed and mapped using nmap, a more detailed evaluation of
specific machines was then conducted, specifically with regards to
machines running inappropriate services, weaker operating systems,
\etc, and finally, on the last afternoon of the test, a
denial-of-service attack on the machines was conducted.

The main discovery of the first evaluation was that the system was not
``tightened down'' insofar as test and management machines which were
running insecure versions of particular operating systems (\eg
Microsoft Windows) were on the deployed subnet, no hardware or
software firewall was in place on the system, machines has likely
external exploits available, and nearly all systems had inappropriate
services running (\eg unused mail servers, databases, file sharing,
\etc).  Also, the SoS group was able to significantly harm their
service quality with our (admittedly very small) denial-of-service
attack.

After the authorities realized the SoS group was responsible for this
attack they asked us for a report of our findings.  Given the feedback
and analysis, they then asked the SoS group to perform an ``official''
external evaluation once they incorporated all of our suggestions and
tightened-down the network.

The second evaluation found that their systems were adequately hosted,
monitored and configured, their software was up-to-date, and no
unnecessary services were running.  Furthermore, adequate measures
were in place for detecting basic probes by adversaries.  Thus, in the
end, the SoS group did not find any problem with the system that would
have caused the Dutch Ministry to reject it for an experimental run,
and the external evaluation significantly improved the security and
reliability of the system.

%---------------------------------------------------------------------
\subsection{Vote Counting System}
\label{subsubsec:vote-counting-system}

As seen in the previous section, one of the results of the recommendation to 
split the responsibilities of the parties involved, was that the government 
decided to accept bids for the creation of a separate vote counting subsystem, 
to be implemented in isolation by a third party. This separate tally 
application would allow the vote counting to be independently verified. The SoS 
group put forward a proposal to write this application, and were successful in 
this bid. The key idea behind their tender was that the vote counting program 
should be formally verified using the JML~\cite{BurdyEtal05-STTT} and 
ESC/Java2\footnote{ESC/Java2 is a programming tool that attempts to partially 
verify JML annotated Java programs by static analysis of the program code and 
its formal annotations. It translates the specifications into verification 
conditions that are modularly discharged by an automatic theorem 
prover.}~\cite{KiniryCok04} tools.

The vote counting system formed a small but important part of the whole KOA 
system. This provided the SoS group with a suitable opportunity to test the use 
of some of the formal techniques and practices that they had been developing. 
Given the severe time constraints placed upon them due to the impending 
election, the application was built by three members of the group over a 
barely-sufficient period of four weeks. Java was chosen as the programming
language 
in which to implement the system so that JML could be used as the formal 
specification language. Due to the time constraints, verification was only 
attempted with the core modules.

Counting votes within KOA proceeds offline using a separate tally application. 
The input to this application consists of two XML files (one containing the 
list of candidates and their codes, and one containing the encrypted votes), 
and a public/private keypair used to decrypt the votes.

As the informal requirements of vote-counting are obvious (for every candidate 
in the candidate list count the number of votes for that candidate), the 
functional specification~\cite{KOAspec} (in Dutch) mostly prescribes details of 
file formats and encryption algorithms to be used.

Nevertheless, the functional specification does impose some requirements that 
greatly influence the structure of the Java application and its JML 
specification. First, the different tasks that 
need to be performed in order to count the votes (reading in the two files, 
reading in the keys, decrypting the contents of the votes file, counting the 
votes, generating reports) are made explicit in this document and, more 
importantly, the order in which they have to be performed is specified. Second, 
the document provides a rough sketch of the user interface and its contents. 
Finally, the document gives some bounds on the data, such as the lengths of 
fields or the maximum number of candidates in each list, which are incorporated 
in the JML specifications of the data structures.

In accordance with the above high-level specification, the resulting tally 
application consists of some 30 classes, which can be grouped into three 
categories: the data structures, the user interface, and the tasks.

The data structure classes form an excellent opportunity to write JML 
specifications. Typical concepts from the domain of voting, such as candidate, 
district and municipality can be modeled with detailed JML specifications. An 
example invariant in Candidate.java is:

\begin{verbatim}
/*@ invariant my_gender == MALE || 
  @           my_gender == FEMALE ||
  @           my_gender == UNKNOWN;	
 */
\end{verbatim}

The different tasks associated with counting votes were mapped to individual 
classes. After successful completion of a task, the application state is 
changed. A task can only be started if the application is in an appropriate 
state. The life-cycle model of the application that therefore emerges is 
maintained in the main class of the application inside a simple integral field. 
This life-cycle model can be specified in JML using invariants and constraints, 
essentially stating that on successful completion of the application, the 
application went from ``initial state'' to ``votes counted state''. The state 
of attributes associated with the individual tasks can be linked to the 
application life-cycle state using invariants. For instance, such an invariant 
could read: `after the application reaches the ``keys imported state'', the 
private key field is no longer null'. This is stated in
\texttt{MenuPanel.java} as follows:
\begin{verbatim}
/*@ invariant
  @   (state >= PRIVATE_KEY_IMPORTED_STATE 
  @     ==> privateKey != null); 
 */
\end{verbatim}

A graphical user interface is usually not very amenable to formal 
specification. Nonetheless, some light-weight specifications were written. One 
of the requirements defined in the original informal specification was that 
users should not be allowed to start certain tasks before certain other tasks 
are successfully completed. For instance, a user should (by means of the user 
interface) not be able to start decrypting votes before the votes are read in 
from file. In the graphical user interface, this demand is met by only enabling 
certain buttons when the application reaches certain states in the life-cycle 
model. The fact that the graphical user interface complies with the life-cycle 
model can be neatly specified in the GUI classes by referring to the 
application state.
%---------------------------------------------------------------------
\subsection{Process}
As already stated, ESC/Java2 was only used to verify the core of the tally 
application. This means that it was used to verify reading in the XML-files 
with the candidates and the votes, decryption of the votes and counting the 
votes. The final generation of the reports is not checked with JML.

Using JML on reading XML files is quite straightforward. Essentially, for every 
object that is read, some methods are called that specify that the total number 
of objects will be increased by exactly one. Naturally, in order to verify code 
that uses functionality provided in external libraries, some of the 
corresponding APIs must also be specified. The JML community has provided 
specifications for most of the APIs that come with Sun's standard edition of 
Java. However, APIs dealing with cryptography, XML parsing, and PDF generation, 
as used by the tally application had not previously been specified. These APIs 
were specified in a light-weight manner: the specifications mostly deal with 
purity and non-null references in the API methods which makes verification of 
client code using ESC/Java2 much easier.

Naturally, the counting process is likewise formally specified in JML, which 
ensures that each valid vote is counted for exactly one candidate. This also 
implies that specifications are easy to check to make sure that the total 
number of votes a party list receives is equal to the sum of votes for each 
candidate\footnote{Including the `blanco' or `blank ballot' candidate.} on this party 
list.

The JML run-time assertion checker was also used in the development process. 
First, for testing the data structure classes, the checker was used to generate 
unit tests. Second, we ran the full application, including user interface, 
using the checker.
%~~~~~~~~~~~~~~~~~~~~~~~~~~~~~~~~~~~~~~~~~~~~~~~~~~~~~~~~~~~~~~~~~~~~~
%\subsection{At Concordia University}

%---------------------------------------------------------------------
\subsection{Analysis of KOA}
In the Dependable Software Research Group at Concordia University, the KOA
source code was used as a subject of a study in the frequency of occurrences of
non-null reference type declarations~\cite{ChalinRioux05}. This work consisted
in adding nullity annotations (or constraints) and then verifying their
correctness by making use of ESC/Java2. 
% Alan: Was originally edited out to due space restrictions
The results were similar to those of F{\"{a}}hndrich and 
Leino~\cite{fahndrich03declaring}, that is to say, it was found that even a 
simple specification exercise of adding nullity annotations can help uncover 
non-trivial bugs both in the code and in the specifications.

For example, in the \texttt{sos.koa.CounterAdapter} class in the Tally
Application it was found that the field named \texttt{errors} is declared
nullable and yet the method \texttt{getErrors}, which uses this field, assumes
that the field is non-null (\ie a \texttt{Null\-Pointer\-Exception} will not be
thrown).

%~~~~~~~~~~~~~~~~~~~~~~~~~~~~~~~~~~~~~~~~~~~~~~~~~~~~~~~~~~~~~~~~~~~~~
% \subsection{At University College Dublin}
% The work at University College Dublin included developing a full
% FLOSS\footnote{Free/Libre/Open-Source Software.} version 
% of KOA, performing a detailed code and specification review of the tally
% application discussed in Section~\ref{subsubsec:vote-counting-system}, 
% and developing a formal specification of the Irish vote counting system.
%---------------------------------------------------------------------
\subsection{Reverse Engineering Missing Components}
The version of KOA released under the GPL was not complete. A number of pieces 
of functionality, constituting roughly 10\% of the deployed KOA system, were 
proprietary and owned by LogicaCMG. Moreover, certain other changes were made 
for publication purposes (\eg the length of public/private key pairs in the 
source code).

In addition, the released KOA system contains no high-level design 
documentation and very little information on how to build the system. This 
means that it is only possible to inspect the (partial) source, not to compile 
and run it. Therefore, it was necessary to perform a full analysis of the 
released system~\cite{Morkan05}.

One of the most beneficial aspects of this analysis was that errors were found 
in the KOA system. One such error was found in the Java Server Pages (JSPs) 
whereby a button that should have guided the user back to the interface 
homepage had, in fact, the same action as that of the ``submit'' button that 
processed and saved a list of candidates to the database. This was due to a 
trivial mistake: placing the HTML tags for the ``Return Home'' button within 
the \texttt{FORM} tag block. This error was discovered during a trivial ``click
through'' of the user interface followed by an examination of the code.

% Alan: Was originally Edited out due to space restrictions
Such a basic mistake in the design of the user
interface of a critical system is unacceptable. The fact that such a mistake
could be made, remain unnoticed in the testing and evaluation phase of the
software, and actually be used in the elections to the European Parliament,
would suggest that there is in all likelihood further errors in this software.

Once the analysis was complete, the missing functionality was reverse 
engineered. 59 additional classes, together with some properties files, were 
added to the system. These classes carry out the base functionality of the 
servlets, error reporting, logging functionality, event handling, \etc

%---------------------------------------------------------------------
% Alan: Was originally edited out due to space restrictions
\subsection{Full Open Source Foundations}
\label{subsubsec:full-open-source}

One of the major goals in the redevelopment of the KOA system was that
it would be entirely composed of, and dependent upon, Open Source
software.  The original system was developed in, deployed upon and
tightly coupled to the IBM WebSphere IDE.  During the
reimplementation, the KOA system was ported to an Open Source
alternative.  This foundation consisted of a MySQL database server
backend paired with a JBoss application server front-end, the latter
of which incorporated the Tomcat servlet container.  The other major
restriction in terms of making the system fully GPL-compliant was its
use of proprietary security and encryption utilities developed by IAIK
and Sun.  These were seamlessly replaced using the BouncyCastle Open
Source alternatives.
% 
%---------------------------------------------------------------------
\subsection{Formal Specification and Extended Static Checking Review}

As has already been stated, the Vote Counting Application of the KOA system was 
specified with formal methods, extensively tested and partially verified to the 
extent that was possible within the given time-frame. Subsequently, efforts were 
made to complete the specification and verification~\cite{Fairmichael05}.

When the KOA vote counting system was being designed, precedence was given to 
verifying the core units. These were designed by contract and as result have 
good specification coverage. The remaining parts, however, were only lightly 
annotated with JML notation.

% Alan: Was originally edited out due to space restrictions
\begin{table}
\begin{center}
\begin{tabular}{|l|l|l|l|}
\hline & {\bf File I/O } & {\bf Graphical I/O} & {\bf Core} \\ \hline Classes &
8 & 13 & 6 \\

Methods & 154 & 200 & 83 \\

NCSS & 837 & 1599 & 395 \\

Specs & 446 & 172 & 529 \\

Specs:NCSS & 1:2 & 1:10 & 5:4\\ \hline
\end{tabular}
\end{center}
\caption{KOA initial release system summary}
\label{table}
\end{table}

Table~\ref{table} summarizes the size (in number of classes and methods),
complexity (non-comment size of source (NCSS)), and specification coverage of
the three subsystems, as measured with the JavaNCSS tool version 20.40 during
the week of 24 May, 2004. This is the version of the program that was released
and used in the elections to the European Parliament in June 2004.

At the time of its initial release, verification coverage of the core subsystem 
was good, but not 100\%. Approximately 10\% of the core methods (8 methods) 
were unverified due to issues with ESC/Java's Simplify theorem prover (\ie
either the prover did not terminate or terminated abnormally). Another 31\% of 
the core methods (26 methods) had postconditions that could not be verified, 
typically due to completeness issues in ESC/Java, and 12\% of the methods (10 
methods) failed to verify due to invariant issues, most of which are due to 
suspected inconsistencies in the specifications of the core Java class 
libraries or JML model classes. The remaining 47\% (39 methods) of the core 
verified completely. Since 100\% verification coverage was not possible in the 
time-frame of the original project, to ensure the KOA application was of the 
highest quality level possible, a large number unit tests were 
generated\footnote{The tool generates unit tests that deal with 
\emph{interesting} values. Interesting values are generally boundary values for 
a given data type. For example, -1, 0, 1, \emph{n} and \emph{n+1} for an array 
of integers. Users are also free to handwrite their own test cases, in the case 
where the jmlunit tool does not test all important values.} for all core 
classes with the jmlunit~\cite{Cheon-Leavens02} tool, which is part of the JML 
suite. A total of nearly 8,000 unit tests were generated, focusing on key 
values of the various datatypes and their dependent base types. These tests 
cover 100\% of the core code and are 100\% successful.


After this analysis was completed, the specifications were gradually augmented. 
As an example, consider the \texttt{AuditLog} class. This class records 
information about the vote counting as the application proceeds. This 
information is then used at the end of the vote counting to help fill in the 
details for two of the reports that are generated. This class keeps track of 
the program's progress in a similar manner to that which was used for the 
overall program state. There were multiple invariants used to ensure the 
program and auditing proceeded in the correct fashion. Several corrections were 
required for this class, the bulk of which were modifications to the behaviours 
of the methods that allowed the audit log's state to change. The original 
specifications allowed the possibility that the variables could be changed to a 
state where the invariants would not hold. The changes made to this class' 
specifications disallowed any actions that would violate the object invariants.
% Ala: Possible examples of these changes here?

%---------------------------------------------------------------------
% Alan: Was originally removed due to space constraints
\subsection{Documentation Writing and Translation}
\label{subsubsec:docum-writ-transl}

The vast majority of the voting system, including high-level documentation, web
interfaces, Java comments and variable names are in Dutch. Furthermore, much of
the voting system is sparsely commented and unspecified. This clearly poses an
obstacle to the understanding and adoption of such a system by a wider,
international audience. It was therefore decided at an early stage that a
complete translation of the system into an international language such as
English, together with the production of additional documentation, was
necessary in order to facilitate a larger number of people to carry out the
necessary specification, development and testing. Consequently, the major
high-level specification document and all of the JSPs have been translated from
Dutch into English.

%---------------------------------------------------------------------
\subsection{Other Voting Systems}
\label{subsubsec:other-voting-systems}
Naturally, there are relatively considerable variations in electoral systems 
between countries. This is the case between \NL and Ireland. Not only are these 
differences linguistic, but more significantly there are different vote 
counting procedures in \NL and in Ireland. The Dutch Voting system is list 
based while Ireland uses Proportional Representation with a 
Single Transferable Vote (PR-STV).

% Alan: Was originally cut due to space restrictions.
\subsubsection{The Irish Voting System}

The D{\'a}il, Ireland's lower house of parliament, is composed of 166 members
representing 41 constituencies. Each constituency elects multiple members to
parliament. The average constituency elects four representatives with every
constituency electing at least three representatives. The system used is
PR-STV. This combination is considered to
increase the representativeness of the D{\'a}il.

Irish voters, by ranking the candidates, give instructions as to who should
receive their support should the first choice candidate be eliminated or
elected. Surplus votes are the number of votes in excess of the threshold of
election a candidate receives. Surplus votes are transferred proportionally to
the remaining candidates according to the indicated second preference of the
voters. If the election is undecided after counting the first preferences and
transferring surplus votes, then the lowest polling candidate is eliminated.
The ballots cast initially in support of this candidate are now counted
according to their indicated second preference. If any candidate has more than
a quota of votes then he or she is elected and his or her surplus votes are
transferred to the next preference candidate.  If there are more candidates
than seats and all surpluses have been transfered, then the candidate with
least votes is excluded and his or her votes transfered to the next preference
on each ballot paper.  This process is repeated until the number of candidates
remaining equals the number of seats remaining.

%---------------------------------------------------------------------
% Alan: The following is being commented out as it is completely out of sync
% % with the rest of the paper.
% \subsubsection{Commission for Electronic Voting}
% The second report of the Irish Commission for Electronic Voting \url{http://www.cev.ie/} found that the existing software used in pilot elections
% was in need of modification and that overall the current electronic voting
% system is slightly less secure than the existing paper based system.  In
% addition to non-technical issues such as the need for parallel testing and an
% independent audit the CEV also recommended that the entire system including the
% software needs to be independently verifiable and testable with respect to the
% documented requirements, and that the software be developed according to a
% formal software engineering methodology.

%---------------------------------------------------------------------
\subsubsection{Formal Specification}
\Votail is the Irish word for voting. The \Votail specification is a JML 
specification for the Irish vote counting system~\cite{Cochran06}. This formal 
specification is derived from the complete functional specification for the 
D{\'a}il election count algorithm~\cite{CEV00,CEV02}.

Thirty nine formal assertions were identified in the Commentary on Count Rules 
published by the Irish Department of Environment and Local Government. Each 
assertion expressed in JML was identified by a Javadoc comment. In addition, a 
state machine was specified so as to link all of the assertions together. Java 
classes were specified for the vote counting algorithm, to represent the ballot 
papers and candidates. A concrete example of how the 
methodology was applied will clarify this work.

Section 7, item 3.2 on page 25 of~\cite{CEV00} states:
\begin{quote}
As a first step, a transfer factor is calculated, viz. the number of votes in 
the surplus is divided by the total number of transferable votes in the last 
set of votes. This transfer factor is multiplied in turn by the total number of 
votes in each sub-set of next available preferences for continuing candidates 
(note that the transfer factor is not applied to the sub-set of 
non-transferable votes in the set of votes).
\end{quote}

The requirement is translated into formal natural language as follows:
\begin{quote}
The number of votes in the surplus is divided by the total number of 
transferable votes in the last set of votes. This transfer factor is multiplied 
in turn by the total number of votes in each sub-set of next available 
preferences for continuing candidates.
\end{quote}

Finally, this formal natural language is formally specified in the
architecture as a JML postcondition for the method that is
specifically for this requirement (the
\\\texttt{get\-Actual\-Transfers} method).  The Javadoc and JML
specification for this method follows.

\footnotesize
\begin{verbatim}
/**
 * Determine actual number of votes to transfer to 
 * this candidate, excluding rounding up of 
 * fractional transfers
 *
 * @see requirement 25 from section 7 item 3.2 
 * on page 25
 *
 * @design The votes in a surplus are transfered in 
 * proportion to the number of transfers available 
 * throughout the candidates ballot stack.  The 
 * calculations are made using integer values 
 * because there is no concept of fractional votes 
 * or fractional transfer of votes, in the existing 
 * manual counting system. If not all transferable 
 * votes are accounted for the highest remainders 
 * for each continuing candidate need to be examined.
 *
 * @param fromCandidate Candidate from which to 
 *        count the transfers
 * @param toCandidate Continuing candidate eligible 
 *        to receive votes
 * @return Number of votes to be transfered, 
 *         excluding fractional transfers
 */

//@ ensures 
//@  \result == 
//@     (getSurplus(fromCandidate) * 
//@      getPotentialTransfers(fromCandidate, 
//@         toCandidate.getCandidateID()) /
//@      getTotalTransferableVotes(fromCandidate);
\end{verbatim}

\normalsize

The \Votail specification was typechecked and checked for consistency
using ESC/\-Java2\footnote{The consistency of JML specifications is
  checked using an experimental extension to ESC/Java2 that
  manipulates the JML abstract syntax tree in order to determine
  whether certain combinations of assertions are inherently
  unsatisfiable.}.

%=====================================================================
\section{Security Assessment}
\label{sec:security}
Issues of security and correctness are paramount in any voting system. This is
especially the case for a remote internet voting system due to the inherent
vulnerabilities of the architecture. Any such system must be as secure as the
system it is designed to replace. Otherwise, trust in the electoral and
democratic systems of a country can be severely damaged.

The KOA system was designed to replace absentee postal ballots. It has always 
been accepted that postal voting is not as secure as voting in a polling 
booth. KOA follows all of the standard security mechanisms and also introduces 
some novel approaches. These security mechanisms are focused on attack 
prevention and, where this is impossible, on detection of intrusion. This 
section discusses these security mechanisms.

\subsection{Data Integrity}
\label{subsec:integrity}
The most significant method used in the KOA system to ensure data integrity is 
the use of candidate codes. 1,000 codes are generated for each candidate and 
only one of these is randomly assigned to each voter. Therefore, even 
if a malicious agent (\eg a worm, virus or Trojan horse) can access a ballot, 
all the attacker can see are the encoded candidate and party IDs, which in the 
optimal case are unique to the voter in question. Consequently, it will be 
virtually impossible to substitute the ballot by choosing the appropriate code 
for a different candidate.

In addition, the votes are doubly-encrypted. The only way to decrypt these votes
on the server side is to close the polls. Closing the polls is an irreversible
action. Consequently, altering the votes at the server-side is precluded.

In the case where the voter tries to cast multiple votes at once (\eg via 
both telephone and internet) there will always be one first vote. This vote 
will be stored. The second attempt will fail because the voter has already cast 
his/her vote.

Finally, the KOA system has the capability to take snapshots of the candidate
and voter lists called ``electronic fingerprints.'' These fingerprints can be
generated at any time to ensure that these lists have not been maliciously
altered. One possible extension to the system is to automate the generation of
these fingerprints at regular intervals to ensure a regular verification of data
integrity. 

\subsection{Verifiability} 

Voters using the KOA system are able to verify that their vote is recorded 
correctly and is included in the final tally of the election using the 
transaction code they receive upon casting their ballot. This is possible due 
to the publication of a list of the transaction codes of votes for each 
candidate after the election.  Such a check can identify any compromised PCs 
and in the worst case invalidate the election.

% Alan: The proposed model in the Geneva system is better than this - discuss it
% here or in the related work section??

\subsection{Insider Threats}

The power to change the state of the system and to decrypt the votes is
restricted to a small number of polling station officials. These officials hold
the private key for the system and each has a PIN code to use this private
key. One of these officials is designated as the current ``president'' or
``chairman.''

In order to change the state of the system (\eg open/close the polls, decrypt 
the votes, \etc), the chairman and one other official must enter their PIN 
codes. If the role of chairman is alternated at set time intervals among random 
officials (or some similar mechanism), then all officials need to be in 
collusion in order to tamper with the system. Even then, access to the 
decrypted ballots is precluded, as is mentioned in Section~\ref{subsec:integrity}.

% Alan: This is being left out at the moment as it's a bit vehement IMO and
% we don't really have space for something so long.

In addition, an insider attack would require massive, undetectable
client and/or network subversion (\eg large numbers of client
computers \emph{and/or} network web proxies being compromised by a
virus written by attacker's henchmen).  Given the scale and complexity
of such an attack, it is nearly inconceivable that it is possible.
Such an attack would be (many) orders of magnitude more difficult to
pull off than any attack on existing electronic or manual voting
hardware/mechanisms due to its scale: millions of PCs versus thousands
of voting machines, and millions of individuals (many of which are
experts like network service providers, IT workers voting from home,
\etc) participating and monitoring the election versus thousands of
volunteers running the election.  This is analogous to the Open Source
``thousands of eyeballs'' argument, but applied to voting.

% Alan: Was originally cut due to space restrictions
\subsection{Other Security Features}
A part from the use of SSL, there are a couple of further noteworthy security
features.

Firstly, random data is added to the votes when they are encrypted. This ensures
that votes within the same voter district and for the same candidate have a
different encryption result for each vote, making it impossible to interpret
encrypted votes.

Secondly, the votes are decrypted in a random order in order to
making tracing voters by the order in which they voted impossible.

\subsection{Problems}
Despite the best efforts to make KOA as secure as possible, certain security
flaws still remain. These need to be addressed before further use of the system.

Firstly, if the electronic fingerprints of the system are not identical at a
particular point in time, the chairman can overrule and allow the election to
continue. This should not be permitted.

Like other forms of remote voting (\eg postal voting), KOA does not provide 
protection for voter anonymity in the case where another person is in the 
vicinity of the voter during the voting process or if another person gains 
access to a voter's transaction code. However, due to the use of candidate 
codes, excluding these two scenarios, it is virtually impossible to connect a 
voter to his/her vote.

\subsubsection{Denial of Service Attacks (DoS)}
As has already been stated in Section~\ref{subsubsec:ddos}, the KOA system is
vulnerable to DoS Attacks. This is practically impossible to
prevent and is a feature of all remote internet voting systems.

One feature of the KOA system that lessens some of the problems caused by DoS 
attacks is that the system can be interrupted. When this state change happens, 
an electronic fingerprint of all the system data is taken and this can be 
checked against a subsequent fingerprint on system resumption. Clearly, this 
does not solve the problem of potential temporary disenfranchisement, but it does
ensure data integrity in the face of a such an attack.

\subsection{Summary}
As has been described, all of the standard security mechanisms have been used 
together with some innovative techniques to ensure data integrity and 
verifiability. However, obviously the issue of security is one of the open 
questions of remote internet voting and there are a number of problems yet to 
be overcome. We believe these problems can be addressed by research and experimentation
on a verified open source framework, like the one which KOA aims to provide.

%Alan: I'd like to put some reference to the following in but I don't think we
% have space and it also seems a bit vehement for this part of the paper.
% In Point, Click, and Vote, voting experts Michael Alvarez and Thad Hall make a strong
% case for greater experimentation with internet voting. In their words, �There
% is no way to know whether any argument regarding internet voting is accurate
% unless real internet voting systems are tested, and they should be tested in
% small-scale, scientific trials so that their successes and failures can be
% evaluated.� In other words, you never know until you try, and it�s time to try
% harder.

%=====================================================================
% Alan: Was originally cut due to space constraints
\section{Academic Current Work}
\label{sec:acad-curr-work}

%---------------------------------------------------------------------
\subsection{Generalisation of System for non-Dutch Voting Systems}
\label{subsubsec:gener-syst-non}

The Java code for \Votail was written in JML using a kind of
``verification-centric'' Design by Contract methodology. This means that not
only are we writing each method implementation according to its JML
specification, but we are checking each method's correctness with ESC/Java2 and
automatically generating thousands of unit tests using
JML-JUnit~\cite{Cheon-Leavens02}.

The KOA system has a state machine similar to that used in the \Votail
specification. This allows KOA to make calls to the appropriate part of the
\Votail code. The \texttt{ElectionAlgorithm} class in \Votail will be invoked
from within the KOA system using the following four method calls:
\texttt{setup}, which defines election parameters such as candidate list and
number of seats, \texttt{load}, which loads all valid ballots and then
calculate quota and deposit saving thresholds, \texttt{count}, which assign
votes to candidates, distribute surpluses and exclude candidates until
finished, and \texttt{report}, which reports the election results. These
methods must be called in the order shown, and this fact is captured by the
invariants of the state machine. Only the \texttt{report} method is called
more than once for each instance of the \texttt{ElectionAlgorithm} class.

The user interface is being designed in a flexible fashion so as to present
non-Dutch ballot papers to the voter. The original KOA system was designed for
use with a party-list system with a single national constituency. Its user
interface is being extended in line with the guidelines for the Irish voting
system.  The KOA system allows the voter to select a list of candidates. In the
Irish system each candidate is in a list of size one. The KOA system allows only
one selection by the voter. In the Irish system the voter makes multiple
selections in order of preference.

%=====================================================================
\section{Related Work}
\label{sec:related-work}

\subsection{A Security Analysis of SERVE}

The security analysis of the SERVE project~\cite{Jefferson04Serve} is one of 
the best known examinations of remote internet voting. It is very critical of 
current efforts and advises against any use of such methods given the current 
state of technology, due to its inherent vulnerabilities.

Two main arguments against internet voting can be distinguished in the report. 
Firstly, it is argued that the system allows for vote buying and selling. 
However, this holds for any voting system in which voters vote at home. 
Internet voting can only be fairly compared to postal ballots, not to voting at 
polling stations. If we want to introduce remote voting on a large scale, 
measures can be taken (technical, organisational, and legal) that make it 
unattractive to buy or sell votes.

A second argument against internet voting is that the technology is vulnerable 
to attacks. Unfortunately, despite claiming to have examined alternatives to 
the SERVE system, it ignores systems that have overcome some, but not all, of 
the problems mentioned. Although, the KOA system was not fully developed at the 
time of writing, the recommendations presented in 2002 by Dr. Rolf 
Oppliger\footnote{\emph{How to Address the Secure Platform Problem for Remote 
Internet Voting in Geneva} --- available from 
\url{http://www.ifi.unizh.ch/~oppliger/Docs/sis_2002.pdf}.} for the use of a 
remote internet voting system in 
Geneva\footnote{\url{http://www.geneve.ch/evoting/english/welcome.asp}.}, describe 
security mechanisms, such as code sheets, that the authors of the SERVE report 
do not mention.

KOA is a much more secure system than SERVE in that it uses code lists for data
integrity, transaction codes for verifiability and is not closed and proprietary.
 
% \subsection{Remote Internet Voting Experience in Geneva}
% Alan: I can't find any information on the actual voting process or
% implementation of the system in use in Geneva. There's a great report about
% recommendations of what they should use, but nothing on what they do use. The
% information that i've found is trivial and aimed at swiss voters or CNN!!

% Alan: The following comments can be addressed in a ``Related Work''
% section in which I'll give some info on the Geneva stuff I've read
% about.

% Reviewer: Overall, this is interesting, but I am frustrated by the
% lack of reflection in the paper.  I expected to see at least some
% learning from the Geneva internet voting experience, since Geneva
% has been doing internet voting longer than any other government on
% earth.  They made some interesting mistakes and they had some very
% good ideas, both of which are worth learning from..


\subsection{The RIES System}
The RIES system was developed for elections for public water management 
authorities in the Netherlands. It has two main features which create 
confidence in the limited possibilities of attacking the system. First of all, 
a reference table is published before the elections, including (anonymously) 
for each voter the hashes of all possible votes, linking those to the 
candidates. It is possible to compare the number of voters in this table with 
the number of registered voters. After the elections, a document with all 
received votes is published. This allows for two important verifications:

\begin{enumerate}
  \item A voter can verify his/her own vote, including the correspondence to 
  the chosen candidate.
  \item Anyone can do an independent calculation of the result of the 
  elections, based on this document and the reference table published before 
  the elections.
\end{enumerate}

If your vote has been registered incorrectly, or not at all, it can be 
detected. And if the result is incorrect given the received votes, this can 
also be detected. The main technique that achieves this is the clever 
use of hash functions. Whereas the hashes of all possible votes are public, it 
is impossible to deduce valid votes from them without the required voter key. 
Of course, the relation between voter and voter key should not be stored 
anywhere, as is the case for bank access codes. The system has worked
well in an actual election with 70,000 voters.

A disadvantage of the RIES system in comparison with the KOA system is that 
a voter needs to compute hash values in order to verify that a vote has been 
correctly recorded. This is far more complicated than simply checking a 
transaction code in the list of votes after the election.
%=====================================================================
\section{Future Work}
\label{sec:future-work}

Several pieces of future work have been identified and some of them are 
currently underway by researchers at UCD.

%~~~~~~~~~~~~~~~~~~~~~~~~~~~~~~~~~~~~~~~~~~~~~~~~~~~~~~~~~~~~~~~~~~~~~
\subsection{Development of a Mobile E-Voting Application}

The EU MOBIUS Project\footnote{The MOBIUS Project ---
\url{http://mobius.inria.fr/}.}, of which UCD and Nijmegen are both members, 
focuses on several topics including the specification and verification of 
security properties at several levels.

As part of this work, the security properties, including a functional
specification, for a MIDP-based remote voting application are in the process of
being defined.  An example of such a security property is: ``The
application must not have access to personal information (\eg phone book) on
the mobile phone''. 

Additionally, a MIDP-based remote voting applet has been developed at UCD. This 
application has been reviewed and will be refactored, including the security 
and functional requirements expressed in JML, for incorporation into KOA.

%~~~~~~~~~~~~~~~~~~~~~~~~~~~~~~~~~~~~~~~~~~~~~~~~~~~~~~~~~~~~~~~~~~~~~
% Alan: Joe - Could you do this?
% In Section 6.2, I would like to know in which points
% this project is more ambitious and which the difficulties are going
% to be.
% Joe: I have written a final subsection to address this comment.

\subsection{Full-blown Verification}
We intend to fully specify and verify critical subsystems of the KOA system as 
a case study for the new MOBIUS Integrated Verification Environment (IVE) that 
is being developed by UCD and others. This goal is much more ambitious than
simply performing extended static checking on various critical classes.

%~~~~~~~~~~~~~~~~~~~~~~~~~~~~~~~~~~~~~~~~~~~~~~~~~~~~~~~~~~~~~~~~~~~~~
\subsection{Just-in-Time Deployment with PCC}

One of the primary problems with electronic voting systems is that new software 
updates, at both operating system and application levels, are typically 
installed in the field without any certification~\cite{kitcat}. One technology 
that can help solve this deployment issue is Proof-Carrying Code 
(PCC)~\cite{Albert06,Necula97}, the primary underlying formal foundation and
technology used by the MOBIUS IVE.

Using a PCC technology foundation, new system and application patches could be 
just-in-time deployed to the thousands of voting machines used in an election 
with complete assurance.  Developing such a foundation is part of the MOBIUS 
project's mandate, so the KOA system may be used as a deployment case study in 
the coming years.
%~~~~~~~~~~~~~~~~~~~~~~~~~~~~~~~~~~~~~~~~~~~~~~~~~~~~~~~~~~~~~~~~~~~~~
\subsection{American Voting System}
The American voting system is the focus of an intense amount of discussion and 
work, given the ongoing fiasco in electronic voting we have witnessed in the 
U.S. over the past several years.

After integrating the \Votail Irish voting subsystem, we would be interested in 
collaborating to formally specify and verify a voting subsystem for use in 
American presidential and/or congressional elections using the same 
verification-centric methodology we have followed thus far.

%~~~~~~~~~~~~~~~~~~~~~~~~~~~~~~~~~~~~~~~~~~~~~~~~~~~~~~~~~~~~~~~~~~~~~
\subsection{Electronic Voting Systems}

An electoral-system independent, formally specified and verified
remote voting system can be used in an electronic voting system, as
the latter is just a trivial, non-remote version of the former.  It is
our intention to build and demonstrate such a system, incorporating a
new formally specified and verified voter-verifiable paper trail
subsystem.

%~~~~~~~~~~~~~~~~~~~~~~~~~~~~~~~~~~~~~~~~~~~~~~~~~~~~~~~~~~~~~~~~~~~~~
\subsection{Reflections Future Plans}

Many of these plans are ``just'' a matter of good software engineering
and thus can be accomplished by undergraduate and postgraduate
students as case studies, theses work, \etc  Others are \emph{much}
more difficult.  In particular, attempting verification in any form
and incorporating PCC techniques into the system are quite difficult,
time consuming, and even require new research to be conducted.  This
work will take several years to accomplish, and only if the number of
individuals and groups working on and with the system grows over time.

%=====================================================================
\section{Conclusion}
\label{sec:conclusion}
The availability of an American voting subsystem will make KOA the first 
general-purpose, formally specified and verified remote and local voting system 
available in the world, and furthermore it will be available under the GPL 
license.  Furthermore, the KOA system is being donated to the UK Grand 
Challenge Verified Code Repository as a major case study for the
application of formal methods to critical, large-scale software development.

It is unclear how to compare such a system to the current commercial
and Free/\-Libre/\-Open Source Software (FLOSS) voting systems being
proposed by others, given that none of them, to our knowledge, even
write formal specifications, let alone perform verification.  We hope
that this work will encourage other similar projects to seriously
consider the use of lightweight formal methods in such critical
systems development.

While integrating the \Votail subsystem into the KOA system, and prior 
to/during the new full FLOSS foundation release of KOA, a number of new pieces 
of English documentation and functional specification must be written. Given 
that remote voting is a key case study in verified computing, we hope that the 
availability of such documentation and specification will provide additional 
motivation for researchers and developers to seriously consider using the KOA
system as a foundation for Verified Verifiable Voting (VVV).

We propose that the KOA system should be used as an \emph{experimental
  platform} for research in electronic and internet voting; we are
\emph{not} saying that we have solved any of the major problems
inherent in voting with computers.  We encourage researchers
interested in electronic and internet voting to contact us and join
this effort.

%=====================================================================
\section{Acknowledgments}

This work is being supported by the European Project Mobius within the frame of 
IST 6th Framework, national grants from the Science Foundation Ireland and 
Enterprise Ireland and by the Irish Research Council for Science, Engineering 
and Technology.  This paper reflects only the authors' views and the Community 
is not liable for any use that may be made of the information contained therein.
%======================================================================
%% \nocite{ex1,ex2}
%\bibliographystyle{splncs}
\bibliographystyle{plain}
%\bibliography{abbrev,ads,category,complexity,hypertext,icsr,knowledge,languages,linguistics,meta,metrics,misc,modeling,modeltheory,reuse,rewriting,softeng,specification,ssr,technology,theory,web,upcoming,upcoming_conferences,conferences,workshops,verification,escjava,jml,nijmegen}
% \bibliography{icse04}
\bibliography{koa}
%======================================================================
% Fin

\end{document}

%%% Local Variables:
%%% mode: latex
%%% TeX-master: t
%%% End:
