%
% $Id: paper.tex $
%

\documentclass{llncs}
\usepackage{llncsdoc}
\usepackage{times}
\usepackage{ifpdf}

\ifpdf \usepackage[pdftex]{graphicx} \else
\usepackage{graphicx}
\fi

\usepackage{xspace}
\usepackage{tabularx}
\usepackage{epsfig}
\usepackage{amsmath}
\usepackage{amsfonts}
\usepackage{amssymb}
\usepackage{eucal}
\usepackage{float}

\ifpdf \usepackage[pdftex,bookmarks=false,a4paper=true, 
plainpages=false,naturalnames=true, colorlinks=true,pdfstartview=FitV, 
linkcolor=blue,citecolor=blue,urlcolor=blue]{hyperref} \else 
\usepackage[dvips,linkcolor=blue,citecolor=blue,urlcolor=blue]{hyperref} \fi


\newcommand{\todo}{\textbf{TODO:}}
\newcommand{\tablesize}{\footnotesize}
\newcommand{\eg}{e.g.,\xspace}
\newcommand{\ie}{i.e.,\xspace}
\newcommand{\etc}{etc.\xspace}
\newcommand{\myhref}[2]{\emph{#2}}
\newcommand{\NL}{the Netherlands\xspace}
\newcommand{\Votail}{Vot{\'a}il\xspace}

%---------------------------------------------------------------------
% New commands, macros, etc.
%---------------------------------------------------------------------

%% \input{kt}

%=====================================================================

\begin{document}

\date{}

\title{\Large \bf Formally Specified Security Properties for a Remote
Voting System}

\maketitle

\thispagestyle{empty}

%======================================================================
\subsection*{Abstract}
security requirements, formal analysis and evaluation of KOA/2 remote voting
system


%=====================================================================
\section{Introduction}

 
% in 2004.
%~~~~~~~~~~~~~~~~~~~~~~~~~~~~~~~~~~~~~~~~~~~~~~~~~~~~~~~~~~~~
\subsection{KOA: A (Remote, Trustworthy) Voting Architecture for Voting Research}

The KOA system, originally developed for and released by the Dutch government, 
can be used as both a standard
kiosk-based computer-based voting system as well as a remote telephone and
Internet-based voting system.  It has a general-purpose voting
system-independent core and a ``plugin'' model for supporting various
voting systems like the list-based system of Holland or Ireland's
Proportional Representation - Single Transferable Vote (PR-STV) based
system~\cite{KiniryEtAl06}.

%=====================================================================
\section{}

%~~~~~~~~~~~~~~~~~~~~~~~~~~~~~~~~~~~~~~~~~~~~~~~~~~~~~~~~~~~~~~~~~~~~~
\subsection{}

\paragraph{The Java Modeling Language.}

The Java Modeling Language (JML) is a formal behavioral specification
language for Java~\cite{LeavensBakerRuby99,LeavensBakerRuby-Prelim}.
Effectively, it is a small extension to the Java programming language
that uses annotations embedded in special comments to formally express
properties of a Java class or interface.

JML is used at two levels.  At a high level one uses JML to describe a
mathematical model-based specification (the ``Modeling'' of
J\textbf{M}L) of a software system system.  JML models are functional,
executable, formally specified and verified constructs like sets,
sequences, bags, maps, etc.  At a lower level JML is used to describe
a concrete software architecture with familiar constructs like
invariants, preconditions, postconditions, etc.

To connect these two levels, a specifier describes a functional or
relational refinement relating the high level model-based specification
and the low level, contract-based, description of the implementation.

 

 

%=====================================================================
\section{Conclusions}
\label{sec:conclusions}

 

%=====================================================================
\section{Acknowledgments}

This work is being supported by the European Project Mobius within the IST 6th
Framework. This paper reflects only the authors' views and the Community is 
not liable for any use that may be made of the information contained therein.

%=====================================================================
\section{Availability}

The KOA system's home page is found at
{\url{secure.ucd.ie/products/opensource/KOA}}

%======================================================================
%% \nocite{ex1,ex2}
% \bibliographystyle{latex8}
{\footnotesize \bibliographystyle{acm}
  \bibliography{abbrev,ads,category,complexity,hypertext,icsr,knowledge,languages,linguistics,meta,metrics,misc,modeling,modeltheory,reuse,rewriting,softeng,specification,ssr,technology,theory,web,upcoming,upcoming_conferences,conferences,workshops,verification,escjava,jml,nijmegen,paper}
}

%\theendnotes

%======================================================================
% Fin

\end{document}

%%% Local Variables:
%%% mode: latex
%%% TeX-master: t
%%% End:
