%
% $Id: paper.tex $
%

\documentclass[letterpaper,twocolumn,10pt]{article}
\usepackage{usenix,epsfig,endnotes,xspace,float}
\usepackage{ifpdf}

% \ifpdf \usepackage[pdftex]{graphicx} \else
% \usepackage{graphicx}
% \fi

% \usepackage{html}
\usepackage{url}
% \usepackage{xspace}
% \usepackage{doublespace}
% \usepackage{tabularx}
% \usepackage{epsfig}
% \usepackage{amsmath}
% \usepackage{amsfonts}
% \usepackage{amssymb}
% \usepackage{eucal}
% \ifpdf
% \usepackage[centredisplay]{diagrams}
% \else
% \usepackage[centredisplay,PostScript=dvips]{diagrams}
% \fi
% \usepackage{float}

\ifpdf
\usepackage[pdftex,bookmarks=false,a4paper=false, 
            plainpages=false,naturalnames=true, colorlinks=true,pdfstartview=FitV, 
            linkcolor=blue,citecolor=blue,urlcolor=blue,
            pdfauthor="Joe Kiniry"]{hyperref}
\else 
\usepackage[dvips,linkcolor=blue,citecolor=blue,urlcolor=blue]{hyperref} \fi

\newcommand{\todo}{\textbf{TODO:}}
\newcommand{\tablesize}{\footnotesize}
\newcommand{\eg}{e.g.,\xspace}
\newcommand{\ie}{i.e.,\xspace}
\newcommand{\etc}{etc.\xspace}
% \newcommand{\myhref}[2]{\ifpdf\href{#1}{#2}\else\htmladdnormallinkfoot{#2}{#1}\fi}
\newcommand{\myhref}[2]{\emph{#2}}
\newcommand{\NL}{the Netherlands\xspace}
\newcommand{\Votail}{Vot{\'a}il\xspace}

%---------------------------------------------------------------------
% New commands, macros, etc.
%---------------------------------------------------------------------

%% \input{kt}

%=====================================================================

\begin{document}

\date{}

\title{\Large \bf Verification-Centric Realization of Electronic Vote Counting}

\author{{\rm Joseph R.~Kiniry, Dermot Cochran, and Patrick E. Tierney}\\
School of Computer Science and Informatics\\
University College Dublin\\
Belfield, Dublin 4, Ireland}

\maketitle

\thispagestyle{empty}

%======================================================================
\subsection*{Abstract}

Activist computer scientists, including some of the authors of this
paper, have been working against the adoption by governments of commercial,
proprietary, insecure, poorly designed and implemented voting systems
the world-over.  And, while we mainly work to accomplish our goals by educating
citizens and communicating with the press, we also must propose solutions to the
problems of trustworthy e-voting.  If a computer-based voting system is to ever
be adopted, that system must be demonstrably of extremely high quality.  This
paper discusses a methodology and a set of tools we have used to implement a
vote counting plugin, for an experimental computer-based voting system using
applied formal methods.

%=====================================================================
\section{Introduction}

Regardless of the concerns raised by experts and activists, governments are
adopting computer-based technology for e-voting.  Researchers and
activists concerned about this situation fall in two overlapping
camps: researchers that are struggling against the adoption of poor
process, technology, and implementation in computer-based voting, and
researchers that are proposing new techniques, theories, technologies
to solve some of the computer-based voting challenges.  Our
work falls broadly into both of these camps, but this work in
particular focuses on the latter, as we propose a concrete software
platform and software engineering process for trustworthy e-voting
research.

Computer-based voting, from a mathematical and computer science
point-of-view, is full of interesting algorithmic
challenges---challenges that are sometimes radically different than
those that we understand in other domains.  For example, in
many voting systems the anonymity of the voter must be preserved, but
we must guarantee that the legitimate ballot of the voter is counted.

If a computer-based technology is used in trustworthy voting, then the
software system must be of extremely high quality.  There are a
variety of techniques for constructing a very high quality system,
e.g., model-based design, rigorous unit testing, and formal
verification, to name a few.  Therefore, usable, practical
best-practices in rigorous software development, especially as applied
to a real, concrete, trustworthy computer-based voting system, are
very relevant today.

The work described in this article is useful to researchers because it
provides a concrete foundation for future trustworthy voting
experimentation.  On the other hand, this research is useful to
governments and their experts because it provides concrete evidence
that modern verification-centric software development with formal
methods is achievable without significant cost or time.  
%
%Consequently,
%if a government spends a large amount of money on a commercial voting
%system, and that system is not of at least the same quality of this
%work (work performed by a handful of academics part time), then there
%is a \emph{serious} problem with the decision-making procedures of
%that government and the realization of its computer-based voting
%mandates.
%~~~~~~~~~~~~~~~~~~~~~~~~~~~~~~~~~~~~~~~~~~~~~~~~~~~~~~~~~~~~
% \subsection{e-Voting Activism and Research}
% 
% Over the past four years we have been wearing two contrary hats when
% it comes to voting.
% 
% Wearing our ``computer security consultant'' hat, we have been working
% with groups like the Free Software Foundation Europe, the Irish
% Citizens for Trustworthy Evoting, and the FREE e-democracy project in
% the U.K., advocating against the adoption of existing computer-based
% voting technologies.  This work includes cracking a Dutch remote
% voting pilot project (discussed below), performing security audits of
% voting systems, writing the aforementioned report on e-voting for the
% Dutch parliament, and implementing a formally verified tally system
% for the Dutch government.
% 
% Meanwhile, wearing our ``applied formal methods researcher'' hat, and
% working with students in our research group here in CSI at UCD (in
% particular, Alan Morkan, Dermot Cochran, Fintan Fairmichael, and
% Patrick Tierney) and our previous research group at the Security of
% Systems Group at Radboud University Nijmegen (with Bart Jacobs, Wolter
% Pieters, Martijn Oostdijk, and Engelbert Hubbers), we has been leading
% the development a new Open Source software platform for computer-based
% voting research.  This software platform is called KOA (``Kiezen op
% Afstand'', or ``remote voting'' in Dutch) and is based upon a software
% platform developed for, and released by (on recommendation of the
% aforementioned report), the Dutch government under the GPLv2 license
% in 2004.
%~~~~~~~~~~~~~~~~~~~~~~~~~~~~~~~~~~~~~~~~~~~~~~~~~~~~~~~~~~~~
\subsection{KOA: A (Remote, Trustworthy) Voting Architecture for Voting Research}

The KOA system, originally developed for and released by the Dutch government, 
can be used as both a standard
kiosk-based computer-based voting system as well as a remote telephone and
Internet-based voting system.  It has a general-purpose voting
system-independent core and a ``plugin'' model for supporting various
voting systems like the list-based system of Holland or Ireland's
Proportional Representation - Single Transferable Vote (PR-STV) based
system~\cite{KiniryEtAl06}.

As mentioned previously, this software system is only meant to provide
a very high-quality, extensible research platform for computer-based
voting --- it is not meant to be used in any governmental elections.  To
highlight this extensibility, as well as to provide evidence that a
verification-centric software engineering methodology results in a
software system of very high quality, the Irish voting system has
been formally specified and verified.

%~~~~~~~~~~~~~~~~~~~~~~~~~~~~~~~~~~~~~~~~~~~~~~~~~~~~~~~~~~~~
% \subsection{Aside: Criticisms of trustworthy computer-based
%   voting---SERVE and similar reports}
% 
% The SERVE report points out a number of general challenges in
% Internet-based voting, and a number of specific problems with the
% U.S. Department of Defense's Federal Voting Assistance Program in
% 2004~\cite{SERVE}.  The \emph{Beveiligingsanalyse van
%   Internetstembureau.nl} report, co-written by one of the authors
% (Kiniry) of this paper in 2003~\cite{BVI03}, contains comparable
% content and criticisms of a Dutch remote voting system.
% 
% Thus, we are well aware of the challenges inherent in trustworthy
% computer-based voting, and particularly in remote voting over a public
% network.  The point of this work is to provide a quality software
% architecture in which other researchers can begin to address the
% challenges identified in these, and other similar, reports.

%~~~~~~~~~~~~~~~~~~~~~~~~~~~~~~~~~~~~~~~~~~~~~~~~~~~~~~~~~~~~
%\subsection{Tools of the Trade}

% As trustworthy electronic voting system must be both secure and
% correct, ballots must be collected and counted accurately and
% precisely as specified by electoral law.  

%Researchers agree that
%neither manual inspection of source code, nor manual testing can find
%all of, or perhaps even most of, the errors in the source code or
%system design.  On the other hand, most practitioners believe that
%formal specification and verification is unrealistic, unattainable,
%and too expensive.  Unfortunately, this attitude is rarely based upon
%personal experience, especially using modern languages, tools, and
%techniques.

%Our research group focuses on \emph{applied} formal methods: the
%creation and evolution of new \emph{theories}, \emph{tools},
%\emph{processes}, and \emph{techniques} for developing \emph{real},
%\emph{concrete}, \emph{industry-scale} software systems.  The ``tools
%of the trade'' are thus more than just software tools.  Choosing the
%right concepts, ensuring a consistent design style, and following a
%(mathematically) rigorous but flexible process has as much (or more)
%to do with the quality of a software system as one's choice of
%programming language or operating system.

% The rest of the paper is organized as follows.  In section 2 we describe the 
% software modeling and verification  used.  In sections 3 and 4 we describe 
% KOA's architecture, and our verification-centric
% methodology for the analysis, design, specification, implementation,
% testing, and verification of trustworthy computer-based vote counting.
% As a case study, we have developed a plugin, called \emph{\Votail},
% for the KOA system that specializes it in the Irish context.  Finally,
% the paper concludes by reflecting on the state of e-voting research in
% computer science and mathematics.

%=====================================================================
\section{Verification-Centric\\Software Development}

The following summary is organized from a process-centric
point-of-view.  Each stage in the (mostly linear) development process
is described.  After describing the various aspects of
our verification-centric process, we discuss the Irish voting case
study that uses this methodology.

%~~~~~~~~~~~~~~~~~~~~~~~~~~~~~~~~~~~~~~~~~~~~~~~~~~~~~~~~~~~~~~~~~~~~~
\subsection{Analysis}

The first stage of our process is to perform domain analysis.  The
distinct core concepts of the domain that we are trying to model are
identified, usually while interacting as a group, and sometimes with
domain experts present.  Considering input from several parties
elucidates different perspectives on the system and helps the group
find agreement on which concepts are core, how to define these
concepts, and how these concepts relate to one another.

% replace all of this with one paragraph about EBON with appropriate
% references
% \paragraph{Extended BON.}
We use a language called ``Extended Business Object Notation'' (or EBON for 
short) to describe our system at this early stage ~\cite{WaldenNerson95}.  
EBON is a system specification language akin to UML that differs in several key
respects~\cite{Kiniry02-PhDThesis,Kiniry04-R0420}.
Contrary to UML, EBON has a relatively small number of charts and
diagrams, a simple semantics, a graphical and human-readable textual
syntax, and it is easy to maintain consistency between an EBON
specification and a concrete realization of that specification.

% EBON is based upon BON, which was created by Walden and
% Nerson~\cite{WaldenNerson95} in the mid-1990s.  EBON was created by
% Kiniry by adding \emph{semantic properties} to
% BON~\cite{Kiniry04-R0420,Kiniry02-PhDThesis}.  Semantic properties are
% formally specified domain-specific constructs that are added to a
% general-purpose like BON.  For example, semantic properties are used
% to describe code ownership, software licenses, feature and bug
% tracking, code reviews, concurrency requirements, etc.
% The key constructs in EBON that we use in this work are class
% dictionaries and several kinds of informal charts, (system, cluster,
% class, creation, and event charts), which describe concepts using
% structured natural language.
% \paragraph{Class Dictionaries.}
% In BON, domain constructs are known as \emph{classes}, not in a
% concrete object-oriented sense, but rather ``classifiers'' in a
% generic sense.  In our process, a name and a single sentence
% definition is agreed upon for each class \emph{completely
%   independently of design and implementation considerations}.  When
% writing a definition one uses only classes that are defined in the
% model under analysis or classes from models on which the domain
% depends.
% \paragraph{Informal Charts.}
% Due to space constraints, we will only describe class charts in detail
% here.  In short, a system chart enumerates a system's purpose and
% major subsystems (clusters, in the BON vernacular), a cluster chart
% explains the purpose of a subsystem and, in turn, what subsystems and
% classes it contains, and event charts identify important external and
% internal events (state transitions) in a subsystem.
% Each class is described, again in English, using a class chart.  A
% class chart contains four sections: a \emph{description},
% \emph{queries}, \emph{commands}, and \emph{constraints}.  Queries and
% commands are, collectively, known as features.
% The description of a class is an expanded version of the
% aforementioned class's definition.  Within our process, the first
% sentence of the of the description is the definition, thus the formal
% refinement between the description and the definition is simply a
% prefix.
% Queries are operations that do not change the class's state, whereas
% commands are operations that do change a class's state.  Constraints
% are restrictions on a class's state and behavior.
% \begin{figure}
% {\footnotesize
% \begin{verbatim}
% class BALLOT
% description
%   "A ballot paper in an Irish election."
% query
%   "What is the location of this ballot \
%    in the current count?"
%   "What is the count number for the last \
%    transfer of this ballot?"
%   "Is this ballot non-transferable?"
%   "What is the first preference of this \
%    ballot?"
%   "To which candidate is this ballot \
%    assigned?"
%   "What is the next preference candidate?"
%   "What is this ballot's ID number?"
%   "Is this ballot paper assigned to a \
%    given candidate?"
%   "How many preferences remain on this \
%    ballot?"
%   "Is this ballot on top of a given \
%    different ballot?"
% command
%   "Set this ballot's candidate list."
%   "Transfer this ballot to the next \
%    preference candidate."
% constraint
%   "No two ballots have the same ballot ID."
%   "A ballot must be assigned to a \
%    candidate that is on its candidate \
%    list."
%   "No two preferences given to \
%    candidates on this ballot may be \
%    identical."
%   "Preferences expressed on this ballot \
%    must start with 1, monotonically \
%    increase, and grow no larger than the \
%    number of candidates available \
%    on this ballot."
%   "If this ballot is transferred, the \
%    count number at which this ballot \
%    was last transferred must be positive."
% end
% \end{verbatim}
% }
% %\normalsize
% \caption{An EBON class chart.}
% \label{fig:EBON_class_chart_example}
% \end{figure}
% An example class chart for the class \emph{BALLOT} is in
% Figure~\ref{fig:EBON_class_chart_example}.  Notice how queries are all
% written as questions, commands as imperative sentences, and
% constraints only use features introduced by queries and constraints.
% We also must describe how classes relate to one another with respect
% to how classes come into being.  A creation chart summarizes these
% relationships.  With respect to a verification-centric approach, our
% goal in defining such relations is to minimize the exposure of a
% class's creation and to identify ownership relations between classes.
% The former goal reduces the amount of reasoning we must do about
% object allocation and aliasing and the latter supports the use of
% ownership type systems to support modular reasoning.
%~~~~~~~~~~~~~~~~~~~~~~~~~~~~~~~~~~~~~~~~~~~~~~~~~~~~~~~~~~~~~~~~~~~~~
\subsection{Design}

After high-level analysis is complete, we move on to defining
medium-level contracts using EBON.  During this design stage, we
identify inheritance relationships between classes, client-supplier
relationships between classes, specify the formal semantics of
features and constraints, and identify interesting scenarios in which
classes interact.

Because we specify classes with invariants and assertions like pre-
and postconditions, we are following a strict design-by-contract
approach to program construction.  This methodology is further
emphasized below when we discuss our implementation strategy.

% Each of these constructs helps in a different way with respect to a
% verification-centric methodology.  For example, identifying
% inheritance relationships helps identify behavioral inheritance (which
% is useful for feature reuse and necessary for modular reasoning)
% versus code inheritance (which is useful for code reuse, and,
% potentially, for reducing verification burden).  Likewise, elucidating
% client-supplier relationships helps identify how interdependent an
% architecture is (and thus, how fragile verification results are) and
% better understand how weak or strong feature preconditions must be.

Scenarios, which are similar to UML's use cases, identify interesting
interactions between classes.  Scenarios are used to concretely
identify and create module- and subsystem-level unit tests during
implementation and testing.

%~~~~~~~~~~~~~~~~~~~~~~~~~~~~~~~~~~~~~~~~~~~~~~~~~~~~~~~~~~~~~~~~~~~~~
\subsection{Specification}

Next, our medium-level specifications are made concrete in a popular
specification language called the Java Modeling Language (JML).

\paragraph{The Java Modeling Language.}

The Java Modeling Language (JML) is a formal behavioral specification
language for Java~\cite{LeavensBakerRuby99,LeavensBakerRuby-Prelim}.
Effectively, it is a small extension to the Java programming language
that uses annotations embedded in special comments to formally express
properties of a Java class or interface.

JML is used at two levels.  At a high level one uses JML to describe a
mathematical model-based specification (the ``Modeling'' of
J\textbf{M}L) of a software system system.  JML models are functional,
executable, formally specified and verified constructs like sets,
sequences, bags, maps, etc.  At a lower level JML is used to describe
a concrete software architecture with familiar constructs like
invariants, preconditions, postconditions, etc.

To connect these two levels, a specifier describes a functional or
relational refinement relating the high level model-based specification
and the low level, contract-based, description of the implementation.

We use JML not just because of our local expertise and involvement in
the JML community, but because there is a large range of powerful
tools that understand JML, some of which we discuss
below~\cite{BurdyEtal05-STTT,Leavens-etal03b}.

For example, we compile JML specification into runtime checks, we
generate (tens of thousands of) unit tests from JML specifications,
and we formally verify that Java method implementations fulfill their
contracts expressed in JML.

\paragraph{Refinement between EBON and JML.}

At the present time we initially refine from EBON specifications to
JML by hand, translating EBON classes into concrete Java primitive
types, pre-provided JML model classes, new JML models, and concrete
Java modules (classes and interfaces).  We maintain the connection
between the specification layers by using a combination of simple
scripts and a development process that emphasizes manual inspection
and specification conformance checking.

During this refinement phase concrete choices of data representation
are made by identifying which EBON classes should be full-fledged Java
types, and how these Java types should relate to each other. E.g, if two
classes inherit from each other in EBON, then they must inherit from
each other in the JML/Java architecture.  

%To refine EBON classes to Java types, each EBON feature is mapped to a
%single method or, in the case of an EBON query with no associated
%constraint, a read-only field.  In particular, each query is mapped to
%a \emph{pure} method, each command is mapped to a non-pure method, and
%each constraint is mapped to one or more class or object invariants.
%When performing this refinement, each informal class or feature
%description is literally cut-and-pasted into the Javadoc documentation
%of the corresponding Java construct.  This means that the results of
%our analysis not only helps us understand the system at a high-level,
%but supports documentation reuse.  Consequently, when Javadocs are
%updated in later stages of development, high-level charts are
%regenerated automatically with the aforementioned scripts.

%~~~~~~~~~~~~~~~~~~~~~~~~~~~~~~~~~~~~~~~~~~~~~~~~~~~~~~~~~~~~~~~~~~~~~
\subsection{Implementation}

After a model-based specification is written in JML and medium-level
EBON contracts are refined into JML contracts on a Java modules, the
implementation of the system commences.  Because we use JML, our
specifications are formal and executable, and our implementation is
testable against, and verifiable with respect to, the JML
specification using tools such as the JML tool suite and
ESC/Java2~\cite{ChalinEtal06,KiniryCok04} (discussed below).  This
combination of tools and techniques provides a high level of
confidence that the software system implements the requirements
correctly.

Our implementation strategy is, generally, to implement the
constructors, factories, and initializers first, as these features let
one focus on invariants and the initial state of objects.  Next, we
focus on the easiest/smallest ``leaf'' methods---methods like basic
getters and setters, overridden base methods from
\texttt{java.lang.Object}, etc.  We then move up the chain of methods,
aiming to tackle less complex methods before more complex ones,
implementing methods deeper in the expected call stack of the
application.

It is important to note that we do not need to implement the whole
system, or even a whole class, before checking the correctness of
method implementations.  Firstly, we unit test methods early and
independently as soon as we have constructors and factories
implemented.  And secondly, extended static checking is modular, as
each method body is checked independently of any other method body.
Thus, unit tests and ESC/Java2 are run very frequently, typically
after only a few new lines of code are written.

%~~~~~~~~~~~~~~~~~~~~~~~~~~~~~~~~~~~~~~~~~~~~~~~~~~~~~~~~~~~~~~~~~~~~~
\subsection{Testing}

Testing includes manual development of automated tests at the method,
class, component, and subsystem levels, and automatic generation of
tests at the method and constructor level.

\paragraph{JML-JUnit.}

The JML-JUnit tool is used to generate test cases from the JML
specification~\cite{Cheon-Leavens02}.  This testing methodology uses
method specifications as test oracles and focuses on unit testing
methods and constructors independently.  Test data generators are
written for all types in the system by manually identifying
``interesting'' data values in the whole system-under-test.  

%For a method with a signature of $m(t_0, \ldots, t_n)$ and a set of
%``interesting'' data values for each formal parameter type, the
%JML-JUnit tool automatically generates sufficient tests to execute $m$
%with every possible combination of values appropriate for the types of
%the parameters.  Thus, if there are only two interesting values for
%each type in the signature of $m$, then up to $2^n$ tests are
%generated.  Obviously, this test generation methodology results in
%extensive tests that need not be written or maintained by hand and
%leads to very good test coverage.

%Tests are also written manually for specific scenarios which are
%interesting, e.g., where three lowest continuing candidates have equal
%votes.  Many of these tests are motivated by events and scenarios
%identified during analysis and design.  In fact, we aim to have at
%least one class- or component-level test for each and every event and
%scenario.

%Additionally, existing subsystem and system-level test data is used
%when available.  For example, as discussed below in our case study,
%the ballot data that had been used by the Irish CEV to test the
%current, now discredited, electronic voting system is being used in
%testing our verification-centric implementation.

%~~~~~~~~~~~~~~~~~~~~~~~~~~~~~~~~~~~~~~~~~~~~~~~~~~~~~~~~~~~~~~~~~~~~~
\subsection{Verification}

In addition to the testing, the implementation is verified with
respect to the JML specifications.  Whereas testing covers a finite
number of data values, verification is used to prove check the logical
correctness of the implementation.

\paragraph{The Extended Static Checking tool for Java.}

ESC/Java2 is a modular static verification tool that checks for common
runtime errors and verifies that a Java method implementation conforms
to its JML specification.  ESC/Java2 translates program source and its
specifications into verification conditions that are passed to one of
several automated and interactive theorem prover, which in turn either
verify that no problems are found or generate a counterexample
indicating a potential bug.  The tool and its built-in provers operate
automatically with reasonable performance---most methods are checked
in less than a handful of seconds.

\vspace{2pt}

Now that we have reviewed the our verification-centric process and
methodology, a case study focusing on the Irish voting system is
discussed.

%=====================================================================
\section{A Case Study: The Irish Voting System} 

The Irish voting system represents an interesting case study for
applied formal methods because Ireland's PR-STV voting system is
non-trivial and, given we are working in an Irish university, it has
local relevance.

%~~~~~~~~~~~~~~~~~~~~~~~~~~~~~~~~~~~~~~~~~~~~~~~~~~~~~~~~~~~~~~~~~~~~~
\subsection{Informal Analysis and Specification}

A voting system consists of several interrelated concepts.  There is
either one or more positions to be filled by election, or an outcome to be 
decided by referendum.  This case study is concerned with the voting system
for Irish general elections.  

In Irish elections, there are a number of seats to be filled in each
constituency.  There are almost always more candidates than seats, and
there are a large number of registered voters.  Each voter casts a
ballot using the PR-STV system, and those ballot papers are counted
using an process (an algorithm) that is defined by Irish law.  The
voting algorithm is iterative, requiring that a decision is made at
each iteration either to deem a candidate elected, or to eliminate a
candidate with lowest votes received.

These concepts are represented by EBON classes or Java types in the
specification: \emph{Ballot}, \emph{Ballot Box}, \emph{Candidate},
\emph{Decision}, \emph{Election Algorithm}, \emph{Election Details},
and \emph{Election Results}.

%~~~~~~~~~~~~~~~~~~~~~~~~~~~~~~~~~~~~~~~~~~~~~~~~~~~~~~~~~~~~~~~~~~~~~
\subsection{Formal Specification}

\Votail is the Irish word for Voting.  The \Votail specification is a
JML specification for the Irish vote counting system~\cite{Cochran06}.
This formal specification is derived from the complete functional
specification for the D{\'a}il election count
algorithm~\cite{CEV00,CEV02}.

Thirty nine formal assertions are identified in the Commentary on
Count Rules published by the Irish Department of Environment and Local
Government.  Each assertion expressed in JML is identified by a
Javadoc comment.  In addition, a state machine is specified so as to
link all of the assertions together.  Java classes are specified
for the vote counting algorithm, e.g., to represent the ballot papers
and to represent the candidates.  Concrete examples of how the
methodology was applied will clarify this work.

%~~~~~~~~~~~~~~~~~~~~~~~~~~~~~~~~~~~~~~~~~~~~~~~~~~~~~~~~~~~~~~~~~~~~~
\subsection{Detailed Example}

We summarize a detailed example of specification of a Java
method in \Votail.  Each method is specified with English by using
both an extended version of Javadoc, used to encode the EBON informal
specifications, and JML to express detailed formal specifications.
JML specification encode not only functional behavior, but also safety
and progress properties of the voting system.

% %---------------------------------------------------------------------
% \paragraph{Example of a simple requirement.}  

% \begin{figure}
% {\footnotesize
% \begin{verbatim}
% /**
%  * Minimum number of votes needed to 
%  * guarantee an election.
%  */
%  //@ public model long quota;
%  //@ public invariant 0 <= quota;
%  //@ public invariant quota <= totalVotes;
%  /**
%   * @see requirement 3, section 3, 
%   *      item 3, page 13
%   */
%  /*@ public invariant (PRECOUNT < state) ==> 
%    @  quota == 1 + (totalVotes / (seats + 1));
%    @*/
% \end{verbatim}
% }%\normalsize
% \caption{JML invariant for calculating a quota.}
% \label{fig:JMLSimpleExample}
% \end{figure}

% First we will start with a simple requirement in \Votail.  The
% specification for calculating a quota is shown in
% Figure~\ref{fig:JMLSimpleExample}, encoded in EBON and JML.

% Note how the specification summarizes the EBON constraint in the
% Javadoc comment, the EBON class is encoded in a JML model field with
% appropriate formal invariants, and the requirement is traced back to
% the original legal documentation with a cross-reference.  Each and
% every requirement is expressed in a similar manner.

%---------------------------------------------------------------------
%\paragraph{Example of a moderately complicated requirement.}  
\paragraph{Example of a formal requirement.}  
 
For an example of a formal requirement, we look to Section
 5, item 2 on page 18 of the \emph{Count Requirements and Commentary
   on Count Rules}~\cite{CEV00} which states:
 \begin{quote}
   The distribution of the only or the largest available surplus is mandatory if
   it - or, where there is more than one surplus, the sum of the surpluses - 
   could possibly do any of the following:-
   
   Elect a continuing candidate. This condition is satisfied if the (sum of the)
   surplus(es) and the votes of the highest continuing candidate equals or
   exceeds the quota;

   Save the lowest candidate from exclusion. This condition is satisfied if the
   (sum of the) surplus(es), together with the sum of the votes of the lowest
   continuing candidate, is equal to or greater than the number of votes
   credited to the second lowest continuing candidate; \ldots
 \end{quote}
 
 The requirement is translated into EBON as
 follows:
 \begin{quote}
   If the number of continuing candidates is equal
   to the number of seats remaining unfilled, or the
   number of continuing candidates exceeds by
   one the number of unfilled seats or there is one
   unfilled seat, then do not distribute any surplus
   unless it could allow one or more candidates
   with at least one vote to save their deposits.
 \end{quote}
 
 \begin{figure*}[t]
 {\footnotesize
 \begin{verbatim}
   /** 
    * Distribute the surplus votes.
    * 
    * @param candidateWithSurplus the candidate whose surplus is to be distributed.
    * @design The highest surplus must be distributed if the total surplus could 
    *         save the deposit of a candidate or change the relative position of
    *         the two lowest continuing candidates, or would be enough to elect the
    *         highest continuing candidate.
    * @see requirements 14-18, section 5, item 2, page 18
    * @see requirement 8, section 4, item 2, page 15    */
   /*@ requires getSurplus(candidateWithSurplus) > 0;
     @ requires state == COUNTING;
     @ requires numberOfContinuingCandidates > remainingSeats;
     @ requires (numberOfContinuingCandidates > remainingSeats + 1) ||
     @          (sumOfSurpluses + lowestContinuingVote > nextHighestVote) ||
     @          (numberOfEqualLowestContinuing > 1);
     @ requires remainingSeats > 0;
     @ requires (remainingSeats > 1) ||
     @          ((highestContinuingVote < 
     @            sumOfOtherContinuingVotes + sumOfSurpluses) &&
     @          (numberOfEqualHighestContinuing == 1));
     @ requires getSurplus (candidateWithSurplus) == highestSurplus;
     @ requires (sumOfSurpluses + highestContinuingVote >= quota) ||
     @   (sumOfSurpluses + lowestContinuingVote > nextHighestVote) ||
     @   (numberOfEqualLowestContinuing > 1) ||
     @   ((sumOfSurpluses + lowestContinuingVote >= depositSavingThreshold) &&
     @      (lowestContinuingVote < depositSavingThreshold));
     @ ensures getSurplus (candidateWithSurplus) == 0;
     @*/
   /** @see requirement 9, section 4, item 3, page 16 */
   /*@ ensures countNumber == \old (countNumber) + 1;
     @ ensures (state == COUNTING) || (state == FINISHED);
     @*/
   /** @see requirement 2, section 3, item 3, page 12 */
   /*@
     @ ensures totalVotes == nonTransferableVotes + 
     @   (\sum int i; 0 <= i && i < totalCandidates; 
     @   candidateList[i].getTotalVote());
     @*/
   protected void distributeSurplus(/*@ non_null @*/ ie.koa.Candidate candidateWithSurplus);
 \end{verbatim}
 }
 \caption{JML Specification of the \texttt{distributeSurplus} method.}
 \label{fig:JMLModerateExample}
 \end{figure*}                           
 
 Finally, this requirement is formally specified in the
 architecture as JML pre/postconditions for the \texttt{distributeSurplus}
 method shown in Figure~\ref{fig:JMLModerateExample}.
 
 Note again how each high-level requirement or constraint is specified
 both in formal natural language (akin to ``legal English") and in one or more
 formal assertions.
 
%---------------------------------------------------------------------
% \paragraph{Example of a very complicated requirement.}  
% 
% It is worth noting that the semi-formal documents from which the
% requirements were extracted contain both declarative and imperative
% statements.  In contrast, JML is a declarative language, whereas Java
% is an imperative language.  The procedure for rounding of fractional
% vote transfers in the Irish STV system, for example, is imperative in
% nature and not easy to specify in a declarative way.  This raises the
% difficulty of writing a specification directly from a non-trivial
% functional requirement.  An alternative approach is to restate the
% requirement in a declarative way, perhaps using a domain specific
% language.
% 
% The procedure for rounding of fractional votes has the following two
% invariants:
% \begin{enumerate}
% \item the total number of transferable votes must not be altered by
%   the rounding of votes, and
% \item for each set of fractions to be rounded, there is a threshold
%   such that fractions greater than that threshold are rounded up, and
%   fractions below that threshold are rounded down.
% \end{enumerate}
% 
% \begin{figure}
% \footnotesize
% \begin{verbatim}
%   /**
%    * Determine actual number of votes to 
%    * transfer to this candidate, excluding 
%    * rounding up of fractional transfers
%    * 
%    * @see requirement 21, 
%    *      section 7, item 3.1, page 24
%    * @see requirement 22, 
%    *      section 7, item 3.2, page 25
%    * 
%    * @design The votes in a surplus are 
%    * transferred in proportion to the number 
%    * of transfers available throughout the 
%    * candidates ballot stack. The calculations 
%    * are made using integer values because 
%    * there is no concept of fractional votes 
%    * or fractional transfer of votes, in the 
%    * existing manual counting system.  If 
%    * not all transferable votes are 
%    * accounted for the highest remainders 
%    * for each continuing candidate need 
%    * to be examined.
%    * 
%    * @param fromCandidate 
%    *        the candidate from which to 
%    *        count the transfers.
%    * @param toCandidate 
%    *        the continuing candidate 
%    *        eligible to receive 
%    *        votes.
%    * @return the number of votes to 
%    *         be transferred, 
%    *         excluding fractional 
%    *         transfers.
%    */
%    \end{verbatim}
% \caption{Java Method Comment of the \texttt{getActualTransfers} method.}
% \label{fig:JavaMethodCommentExample}
% \end{figure}
% \normalsize
% 
% \begin{figure}
% \footnotesize
% \begin{verbatim}
%   /*@ requires (state == COUNTING);
%     @ requires (fromCandidate.getStatus() 
%     @          == Candidate.ELECTED) ||
%     @          (fromCandidate.getStatus() 
%     @          == Candidate.ELIMINATED);
%     @ requires toCandidate.getStatus() 
%     @          == Candidate.CONTINUING;
%     @ ensures ((fromCandidate.getStatus() 
%     @         == Candidate.ELECTED) && 
%     @         (getSurplus(fromCandidate) < 
%     @         getTotalTransferableVotes(
%     @         fromCandidate)))
%     @   ==>   (\result == 
%     @         (getSurplus (fromCandidate) * 
%     @         getPotentialTransfers (
%     @         fromCandidate, 
%     @         toCandidate.getCandidateID()) 
%     @         /
%     @         getTotalTransferableVotes (
%     @         fromCandidate)));
%     @ ensures ((fromCandidate.getStatus() 
%     @         == Candidate.ELIMINATED) ||
%     @         (getTotalTransferableVotes(
%     @         fromCandidate) 
%     @         <= getSurplus 
%     @         (fromCandidate)))
%     @   ==>   (\result == 
%     @         (\num_of int j; 
%     @         0 <= j && j < totalVotes;
%     @         ballotsToCount[j]
%     @         .isAssignedTo(
%     @         fromCandidate
%     @         .getCandidateID()) 
%     @         &&
%     @         getNextContinuingPreference(
%     @         ballotsToCount[j]) == 
%     @         toCandidate.
%     @         getCandidateID()));
%     @*/
%   protected /*@ pure @*/  
%       int getActualTransfers (
%           /*@ non_null @*/ Candidate fromCandidate, 
%           /*@ non_null @*/ Candidate toCandidate);
% \end{verbatim}
% \caption{JML Specification of the \texttt{getActualTransfers} method.}
% \label{fig:JMLExampleGetActualTransfers}
% \end{figure}
% 
% \begin{figure}
% \footnotesize
% \begin{verbatim}
%   /**
%    * Determine the rounded value of a 
%    * fractional transfer.
%    * 
%    * @design This depends on the shortfall 
%    * and the relative size of the other 
%    * fractional transfers.
%    * 
%    * @see requirements 23-25, 
%    *      section 7, item 3.2, page 25
%    *
%    * @param fromCandidate
%    *        Elected candidate from 
%    *        which to distribute surplus
%    *        
%    * @param toCandidate
%    *        Continuing candidate 
%    *        potentially eligible to 
%    *        receive transfers
%    *        
%    * @return 
%    *     <code>1</code> if the fractional 
%    *                    vote is to 
%    *                    be rounded up
%    *     <code>0</code> if the fractional 
%    *                    vote is to 
%    *                    be rounded down
%    */
%   /*@ requires state == COUNTING;
%     @ requires isElected (fromCandidate); 
%     @ requires toCandidate.getStatus() 
%     @     == ie.koa.Candidate.CONTINUING;
%     @ requires getSurplus(fromCandidate) 
%     @     < getTotalTransferableVotes
%     @     (fromCandidate);
%     @ ensures (
%     @ getCandidateOrderByHighestRemainder 
%     @    (fromCandidate, toCandidate) <=
%     @    getTransferShortfall 
%     @    (fromCandidate))
%     @   ==> \result == 1;
%     @ ensures (
%     @ getCandidateOrderByHighestRemainder 
%     @    (fromCandidate, toCandidate) >
%     @    getTransferShortfall 
%     @    (fromCandidate))
%     @   ==> \result == 0;
%     @*/
%   protected /*@ pure @*/ long 
%     getRoundedFractionalVote (
%     /*@ non_null @*/ 
%     Candidate fromCandidate, 
%     /*@ non_null @*/ 
%     Candidate toCandidate);
% 
% \end{verbatim}
% \caption{JML Specification of the \texttt{getRoundedFractionalVote} method.}
% \label{fig:JMLComplexExample}
% \end{figure}
% \normalsize
% 
% The Java method comment encoding these requirements is shown in
% Figure-\ref{fig:JavaMethodCommentExample}.  The JML postconditions for
% the rounding of fractional vote transfers are shown in
% Figure-\ref{fig:JMLExampleGetActualTransfers} and
% Figure-\ref{fig:JMLComplexExample}.

The \Votail specification is type checked and checked for consistency
using ESC/Java2.

%=====================================================================
\section{Testing the Specification by \\ Partial Implementation}

As is already obvious, we follow a strict design-by-contract approach
to system design and development~\cite{Meyer92b}.  Thus, we have a
fairly strict set of programming style standards and a process that
restricts a programmer's ability to make changes to the system that
are not tested, verified, and properly specified.

%We should note that the implementation of this trustworthy e-voting
%system is being written by an undergraduate student (Patrick Tierney)
%with only a few months of experience with JML and these tools and
%techniques.  The comments on the implementation below are from the
%point-of-view of a programmer with less experience that a typical
%programmer in industry that might be working on this kind of system.

%~~~~~~~~~~~~~~~~~~~~~~~~~~~~~~~~~~~~~~~~~~~~~~~~~~~~~~~~~~~~~~~~~~~~~
\subsection{Implementation Strategy}

%During the implementation of this system was written to stay as close
%to the specifications as possible.  This is not always possible as
%certain alterations must be made so ensure the system functions
%properly.  Changes made must have justification and must be documented
%accordingly.  Each change to the specification is explained so the
%change is understood by other developers.  As this project adheres to
%the KindSoftware Coding Standard, code reviews are also
%mandatory~\cite{KS-CodeStandard05}.

\paragraph{Incremental extended static checking.}

%At the start of the implementation, various small methods in both the
%\texttt{Ballot} and \texttt{Candidate} classes, such as
%\texttt{declareElected()} and \texttt{getFirstPreference()}, were
%written first to provide a simple start to learning the ESC/Java2 and
%JML tools.  A good understanding of the tools is vital in order to
%implement each method correctly, especially in the cases of the more
%difficult methods.

ESC/Java2 is used to ensure that the specifications for each class,
method, and field were strictly adhered to.  Recall that ESC/Java2 is
reasoning about all code paths simultaneously, modularly, and
statically.

%ESC/Java2 is excellent at discovering sections of the code that
%violated the specification.  As a programmer, it increases your
%knowledge of Java and improves your programming immensely.  ESC/Java2
%does have some faults, as it does not cater for some operators in JML
%used in this specification (e.g., the generalized quantifier
%\texttt{\\sum} is not supported).  Also, while ESC/Java2 does
%type check an ownership type system (the universes type system), it
%does not yet generate verification conditions that leverage such
%specifications.  Thus, we sometimes struggle with reasoning about
%aliasing, particularly with respect to class invariants.

%~~~~~~~~~~~~~~~~~~~~~~~~~~~~~~~~~~~~~~~~~~~~~~~~~~~~~~~~~~~~~~~~~~~~~
\subsection{Testing Strategy}

As discussed earlier, JML-JUnit automatically generates an extensive
unit test suite.

Many of the ``interesting'' data values necessary for testing are
identified quickly due to the fact that arrays are used frequently in
the system.  Therefore, values like $0$, $1$ and the length of each
array were very useful.  Other values that were used were already
specified in the contracts, such as maximum possible rounds of
counting.  Some other values identified are the maximum values of
Java primitive types like integers and longs.

For each value added to the test data, the number of tests increases
by between 100 to 1000 tests depending on the complexity and size
method being tested.  For example, by adding one extra data value
while testing the method \texttt{Ballot.load()}, the number of tests
increased by over 700 tests.

JML-JUnit has been very useful and provided extensive information
about how the system works and performs.  The fact that the JML
specifications are also tested adds precision to the testing and helps
us identify problems with both the implementation and the
specifications.  By continually adding particular data values, we aim
to have 100\% test coverage of the system.   

%=====================================================================
\subsection{Observations}

The main errors in specifications that have been discovered are
concentrated on arrays that had no upper bounds specified for them.
Also, we discover that some specifications are too weak to verify the 
class invariants. To solve these problems we correct and strengthen
the specifications, and use helper methods (methods that need not maintain
class invariants).

One of the invariants in the system specifies that random numbers must be 
unique e.g. to simulate random shuffling of ballot papers.  This invariant which
is part of both the \texttt{Candidate} and \texttt{Ballot}
classes, is the source of many errors reported by ESC/Java2.

%Most of the specification faults we have discovered using this
%verification-centric methodology, other than the array bounds errors,
%would have been very difficult to foresee, and therefore probably
%could not of been avoided.

{\footnotesize
\begin{verbatim}
//@ public constraint
//@   \old(lastSetAddedCountNumber) <
//@   lastSetAddedCountNumber;
\end{verbatim}
}
%\normalsize

For example, there is a bug in this constraint assertion, as it must
hold for all methods (pure or otherwise), and the JML reference manual
indicates that constraints generally are reflexive and transitive, and
`\texttt{<}' is not reflexive.  To correct this constraint, the
`\texttt{<}' operator is replaced by `\texttt{<=}'.

As another example, the \texttt{addVote()} and \texttt{removeVote()}
methods need to have the following precondition:
{\footnotesize
\begin{verbatim}
//@ requires 0 <= numberOfVotes;
\end{verbatim}
}
%\normalsize
This precondition is added to set a lower bound on the variable
\texttt{numberOfVotes} to ensure that the invariant
{\footnotesize
\begin{verbatim}
//@ public invariant
//@   (\forall int i; 0 < i && i < MAXCOUNT;
//@    0 <= votesAdded[i]);  
\end{verbatim}
}
%\normalsize
is always valid.

%=====================================================================
\section{Conclusions}
\label{sec:conclusions}

Because computer-based voting (a) is full of interesting algorithmic
and security challenges, (b) is an application area ripe for the use
of formal methods, and (c) is having a dramatic broad impact on
society today, we have chosen to work with governments and
independently on a computer-based voting system.  We believe that
governments should use our work as a benchmark against which to
compare other trustworthy voting system.  We offer researchers a
well-documented verification-centric process, a set of techniques and
tools for rigorously developing quality software, and, as a case study
and foundation for future research, an open source trustworthy voting
system developed with these tools and techniques.

While integrating the \Votail subsystem into the KOA system, and prior
to/during the new full FLOSS foundation release of KOA, a number of
new pieces of English documentation and functional specification must
be written.  We hope that the availability of such documentation and
specifications will provide additional motivation for electronic and
remote voting researchers and developers to seriously consider the KOA
system as a foundation for their work.  While some practitioners
cringe at the ``look-and-feel'' of formal specifications, they are
necessary for obtaining full-blown verification.  In response to this
challenge, exploratory research is underway to bridge the gap between
detailed specifications and higher-level, more readable domain specific
languages for voting.

We intend for KOA to be the first formally specified and verified
remote and local voting system available in the world, and furthermore
it is available under the GPL license.  It is unclear how to
compare such a system to the current commercial and FLOSS voting
systems being proposed by others, given that none of them, to our
knowledge, even write formal specifications, let alone perform
verification.  We hope that this work will inspire and challenge other
groups working on trustworthy voting.

%=====================================================================
\section{Acknowledgments}

This work is being supported by the European Project Mobius within the IST 6th
Framework. % national grants from the Science
% Foundation Ireland and Enterprise Ireland, and by the Irish Research
% Council for Science, Engineering and Technology.  
This paper reflects only the authors' views and the Community is not liable for
any use that may be made of the information contained therein.

%=====================================================================
\section{Availability}

The KOA system's home page is found at
{\url{secure.ucd.ie/products/opensource/KOA}}

%======================================================================
%% \nocite{ex1,ex2}
% \bibliographystyle{latex8}
{\footnotesize \bibliographystyle{acm}
  \bibliography{abbrev,ads,category,complexity,hypertext,icsr,knowledge,languages,linguistics,meta,metrics,misc,modeling,modeltheory,reuse,rewriting,softeng,specification,ssr,technology,theory,web,upcoming,upcoming_conferences,conferences,workshops,verification,escjava,jml,nijmegen,paper}
}

%\theendnotes

%======================================================================
% Fin

\end{document}

%%% Local Variables:
%%% mode: latex
%%% TeX-master: t
%%% End:
