
% \paragraph{Formal Diagrams}

% \paragraph{Inheritance}

% \paragraph{Client-Supplier Relationships}

% \paragraph{Scenarios}

% Note how JML expresses the medium- to low-level design.  It would be
% beneficial if the requirement could be expressed at a higher level,
% without loss of formality or accuracy, so that it could be reviewed by
% a domain expert or business analyst and yet is unambiguous for the
% software engineer.  This is a dilemma which occurs frequently in large
% complex software engineering projects.

% % Without the ability to execute the specification directly, it cannot
% % be proven to be complete until an implementation is developed.

% \paragraph{Resulting design changes.} %Patrick

%=====================================================================    
% \section{High-level System Specifications for Trustworthy Electronic Voting}

% \paragraph{A need for standard terminology of the domain.}

% McGaley's work~\cite{McGaley06}

% %~~~~~~~~~~~~~~~~~~~~~~~~~~~~~~~~~~~~~~~~~~~~~~~~~~~~~~~~~~~~~~~~~~~~~
% \subsection{Generic System Specifications}

% \todo discuss use of B and VDM

% %~~~~~~~~~~~~~~~~~~~~~~~~~~~~~~~~~~~~~~~~~~~~~~~~~~~~~~~~~~~~~~~~~~~~~
% \subsection{Domain Specific Languages}

% \paragraph{Sketch of a DSL for e-Voting.}

% It would be desirable that a high level specification would have these
% three properties:
% \begin{enumerate}
% \item it is mathematically complete and logically sound
% \item it is readable by a domain expert or business analyst
% \item it is executable by a software tool and could be used either to
%   generate or to verify the source code
% \end{enumerate}

% The first and third requirement above could be proven if the
% specification could be transformed into JML annotated Java with JML
% specifications.

% The ability to read and understand a high level specification does not
% imply the ability to write one.  Most people can successfully read a
% novel, only a minority of people can successfully write a novel.
% Likewise the high level specification would most likely be written by
% a software engineer in consultation with a domain expert or business
% analyst.

% The second requirement can be met by using a domain specific language
% (DSL).  We illustrate this by expressing the invariants in formal
% English:
% \begin{verbatim}
%    total fractional votes equals 
%    total rounded fractional votes 
%    during each count

%    when fractional vote less than threshold 
%    then round up otherwise round down
% \end{verbatim}

% This expression is awkward and could be restated as follows:
% \begin{itemize}
% \item if fraction A is greater than fraction B and fraction A is
%   rounded up, then fraction B is also rounded up
% \item if (A>B) round (A) > round (B)
% \end{itemize}

% However, the imperative procedure is more intuitive:
% \begin{itemize}
% \item if we have a remainder of N then roundup the N highest fractions
%   in the set
% \end{itemize}

% Mathematically this can de defined as a mapping from an ordered set of
% rational numbers to an ordered set of 1s and 0s.

% In DSL pseudo code it might look like this:
% \begin{itemize}
% \item define a fractional number as a pair of integers, the numerator
%   and denominator
% \item method Round Fractional Votes on array of fractional numbers:
%   \begin{itemize}
%   \item sort the array in descending order
%   \item calculate remainder = sum of fractions
%   \item starting at highest fraction not yet rounded up: if number
%     rounded so far is less than remainder then round up and go to next
%     highest otherwise stop
%   \end{itemize}
% \item sort the array in descending order
% \item calculate remainder = sum of fractions
% \item starting at highest fraction not yet rounded up: if number
%   rounded so far is less than remainder then round up and go to next
%   highest otherwise stop
% \end{itemize}

% If we express the DSL in a ruby like syntax, with declarative
% functions and imperative procedures it might look like this:
% \footnotesize
% \begin{verbatim}
%   definition fraction is 
%              integer numerator, 
%              integer denominator.

%   function compare (fraction X, fraction Y) 
%            returns boolean is
%            (numerator of X * denominator of Y) >
%            (numerator of Y * denominator of X).

%   function sum (fraction X, fraction Y) 
%            returns fraction is
%            numerator of result = 
%            (numerator of X * denominator of Y) + 
%            (numerator of Y * denominator of X),
%            denominator of result = 
%            denominator of X * denominator of Y.

%   function sum (array A of fraction) 
%            returns fraction is
%            sum (top of A, tail of A). 

%   comment top and tail are built in functions. 

%   procedure rounding (array A of fraction) does
%             B = sort array of fraction by 
%             compare (fraction, fraction),
%             calculate total = sum (B),
%             calculate length = count (B),
%             if total >= length then 
%               result = (1, rounding (tail B))
%             else result = (array of 0).
% \end{verbatim}
% \normalsize

% This an only an example of what a domain specific language might look
% like.  At first glance, it might not appear to discuss voting concepts
% at all, but we need to be able to discuss the rounding algorithm in
% precise detail because this will determine the allocation of second
% and subsequent preferences in a PRSTV system.

% The declarative part of the specification can easily be transformed
% into JML.  Now we consider the imperative part.

% The alternative to the generation of a JML specification would be to
% generate Java annotated with JML.

% It seems more natural to refine from an imperative specification to an
% imperative language, rather than turning the specification fragment
% inside out i.e. convoluting the logic by attempting to express it in a
% declarative format.
     
% We believe that adopting this methodology will overcome the
% awkwardness of the original specification.

% \paragraph{Overview of proposed DSL}

% Keywords identified by McGaley et al include ballot, cast, voter,
% eligible voter, evoting system, polling period, vote, voter register,
% voting system and voting channel.

% %=====================================================================
% \section{Ongoing Work}
% \label{sec:acad-curr-work}

% \todo introduction

% %---------------------------------------------------------------------
% \subsection{Generalisation of System for other\\Voting Systems} 
% \label{subsubsec:gener-syst-non}

% The Java code for \Votail was written in JML using a kind of
% ``verification-centric'' Design by Contract methodology.  This means
% that not only are we writing each method implementation according to
% its JML specification, but we are checking each method's correctness
% with ESC/Java2 and automatically generating thousands of unit tests
% using JMLUnit~\cite{Cheon-Leavens02}.

% The KOA system has a state machine similar to that used in the \Votail
% specification.  This allows KOA to make calls to the appropriate part
% of the \Votail code.  The \texttt{ElectionAlgorithm} class in \Votail
% will be invoked from within the KOA system using the following four
% method calls: \texttt{setup}, which defines election parameters such
% as candidate list and number of seats, \texttt{load}, which loads all
% valid ballots and then calculate quota and deposit saving thresholds,
% \texttt{count}, which assign votes to candidates, distribute surpluses
% and exclude candidates until finished, and \texttt{report}, which
% reports the election results.  These methods must be called in the
% order shown, and this fact is captured by the invariants of the state
% machine.  Only the \texttt{report} method is called more than once for
% each instance of the \texttt{ElectionAlgorithm} class.

% The user interface is being designed in a flexible fashion so as to
% present non-Dutch ballot papers to the voter.  The original KOA system
% was designed for use with a party-list system with a single national
% constituency.  Its user interface is being extended in line with the
% guidelines for the Irish voting system.  The KOA system allows the
% voter to select a list of candidates; in the Irish system each
% candidate is in a list of one.  The KOA system allows only one
% selection by the voter; in the Irish system. the voter makes multiple
% selections in order of preference.

% %~~~~~~~~~~~~~~~~~~~~~~~~~~~~~~~~~~~~~~~~~~~~~~~~~~~~~~~~~~~~~~~~~~~~~
% \subsection{State of the Current Implementation of \Votail}

% \paragraph{Current test coverage.} %Patrick

% \todo implementation issues: Joe, Patrick

%=====================================================================
%\section{Future Work}
%\label{sec:future-work}

%Several pieces of future work have been identified and some of them
%are currently underway by researchers at UCD.

% %~~~~~~~~~~~~~~~~~~~~~~~~~~~~~~~~~~~~~~~~~~~~~~~~~~~~~~~~~~~~~~~~~~~~~
% \subsection{Specification of Security Properties} % 

% The EU Mobius Project~\cite{MOBIUS}, of which UCD is a member, focuses
% on several topics, including the specification and verification of
% security properties at several levels.

% The domain of voting provides an excellent security case study as a
% relatively small number of concrete security properties are embodied
% in such a system.  We have identified some of these properties,
% formally specify them at a high-level in EBON, and are now using this
% methodology to translate these high-level specifications into
% concrete, low-level specifications and type-annotations so that we can
% formally verify these properties in KOA.

% Additionally, a MIDP-based remote voting applet has been developed at
% UCD by a final year student.  We intend to review this application for
% possible incorporation into KOA, using the same verification-centric
% approach used in the rest of the KOA work.

% %~~~~~~~~~~~~~~~~~~~~~~~~~~~~~~~~~~~~~~~~~~~~~~~~~~~~~~~~~~~~~~~~~~~~~
% \subsection{Full-blown Verification} % 

% We intend to fully specify and verify critical subsystems of the KOA
% system as a case study for the new MOBIUS Integrated Verification
% Environment (IVE) that is being developed by UCD and others.

% This goal is much more ambitious than simply performing extended
% static checking on various critical classes.

% %~~~~~~~~~~~~~~~~~~~~~~~~~~~~~~~~~~~~~~~~~~~~~~~~~~~~~~~~~~~~~~~~~~~~~
% \subsection{Just-in-Time Deployment with PCC} % Joe/Alan

% One of the primary problems with electronic voting systems is that new
% software updates, at both operating system and application levels, are
% typically installed in the field without any certification.  One
% technology that can help solve this deployment issue is Proof-Carrying
% Code (PCC)~\cite{Necula97}, the primary underlying formal foundation
% and technology used by the MOBIUS IVE.

% Using a PCC technology foundation, new system and application patches
% could be just-in-time deployed to the thousands of voting machines
% used in an election with complete assurance.  Developing such a
% foundation is part of the MOBIUS project's mandate, so the KOA system
% may be used as a deployment case study in the coming years.

% %~~~~~~~~~~~~~~~~~~~~~~~~~~~~~~~~~~~~~~~~~~~~~~~~~~~~~~~~~~~~~~~~~~~~~
% \subsection{American Voting Systems} % Joe/Alan

% American voting systems are the focus of an intense amount of
% discussion and work, given the ongoing fiasco in electronic voting we
% have witnessed in the U.S.~over the past several years.

% After integrating the \Votail Irish voting subsystem, we intend on
% formally specifying and verifying an American presidential voting
% subsystem using the same verification-centric methodology we have
% followed thus far.

% %~~~~~~~~~~~~~~~~~~~~~~~~~~~~~~~~~~~~~~~~~~~~~~~~~~~~~~~~~~~~~~~~~~~~~
% \subsection{Electronic Voting Systems} % Joe/Alan

% A voting-system independent, formally specified and verified remote
% voting system can be used in an electronic voting system, as the
% latter is just a trivial version of the former.  It is our intention
% to build and demonstrate such a system, incorporating a new formally
% specified and verified voter-verifiable paper trail subsystem.

% %~~~~~~~~~~~~~~~~~~~~~~~~~~~~~~~~~~~~~~~~~~~~~~~~~~~~~~~~~~~~~~~~~~~~~
% \subsection{Trustworthy Cryptographic Voting Algorithms}

% \todo discuss work starting with Peter Ryan

